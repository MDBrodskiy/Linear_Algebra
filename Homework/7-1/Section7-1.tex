%%%%%%%%%%%%%%%%%%%%%%%%%%%%%%%%%%%%%%%%%%%%%%%%%%%%%%%%%%%%%%%%%%%%%%%%%%%%%%%%%%%%%%%%%%%%%%%%%%%%%%%%%%%%%%%%%%%%%%%%%%%%%%%%%%%%%%%%%%%%%%%%%%%%%%%%%%%%%%%%%%%%%%%%%%%%%%%%%%%%%%%%%%%%
% Written By Michael Brodskiy
% Class: Linear Algebra
% Professor: L. Knight
%%%%%%%%%%%%%%%%%%%%%%%%%%%%%%%%%%%%%%%%%%%%%%%%%%%%%%%%%%%%%%%%%%%%%%%%%%%%%%%%%%%%%%%%%%%%%%%%%%%%%%%%%%%%%%%%%%%%%%%%%%%%%%%%%%%%%%%%%%%%%%%%%%%%%%%%%%%%%%%%%%%%%%%%%%%%%%%%%%%%%%%%%%%%

\documentclass[12pt]{article} 
\usepackage{alphalph}
\usepackage[utf8]{inputenc}
\usepackage[russian,english]{babel}
\usepackage{titling}
\usepackage{amsmath}
\usepackage{graphicx}
\usepackage{enumitem}
\usepackage{amssymb}
\usepackage{physics}
\usepackage{tikz}
\usepackage{mathdots}
\usepackage{yhmath}
\usepackage{cancel}
\usepackage{color}
\usepackage{siunitx}
\usepackage{array}
\usepackage{multirow}
\usepackage{gensymb}
\usepackage{tabularx}
\usepackage{booktabs}
\usepackage{pifont}
\newcommand{\xmark}{\ding{55}}
\usetikzlibrary{fadings}
\usetikzlibrary{patterns}
\usetikzlibrary{shadows.blur}
\usetikzlibrary{shapes}
\usepackage[super]{nth}
\usepackage{expl3}
\usepackage[version=4]{mhchem}
\usepackage{hpstatement}
\usepackage{rsphrase}
\usepackage{everysel}
\usepackage{ragged2e}
\usepackage{geometry}
\usepackage{fancyhdr}
\usepackage{cancel}
\usepackage{multicol}
\geometry{top=1.0in,bottom=1.0in,left=1.0in,right=1.0in}
\newcommand{\subtitle}[1]{%
  \posttitle{%
    \par\end{center}
    \begin{center}\large#1\end{center}
    \vskip0.5em}%

}
\usepackage{hyperref}
\hypersetup{
colorlinks=true,
linkcolor=blue,
filecolor=magenta,      
urlcolor=blue,
citecolor=blue,
}

\urlstyle{same}


\title{Linear Algebra 7.1 Homework}
\date{}
\author{Michael Brodskiy\\ \small Instructor: Prof. Knight}

\begin{document}

\maketitle

\begin{enumerate}

    \begin{center}
      \underline{1-5 odd, 9, 12, 15-27 odd, 41, 47, 60, 78}
    \end{center}

  \item $\begin{bmatrix} 2 & 0\\ 0 & -2\end{bmatrix}\begin{bmatrix} 1\\0\end{bmatrix}=2\begin{bmatrix} 1\\0\end{bmatrix}\text{ and }\begin{bmatrix} 2 & 0\\ 0 & -2 \end{bmatrix}\begin{bmatrix} 0\\1 \end{bmatrix}=-2\begin{bmatrix} 0\\1\end{bmatrix}$

    \setcounter{enumi}{2}

  \item $\begin{bmatrix} 2 & 3 & 1\\ 0 & -1 & 2\\ 0 & 0 & 3\end{bmatrix}\begin{bmatrix} 1 & 0 & 0\end{bmatrix}=2\begin{bmatrix} 1\\0\\0\end{bmatrix}\text{ and }\begin{bmatrix} 2 & 3 & 1\\ 0 & -1 & 2\\ 0 & 0 & 3\end{bmatrix}\begin{bmatrix} 1\\-1\\0\end{bmatrix}=-1\begin{bmatrix} 1\\-1\\0\end{bmatrix}\text{ and }\begin{bmatrix} 2 & 3 & 1\\ 0 & -1 & 2\\ 0 & 0 & 3\end{bmatrix}\begin{bmatrix}5\\1\\2 \end{bmatrix}=3\begin{bmatrix}5\\1\\2\end{bmatrix}$

    \setcounter{enumi}{4}

  \item $\begin{bmatrix} 0 & 1 & 0\\ 0 & 0 & 1\\ 1 & 0 & 0\end{bmatrix}\begin{bmatrix} 1\\1\\1\end{bmatrix}=\begin{bmatrix}1\\1\\1\end{bmatrix}$

    \setcounter{enumi}{8}

  \item 

    \begin{enumerate}

      \item $\begin{bmatrix} 7 & 2\\ 2 & 4\end{bmatrix}\begin{bmatrix} 1\\2\end{bmatrix}=\begin{bmatrix} 11\\ 10\end{bmatrix}$. Not an eigenvector.

      \item $\begin{bmatrix} 7 & 2\\ 2 & 4\end{bmatrix}\begin{bmatrix} 2\\1\end{bmatrix}=8\begin{bmatrix} 2\\1\end{bmatrix}$. This is an eigenvector.

      \item $\begin{bmatrix} 7 & 2\\ 2 & 4\end{bmatrix}\begin{bmatrix} 1\\-2\end{bmatrix}=-3\begin{bmatrix}1\\-2\end{bmatrix}$. This is an eigenvector.

      \item Because 2 eigenvectors were already found, this is not an eigenvector.

    \end{enumerate}

    \setcounter{enumi}{11}

  \item 

    \begin{enumerate}

      \item $\begin{bmatrix} 1 & 0 & 5\\ 0 & -2 & 4\\ 1 & -2 & 9\end{bmatrix}\begin{bmatrix} 1\\1\\0\end{bmatrix}=\begin{bmatrix} 1\\-2\\-1\end{bmatrix}$. Not an eigenvector.

      \item $\begin{bmatrix} 1 & 0 & 5\\ 0 & -2 & 4\\ 1 & -2 & 9\end{bmatrix}\begin{bmatrix} -5\\2\\1\end{bmatrix}=0\begin{bmatrix} -5\\2\\1\end{bmatrix}$. This is an eigenvector.

      \item $\begin{bmatrix} 1 & 0 & 5\\ 0 & -2 & 4\\ 1 & -2 & 9\end{bmatrix}\begin{bmatrix} 0\\0\\0\end{bmatrix}=\lambda\begin{bmatrix} 0\\0\\0\end{bmatrix}$. This is not an eigenvector. The zero vector can not be an eigenvector.

      \item $\begin{bmatrix} 1 & 0 & 5\\ 0 & -2 & 4\\ 1 & -2 & 9\end{bmatrix}\begin{bmatrix} 2\sqrt{6}-3\\-2\sqrt{6}+6\\3\end{bmatrix}=\begin{bmatrix} 2\sqrt{6}+12\\4\sqrt{6}\\6\sqrt{6}+3 \end{bmatrix}$. This is not an eigenvector.

    \end{enumerate}

    \setcounter{enumi}{14}

  \item

    \begin{enumerate}

      \item $|\lambda I-A|=0\Rightarrow (\lambda-6)(\lambda-1)-6\Rightarrow \lambda^2-7\lambda\Rightarrow \lambda(\lambda-7)=0$

      \item $\lambda_1=0\Rightarrow\bold{x}_{\lambda_1}=\left[\begin{array}{cc|c} -6 & 3 & 0\\2 & -1 & 0  \end{array}\right]\Rightarrow (t, 2t)\text{ and }\lambda_2=7\Rightarrow\bold{x}_{\lambda_2}=\left[\begin{array}{cc|c} 1 & 3 & 0\\2 & 6 & 0  \end{array}\right]\Rightarrow (-3t, t)$

    \end{enumerate}

    \setcounter{enumi}{16}

  \item

    \begin{enumerate}

      \item $(\lambda-1)^2-4\Rightarrow \lambda^2-2\lambda-3\Rightarrow (\lambda-3)(\lambda+1)=0$

      \item $\lambda_1=3\Rightarrow \bold{x}_{\lambda_1}=\left[ \begin{array}{cc|c} 2 & 2 & 0\\ 2 & 2 & 0  \end{array} \right]\Rightarrow (t,-t)\text{ and }\lambda_2=-1\Rightarrow \bold{x}_{\lambda_2}=\left[ \begin{array}{cc|c} -2 & 2 & 0\\ 2 & -2 & 0 \end{array} \right]\Rightarrow (t,t)$

    \end{enumerate}

    \setcounter{enumi}{18}

  \item

    \begin{enumerate}

      \item $(\lambda-1)(\lambda+1)+\frac{3}{4}\Rightarrow \lambda^2-\frac{1}{4}\Rightarrow \left( \lambda-\frac{1}{2} \right)\left( \lambda+\frac{1}{2} \right)=0$

      \item $\lambda_1=\frac{1}{2}\Rightarrow \bold{x}_{\lambda_1}=\left[ \begin{array}{cc|c} -.5 & 1.5 & 0\\ -.5 & 1.5 & 0  \end{array} \right]\Rightarrow (3t,t)\text{ and }\lambda_2=-\frac{1}{2}\Rightarrow \bold{x}_{\lambda_2}=\left[ \begin{array}{cc|c} -1.5 & 1.5 & 0\\ -.5 & .5 & 0 \end{array} \right]\Rightarrow (t,t)$

    \end{enumerate}

    \setcounter{enumi}{20}

  \item

    \begin{enumerate}

      \item $(\lambda-2)\left[ (\lambda-3)(\lambda-2)-2 \right]\Rightarrow (\lambda-2)\left[ \lambda^2-5\lambda+4 \right]\Rightarrow (\lambda-2)(\lambda-4)(\lambda-1)=0$

      \item $\lambda_1=1\Rightarrow \bold{x}_{\lambda_1}=\left[ \begin{array}{ccc|c} -1 & 2 & -3 & 0\\ 0 & -2 & 2 & 0\\ 0 & 1 & -1 & 0   \end{array} \right]\Rightarrow (-t,t,t)\text{ and }\lambda_2=2\Rightarrow \bold{x}_{\lambda_2}=\left[ \begin{array}{ccc|c} 0 & 2 & -3 & 0\\ 0 & 1 & 2 & 0\\ 0 & 1 & 0 & 0 \end{array} \right]\Rightarrow (t,0,0)\text{ and }\lambda_3=4\Rightarrow \bold{x}_{\lambda_3}=\left[ \begin{array}{ccc|c} 2 & 2 & -3 & 0\\ 0 & 1 & 2 & 0\\ 0 & 1 & 2 & 0   \end{array} \right]\Rightarrow (3.5t,-2t,t)$

    \end{enumerate}

    \setcounter{enumi}{22}

  \item 

    \begin{enumerate}

      \item $(\lambda-1)\left[ (\lambda-5)(\lambda+3)+12 \right]+2\left[ 2(\lambda+3)-12 \right]+2\left[ -12+6(\lambda-5) \right]\Rightarrow (\lambda-1)(\lambda-3)(\lambda+1)+4\lambda-12+12\lambda-84\Rightarrow(\lambda-1)(\lambda-3)(\lambda+1)+16\lambda-96\Rightarrow (\lambda^2-1)(\lambda-3)+16\lambda-96\Rightarrow \lambda^3-3\lambda^2-\lambda+3+16\lambda-96\Rightarrow \lambda^3-3\lambda^2+15\lambda-99=0\Rightarrow (\lambda-3)^2(\lambda+3)=0$

      \item $\lambda_1=3\Rightarrow \bold{x}_{\lambda_1}=\left[ \begin{array}{ccc|c} 2 & -2 & 2 & 0\\ 2 & -2 & 2 & 0\\ 6 & -6 & 6 & 0   \end{array} \right]\Rightarrow (s-t,s,t)\text{ and }\lambda_2=-3\Rightarrow \bold{x}_{\lambda_2}=\left[ \begin{array}{ccc|c} -4 & -2 & 2 & 0\\ 2 & -8 & 2 & 0\\ 6 & -6 & 0 & 0 \end{array} \right]\Rightarrow (t,t,3t)$

    \end{enumerate}

    \setcounter{enumi}{24}

  \item

    \begin{enumerate}

      \item $(\lambda-4)\left[ \lambda(\lambda-4)-12 \right]\Rightarrow (\lambda-4)(\lambda-6)(\lambda+2)=0$

      \item $\lambda_1=4\Rightarrow \bold{x}_{\lambda_1}=\left[ \begin{array}{ccc|c} 4 & 3 & -5 & 0\\ 4 & 0 & 10 & 0\\ 0 & 0 & 0 & 0   \end{array} \right]\Rightarrow (-\frac{5}{2}t,5t,t)\text{ and }\lambda_2=6\Rightarrow \bold{x}_{\lambda_2}=\left[ \begin{array}{ccc|c} 6 & 3 & -5 & 0\\ 4 & 2 & 10 & 0\\ 0 & 0 & 2 & 0 \end{array} \right]\Rightarrow (t,-2t,0)\text{ and }\lambda_3=-2\Rightarrow \bold{x}_{\lambda_3}=\left[ \begin{array}{ccc|c} -2 & 3 & -5 & 0\\ 4 & -6 & 10 & 0\\ 0 & 0 & -6 & 0   \end{array} \right]\Rightarrow (\frac{3}{2}t,t,0)$

    \end{enumerate}

    \setcounter{enumi}{26}

  \item

    \begin{enumerate}

      \item $(\lambda-2)(\lambda-2)\left[ \lambda(\lambda-3)-4 \right]\Rightarrow (\lambda-2)^2(\lambda-4)(\lambda+1)$

      \item $\lambda_1=2\Rightarrow \bold{x}_{\lambda_1}=\left[ \begin{array}{cccc|c} 0 & 0 & 0 & 0 & 0\\ 0 & 0 & 0 & 0 & 0\\ 0 & 0 & -1 & -1 & 0\\ 0 & 0 & -4 & 2 & 0   \end{array} \right]\Rightarrow (s,t,0,0)\text{ and }\lambda_2=4\Rightarrow \bold{x}_{\lambda_2}=\left[ \begin{array}{cccc|c} 2 & 0 & 0 & 0 & 0\\ 0 & 2 & 0 & 0 & 0\\ 0 & 0 & 1 & -1 & 0\\ 0 & 0 & -4 & 4 & 0 \end{array} \right]\Rightarrow (0,0,t,t)\text{ and }\lambda_3=-1\Rightarrow \bold{x}_{\lambda_3}=\left[ \begin{array}{cccc|c} -3 & 0 & 0 & 0 & 0\\ 0 & -3 & 0 & 0 & 0\\ 0 & 0 & -4 & -1 & 0\\ 0 & 0 & -4 & -1& 0   \end{array} \right]\Rightarrow (0,0,t,-4t)$

    \end{enumerate}

    \setcounter{enumi}{40}

  \item $\lambda_i=2,3,1$

    \setcounter{enumi}{46}

  \item 

    \begin{enumerate}

      \item $(\lambda+1)\left[ \lambda(\lambda-3)+2 \right]\Rightarrow (\lambda+1)(\lambda-2)(\lambda-1)$, so $\lambda=-1,2,1$

      \item $\lambda=-1\Rightarrow \left[ \begin{array}{ccc|c} -1 & -2 & 1 & 0\\ 1 & -4 & -1 & 0\\ 0 & 0 & 0 & 0  \end{array} \right]\Rightarrow B_1=\left\{ (1,0,1) \right\}\text{ and }\lambda=1\Rightarrow\left[ \begin{array}{ccc|c} 1 & -2 & 1 & 0\\ 1 & -2 & 1& 0\\ 0 & 0 & 2 & 0  \end{array} \right]\Rightarrow B_2=\left\{ 2,1, 0\right\}\text{ and }\lambda=2\Rightarrow\left[ \begin{array}{ccc|c} 2 & -2 & 1 & 0\\ 1 & -1 & -1 & 0\\ 0 & 0 & 3 & 0  \end{array} \right]\Rightarrow B_3=\left\{(1,1,0)\right\}$

      \item $\begin{bmatrix} -1 &0&0\\ 0&1&0\\0&0&2\end{bmatrix}$

    \end{enumerate}

    \setcounter{enumi}{59}

  \item proj$_{\bold{u}}\bold{v}=\frac{\bold{u}\cdot\bold{v}}{\bold{v}\cdot\bold{v}}\bold{v}$

    \setcounter{enumi}{77}

  \item $|\lambda I-A|=(\lambda-\cos\theta)(\lambda-\cos\theta)+\sin^2\theta=\lambda^2-2\lambda\cos\theta+\cos^2\theta+\sin^2\theta=$\\$\lambda^2-2\lambda\cos\theta+1\Rightarrow \theta=\cos^{-1}\left( \frac{\lambda^2+1}{2\lambda} \right)$

\end{enumerate}

\end{document}

