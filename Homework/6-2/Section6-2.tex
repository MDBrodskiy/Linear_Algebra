%%%%%%%%%%%%%%%%%%%%%%%%%%%%%%%%%%%%%%%%%%%%%%%%%%%%%%%%%%%%%%%%%%%%%%%%%%%%%%%%%%%%%%%%%%%%%%%%%%%%%%%%%%%%%%%%%%%%%%%%%%%%%%%%%%%%%%%%%%%%%%%%%%%%%%%%%%%%%%%%%%%%%%%%%%%%%%%%%%%%%%%%%%%%
% Written By Michael Brodskiy
% Class: Linear Algebra
% Professor: L. Knight
%%%%%%%%%%%%%%%%%%%%%%%%%%%%%%%%%%%%%%%%%%%%%%%%%%%%%%%%%%%%%%%%%%%%%%%%%%%%%%%%%%%%%%%%%%%%%%%%%%%%%%%%%%%%%%%%%%%%%%%%%%%%%%%%%%%%%%%%%%%%%%%%%%%%%%%%%%%%%%%%%%%%%%%%%%%%%%%%%%%%%%%%%%%%

\documentclass[12pt]{article} 
\usepackage{alphalph}
\usepackage[utf8]{inputenc}
\usepackage[russian,english]{babel}
\usepackage{titling}
\usepackage{amsmath}
\usepackage{graphicx}
\usepackage{enumitem}
\usepackage{amssymb}
\usepackage{physics}
\usepackage{tikz}
\usepackage{mathdots}
\usepackage{yhmath}
\usepackage{cancel}
\usepackage{color}
\usepackage{siunitx}
\usepackage{array}
\usepackage{multirow}
\usepackage{gensymb}
\usepackage{tabularx}
\usepackage{booktabs}
\usepackage{pifont}
\newcommand{\xmark}{\ding{55}}
\usetikzlibrary{fadings}
\usetikzlibrary{patterns}
\usetikzlibrary{shadows.blur}
\usetikzlibrary{shapes}
\usepackage[super]{nth}
\usepackage{expl3}
\usepackage[version=4]{mhchem}
\usepackage{hpstatement}
\usepackage{rsphrase}
\usepackage{everysel}
\usepackage{ragged2e}
\usepackage{geometry}
\usepackage{fancyhdr}
\usepackage{cancel}
\usepackage{multicol}
\geometry{top=1.0in,bottom=1.0in,left=1.0in,right=1.0in}
\newcommand{\subtitle}[1]{%
  \posttitle{%
    \par\end{center}
    \begin{center}\large#1\end{center}
    \vskip0.5em}%

}
\usepackage{hyperref}
\hypersetup{
colorlinks=true,
linkcolor=blue,
filecolor=magenta,      
urlcolor=blue,
citecolor=blue,
}

\urlstyle{same}


\title{Linear Algebra 6.2 Homework}
\date{}
\author{Michael Brodskiy\\ \small Instructor: Prof. Knight}

\begin{document}

\maketitle

\begin{enumerate}

    \begin{center}
      \underline{1-27 odd, 31, 33, 39, 41, 44, 45, 47, 54, 55, 57}
    \end{center}

  \item ker$(T)=\mathbb{R}^3$ because $T$ is the zero transformation

    \setcounter{enumi}{2}

  \item ker$(T)$ is $(0,0,0,0)$, because $T(v)=0$ only when all terms are 0

    \setcounter{enumi}{4}

  \item $T(a_0+a_1x+a_2x^2+a_3x^3)=a_1+a_2\rightarrow a_1+a_2=0,$, so $a_1=-a_2$. Therefore, ker$(T)=\left\{a_0+a_1x-a_1x^2+a_3,\,\,a_0,a_1,a_3\text{ are real}\right\}$

    \setcounter{enumi}{6}

  \item $a_1+2a_2x=0\rightarrow T(a_0+0)=T(a_0)$, so ker$(T)=\left\{ a_0,\,\, a_0\text{ is real} \right\}$

    \setcounter{enumi}{8}

  \item $(x+2y,y-x)=(0,0)\rightarrow \left[ \begin{array}{cc|c} 1 & 2 & 0\\ -1 & 1 & 0 \end{array}\right]\widetilde{ }\left[ \begin{array}{cc|c} 1 & 2 & 0\\ 0 & 3 & 0  \end{array} \right]\widetilde{ }\left[ \begin{array}{cc|c} 1 & 0 & 0\\ 0 & 1 & 0 \end{array} \right]\rightarrow \ker(T)=(0,0)$

    \setcounter{enumi}{10}

  \item

    \begin{enumerate}

      \item $\begin{bmatrix} 1 & 2\\ 3 & 4\end{bmatrix}\begin{bmatrix} x_1\\x_2\end{bmatrix}=\begin{bmatrix} 0\\ 0\end{bmatrix}\rightarrow \left\{ \begin{array}{c} x_1+2x_2=0\\ 3x_1+4x_2=0\end{array}\rightarrow \left[ \begin{array}{cc|c} 1 & 2 & 0\\ 3 & 4 & 0 \end{array}\right]\widetilde{ }\left[ \begin{array}{cc|c} 1 & 2 & 0\\ 0 & -2 & 0  \end{array} \right]\widetilde{ }\left[ \begin{array}{cc|c} 1 & 0 & 0\\ 0 & 1 & 0\end{array} \right]$\\$\rightarrow \ker(T)=(0,0)$

      \item $(0,0)$ is in $\mathbb{R}^2$

    \end{enumerate}

    \setcounter{enumi}{12}

  \item

    \begin{enumerate}

      \item $\left[ \begin{array}{ccc|c} 1 & -1 & 2 & 0\\ 0 & 1 & 2 & 0 \end{array}\right]\widetilde{ }\left[ \begin{array}{ccc|c} 1 & 0 & 4 & 0\\ 0 & 1 & 2 & 0\end{array} \right]\Rightarrow\begin{bmatrix} x_1\\x_2\\x_3\end{bmatrix} =t\begin{bmatrix} -4\\-2\\1\end{bmatrix}\Rightarrow \text{span}\left\{ (-4,-2,1) \right\}$

      \item The column space of $A$ spans $\mathbb{R}^2$, so the range is $\mathbb{R}^2$

    \end{enumerate}

    \setcounter{enumi}{14}

  \item

    \begin{enumerate}

      \item $\left[ \begin{array}{cc|c} 1 & 3 & 0\\ -1 & -3 & 0\\ 2 & 2 & 0\end{array} \right]\widetilde{ }\left[ \begin{array}{cc|c} 1 & 3 & 0\\ 0 & 0 & 0\\ 0 & -4 & 0  \end{array} \right]\widetilde{ }\left[ \begin{array}{cc|c} 1 & 0 & 0\\ 0 & 0 & 0\\ 0 & 1 & 0\end{array} \right]\Rightarrow\ker(T)=\left\{ (0,0) \right\}$

      \item The column space simplifies to $\left[ \begin{array}{cc} 1 & 0\\ -1 & 0\\ 0 & 1\end{array} \right]$, so range$(T)=\text{span}\left\{ (1,-1,0),(0,0,1) \right\}$

    \end{enumerate}

    \setcounter{enumi}{16}

  \item

    \begin{enumerate}

      \item $\left[ \begin{array}{cccc|c} 1 & 2 & -1 & 4 & 0\\ 3 & 1 & 2 & -1 & 0\\ -4 & -3 & -1 & -3 & 0\\ -1 & -2 & 1 & 1 & 0  \end{array} \right]\widetilde{ }\left[ \begin{array}{cccc|c} 1 & 2 & -1 & 4 & 0\\ 0 & -5 & 5 & -13 & 0\\ 0 & 5 & -5 & 13 & 0\\ 0 & 0 & 0 & 5 & 0 \end{array} \right]\widetilde{ }\left[ \begin{array}{cccc|c} 1 & 0 & 1 & -\frac{6}{5} & 0\\ 0 & 1 & -1 & \frac{13}{5} & 0\\ 0 & 0 & 0 & 0 & 0\\ 0 & 0 & 0 & 1 & 0\\  \end{array} \right]\widetilde{ }$\\$\left[ \begin{array}{cccc|c} 1 & 0 & 1 & 0 & 0\\ 0 & 1 & -1 & 0 & 0\\ 0 & 0 & 0 & 0 & 0\\ 0 & 0 & 0 & 1 & 0 \end{array} \right]\Rightarrow\begin{bmatrix} x_1\\x_2\\x_3\\x_4\end{bmatrix}=x_3\begin{bmatrix} -1\\1\\1\\0\end{bmatrix}\Rightarrow\ker(T)=\text{span}\left\{ (-1,1,1,0) \right\}$

      \item range$(T)=\text{span}\left\{  (1,0,-1,0),(0,1,-1,0),(0,0,0,1)\right\}$

    \end{enumerate}

    \setcounter{enumi}{18}

  \item

    \begin{enumerate}

      \item $\left[ \begin{array}{cc|c} -1 & 1 & 0\\ 1 & 1 & 0\end{array} \right]\widetilde{ }\left[ \begin{array}{cc|c} 1 & 0 & 0\\ 0 & 1 & 0 \end{array}  \right]\Rightarrow \ker(T)=\left\{ (0,0) \right\}$

      \item nullity$(T)=\dim(\ker(T))=0$

      \item The column space spans $\mathbb{R}^2$

      \item $\rank(T)=n-\text{nullity}(T)=n=2$

    \end{enumerate}

    \setcounter{enumi}{20}

  \item

    \begin{enumerate}

      \item $\left[\begin{array}{cc|c} 5 & -3 & 0\\ 1 & 1 & 0\\ 1 & -1 & 0\end{array}\right]\widetilde{ }\left[ \begin{array}{cc|c} 1 & -7 & 0\\ 1 & 1 & 0\\ 0 & -2 & 0  \end{array} \right]\widetilde{ }\left[ \begin{array}{cc|c} 1 & 0 & 0\\ 0 & 1 & 0\\ 0 & 0 & 0\end{array} \right]\Rightarrow\ker(T)=\left\{ (0,0) \right\}$

      \item nullity$(T)=\dim(\ker(T))=0$

      \item Column space of $A\rightarrow$ $\text{span}\left\{ (5,1,1),(-3,1,-1) \right\}$

      \item $2-0=2$

    \end{enumerate}

    \setcounter{enumi}{22}

  \item

    \begin{enumerate}

      \item $\left[ \begin{array}{cc|c} .9 & .3 & 0\\ .3 & .1 & 0\end{array} \right]\widetilde{ }\left[ \begin{array}{cc|c} 1 & \frac{1}{3} & 0\\ 1 & \frac{1}{3} & 0\end{array} \right]\Rightarrow x_2=-3x_1\Rightarrow\ker(T)=\left\{ (x_1,-3x_1) \right\}$

      \item nullity$(T)=\dim(\ker(T))=1$

      \item Column space of $A\rightarrow\text{span}\left\{ (3x_2,x_2) \right\}$

      \item $2-1=1$

    \end{enumerate}

    \setcounter{enumi}{24}

  \item

    \begin{enumerate}

      \item $\left[ \begin{array}{ccc|c} 1 & 0 & 1 & 0\\ 0 & 1 &0 & 0\\ 1 & 0 & 1 & 0  \end{array} \right]\widetilde{ }\left[ \begin{array}{ccc|c} 1 & 0 & 1 & 0\\ 0 & 1 & 0 & 0\\ 0 & 0 & 0 & 0  \end{array} \right]\Rightarrow\ker(T)=\left\{ (-x_3,0,x_3) \right\}$

      \item nullity$(T)=\dim(\ker(T))=1$

      \item Column space of $A\rightarrow\text{span}\left\{ (s,t,s) \right\}$

      \item $3-1=2$

    \end{enumerate}

    \setcounter{enumi}{26}

  \item

    \begin{enumerate}

      \item $\left[ \begin{array}{ccc|c} 4 & -4 & 2 & 0\\ -4 & 4 & 2 & 0\\ 2 & -2 & 1 & 0  \end{array} \right]\widetilde{ }\left[ \begin{array}{ccc|c} 2 & -2 & 1 & 0\\ 0 & 0 & 0 & 0\\ 0 & 0 & 0 & 0  \end{array} \right]\Rightarrow\ker(T)=\text{span}\left\{ (s,t,2t-2s) \right\}$

      \item nullity$(T)=\dim(\ker(T))=2$

      \item Column space of $A\rightarrow\text{span}\left\{ (2t,-2t,t) \right\}$

      \item $3-2=1$

    \end{enumerate}

    \setcounter{enumi}{30}

  \item

    \begin{enumerate}

      \item  $\left[ \begin{array}{ccccc|c} 2 & 2 & -3 & 1 & 13 & 0\\ 1 & 1 & 1 & 1 & -1 & 0\\ 3 & 3 & -5 & 0 & 14 & 0\\ 6 & 6 & -2 & 4 & 16 & 0  \end{array}\right]\widetilde{ }\left[ \begin{array}{ccccc|c} 1 & 1 & -4 & 0 & 14 & 0\\ 0 & 0 & 5 & 1 & -15 & 0\\ 0 & 0 & -8 & -3 & 17 & 0\\ 0 & 0 & -8 & -2 & 22 & 0 \end{array}\right]\widetilde{ }\left[ \begin{array}{ccccc|c} 1 & 1 & -4 & 0 & 14 & 0\\ 0 & 0 & 5 & 1 & -15 & 0\\ 0 & 0 & -8 & -3 & 17 & 0\\ 0 & 0 & 0 & 1 & 5 & 0  \end{array}\right]\widetilde{ }$\\$\left[ \begin{array}{ccccc|c} 1 & 1 & -4 & 0 & 14 & 0\\ 0 & 0 & 1 & 0 & -4 & 0\\ 0 & 0 & 0 & 1 & 5 & 0\\ 0 & 0 & 0 & 0 & 0 & 0  \end{array}\right]\widetilde{ }\left[ \begin{array}{ccccc|c} 1 & 1 & 0 & 0 & -2 & 0\\ 0 & 0 & 1 & 0 & -4 & 0\\ 0 & 0 & 0 & 1 & 5 & 0\\ 0 & 0 & 0 & 0 & 0 & 0  \end{array}\right]\Rightarrow\begin{bmatrix} x_1\\x_2\\x_3\\x_4\\x_5\end{bmatrix}=s\begin{bmatrix} -1\\ 1\\0\\0\\0\end{bmatrix} + t\begin{bmatrix} 2\\ 0\\4\\-5\\1\end{bmatrix}$\\$\ker(T))=\text{span}\left\{ (2t-s,s,4t,-5t,t) \right\}$

      \item nullity$(T)=\dim(\ker(T))=2$

      \item Column space of $A\rightarrow\text{span}\left\{ (2,1,3,6),(1,1,0,4) \right\}$

      \item $5-2=3$

    \end{enumerate}

    \setcounter{enumi}{32}

  \item nullity$(T)=3-\rank(T)=1$. The dimension of the kernel is one, so it must be a line. The dimension of the range is two, so it must be a plane.

    \setcounter{enumi}{38}

  \item $\left[ \begin{array}{ccc|c} 1 & 2 & 2 & 0\\ 2 & 4 & 4 & 0\\ 2 & 4 & 4 & 0 \end{array}\right]\Rightarrow\ker(T)=\text{span}\left\{ (-2s-2t,s,t) \right\}$, so the nullity and dimension of the kernel is two. The kernel must be a plane, and the range must be a line.

    \setcounter{enumi}{40}

  \item $4-2=2$

    \setcounter{enumi}{43}

  \item $4-2=2$

  \item $8-4=4$

    \setcounter{enumi}{46}

  \item $|A|=-4$. This means it has trivial solution only, which signifies that $\ker(T)=\left\{ \bold{0} \right\}$. This means it is a one-to-one transformation. The dimension of the kernel is therefore 0, which means the rank is of dimension 2. This equals the dimension of $\mathb{R}^2$, which means it is onto.

    \setcounter{enumi}{53}

  \item $A$ has trivial solution only, so it is one-to-one.

  \item \begin{tabular}{c | c | c} Vector Space & Zero Vector & Standard Basis \\\hline $\mathbb{R}^4$ & (0,0,0,0) & $\left\{ (1,0,0,0),(0,1,0,0),(0,0,1,0),(0,0,0,1) \right\}$\\ $M_{4,1}$ & $\begin{bmatrix} 0\\0\\0\\0\end{bmatrix}$ & $\left\{ \begin{bmatrix} 1\\0\\0\\0\end{bmatrix},\begin{bmatrix}0\\1\\0\\0\end{bmatrix},\begin{bmatrix}0\\0\\1\\0\end{bmatrix},\begin{bmatrix}0\\0\\0\\1\end{bmatrix} \right\}$\\ $M_{2,2}$ & $\begin{bmatrix} 0 & 0\\0 & 0\end{bmatrix}$ & $\left\{ \begin{bmatrix} 1 & 0\\0 & 0\end{bmatrix},\begin{bmatrix} 0 & 1\\ 0 & 0\end{bmatrix},\begin{bmatrix} 0 & 0\\ 1 & 0\end{bmatrix},\begin{bmatrix} 0 & 0\\ 0 & 1\end{bmatrix} \right\}$ \\ $P_3$ & $0+0x+0x^2+0x^3$ & $\left\{ 1,x,x^2,x^3 \right\}$ \\ $V$ & $(0,0,0,0,0)$ & $\left\{ (1,0,0,0,0),(0,1,0,0,0),(0,0,1,0,0),(0,0,0,1,0) \right\}$  \end{tabular}

    \setcounter{enumi}{56}

  \item $p'=0\Rightarrow\int p'\,dx=0\Rightarrow p(x)=c$, so any constant number

\end{enumerate}

\end{document}

