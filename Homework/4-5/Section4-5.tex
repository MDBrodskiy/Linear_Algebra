%%%%%%%%%%%%%%%%%%%%%%%%%%%%%%%%%%%%%%%%%%%%%%%%%%%%%%%%%%%%%%%%%%%%%%%%%%%%%%%%%%%%%%%%%%%%%%%%%%%%%%%%%%%%%%%%%%%%%%%%%%%%%%%%%%%%%%%%%%%%%%%%%%%%%%%%%%%%%%%%%%%%%%%%%%%%%%%%%%%%%%%%%%%%
% Written By Michael Brodskiy
% Class: Linear Algebra
% Professor: L. Knight
%%%%%%%%%%%%%%%%%%%%%%%%%%%%%%%%%%%%%%%%%%%%%%%%%%%%%%%%%%%%%%%%%%%%%%%%%%%%%%%%%%%%%%%%%%%%%%%%%%%%%%%%%%%%%%%%%%%%%%%%%%%%%%%%%%%%%%%%%%%%%%%%%%%%%%%%%%%%%%%%%%%%%%%%%%%%%%%%%%%%%%%%%%%%

\documentclass[12pt]{article} 
\usepackage{alphalph}
\usepackage[utf8]{inputenc}
\usepackage[russian,english]{babel}
\usepackage{titling}
\usepackage{amsmath}
\usepackage{graphicx}
\usepackage{enumitem}
\usepackage{amssymb}
\usepackage{physics}
\usepackage{tikz}
\usepackage{mathdots}
\usepackage{yhmath}
\usepackage{cancel}
\usepackage{color}
\usepackage{siunitx}
\usepackage{array}
\usepackage{multirow}
\usepackage{gensymb}
\usepackage{tabularx}
\usepackage{booktabs}
\usepackage{pifont}
\newcommand{\xmark}{\ding{55}}
\usetikzlibrary{fadings}
\usetikzlibrary{patterns}
\usetikzlibrary{shadows.blur}
\usetikzlibrary{shapes}
\usepackage[super]{nth}
\usepackage{expl3}
\usepackage[version=4]{mhchem}
\usepackage{hpstatement}
\usepackage{rsphrase}
\usepackage{everysel}
\usepackage{ragged2e}
\usepackage{geometry}
\usepackage{fancyhdr}
\usepackage{cancel}
\usepackage{multicol}
\geometry{top=1.0in,bottom=1.0in,left=1.0in,right=1.0in}
\newcommand{\subtitle}[1]{%
  \posttitle{%
    \par\end{center}
    \begin{center}\large#1\end{center}
    \vskip0.5em}%

}
\usepackage{hyperref}
\hypersetup{
colorlinks=true,
linkcolor=blue,
filecolor=magenta,      
urlcolor=blue,
citecolor=blue,
}

\urlstyle{same}


\title{Linear Algebra 4.5 Homework}
\date{}
\author{Michael Brodskiy\\ \small Instructor: Prof. Knight}

\begin{document}

\maketitle

\begin{enumerate}

    \begin{center}
    \underline{Problems 1-6, 10, 13, 15, 18, 22, 23, 25, 33, 37, 41, 45, 47, 67, 73, 78, 81}
    \end{center}

  \item $\left\{ (1,0,0,0,0,0),(0,1,0,0,0,0),(0,0,1,0,0,0),(0,0,0,1,0,0),(0,0,0,0,1,0),(0,0,0,0,0,1) \right\}$

  \item $\left\{ (1,0,0,0),(0,1,0,0),(0,0,1,0),(0,0,0,1) \right\}$

\item

    \begin{equation*}
      \begin{split}
        \begin{Bmatrix}\begin{bmatrix}1&0&0\\0&0&0\\0&0&0\end{bmatrix},&\begin{bmatrix}0&1&0\\0&0&0\\0&0&0\end{bmatrix},&\begin{bmatrix}0&0&1\\0&0&0\\0&0&0\end{bmatrix},\\\begin{bmatrix}0&0&0\\1&0&0\\0&0&0\end{bmatrix},&\begin{bmatrix}0&0&0\\0&1&0\\0&0&0\end{bmatrix},&\begin{bmatrix}0&0&0\\0&0&1\\0&0&0\end{bmatrix},\\\begin{bmatrix}0&0&0\\0&0&0\\1&0&0\end{bmatrix},&\begin{bmatrix}0&0&0\\0&0&0\\0&1&0\end{bmatrix},&\begin{bmatrix}0&0&0\\0&0&0\\0&0&1\end{bmatrix}\end{Bmatrix}
\end{split}
\label{1}
\end{equation}

  \item $\left\{ \begin{bmatrix}1\\0\\0\\0\end{bmatrix},\begin{bmatrix}0\\1\\0\\0\end{bmatrix},\begin{bmatrix}0\\0\\1\\0\end{bmatrix},\begin{bmatrix}0\\0\\0\\1\end{bmatrix} \right\}$

  \item $\left\{ 1,x,x^2,x^3,x^4 \right\}$

  \item $\left\{ 1,x,x^2 \right\}$

    \setcounter{enumi}{9}

  \item The set does not span $\mathbb{R}^2$ so it is not a basis

    \setcounter{enumi}{12}

  \item It is not basis because it is linearly dependent and does not span $\mathbb{R}^2$

    \setcounter{enumi}{14}

  \item The set is linearly dependent, and does not span $\mathbb{R}^3$, so it is not a basis for it

    \setcounter{enumi}{17}

  \item The set does not span $\mathbb{R}^3$

    \setcounter{enumi}{21}

  \item The set is linearly dependent

  \item The set is linearly dependent

    \setcounter{enumi}{24}

  \item The set does not span $P_2$

    \setcounter{enumi}{32}

  \item The set is linearly independent

    \setcounter{enumi}{36}

  \item Not a basis because it is linearly dependent

    \setcounter{enumi}{40}

  \item It is a basis for $\mathbb{R}^3$

    \begin{enumerate}

      \item $S$ is linearly independent \textcolor{green}{\checkmark}

      \item $S$ spans $\mathbb{R}^3$ \textcolor{green}{\checkmark}

    \end{enumerate}

    \setcounter{enumi}{44}

  \item It is a basis for $\mathbb{R}^4$

    \begin{enumerate}

      \item $S$ is linearly independent \textcolor{green}{\checkmark}

      \item $S$ spans $\mathbb{R}^4$ \textcolor{green}{\checkmark}

    \end{enumerate}

    \setcounter{enumi}{46}

  \item It is a basis for $P_3$

    \begin{enumerate}

      \item $\begin{vmatrix} 1 & -4 & 0 & 0\\ 0 & 0 & 2 & 5\\ -2 & 1 & 0 & 0\\ 1 & 0 & 1 & 0\\  \end{vmatrix}\neq0$ so it is linearly independent \textcolor{green}{\checkmark}

      \item $S$ spans $P_3$ \textcolor{green}{\checkmark}

    \end{enumerate}

    \setcounter{enumi}{66}

  \item $\left\{ (0,1),(1,0) \right\},\left\{ (1,1),(0,1) \right\},\left\{ (1,1),(1,0)  \right\}$

    \setcounter{enumi}{72}

  \item

    \begin{enumerate}

      \item $W$ forms a line

      \item There is only one term, so it is a basis of itself: $\left\{ (2,1,-1) \right\}$

      \item One term, so dimension = 1

    \end{enumerate}

    \setcounter{enumi}{77}

  \item

    \begin{enumerate}

      \item Basis: $\left\{ (1,0,1,2)  \right\},\left\{ (4,1,0,-1)  \right\}$

      \item Two terms, so dimension = 2

    \end{enumerate}

    \setcounter{enumi}{80}

  \item

    \begin{equation*}
      \begin{split}
        \text{If } S=\{\bold{v}_1,\bold{v}_2,\dots,\bold{v}_n\}\text{, then there exists a solution to:}\\
        \bold{v}_1+\bold{v}_2+\dots+\bold{v}_n=0\\
        \text{The same can be said for }S=\left\{ c\bold{v}_1,c\bold{v}_2,\dots,c\bold{v}_n \right\}\\
        c\bold{v}_1+c\bold{v}_2+\dots+c\bold{v}_n=0\\
        \text{Multiplying both sides by }\frac{1}{c}\text{ the following is obtained:}\\
        \bold{v}_1+\bold{v}_2+\dots+\bold{v}_n=0\\
        \therefore\text{ both are basis for }V
      \end{split}
      \label{2}
    \end{equation}

\end{enumerate}

\end{document}

