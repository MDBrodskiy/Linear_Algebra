%%%%%%%%%%%%%%%%%%%%%%%%%%%%%%%%%%%%%%%%%%%%%%%%%%%%%%%%%%%%%%%%%%%%%%%%%%%%%%%%%%%%%%%%%%%%%%%%%%%%%%%%%%%%%%%%%%%%%%%%%%%%%%%%%%%%%%%%%%%%%%%%%%%%%%%%%%%%%%%%%%%%%%%%%%%%%%%%%%%%%%%%%%%%
% Written By Michael Brodskiy
% Class: Linear Algebra
% Professor: L. Knight
%%%%%%%%%%%%%%%%%%%%%%%%%%%%%%%%%%%%%%%%%%%%%%%%%%%%%%%%%%%%%%%%%%%%%%%%%%%%%%%%%%%%%%%%%%%%%%%%%%%%%%%%%%%%%%%%%%%%%%%%%%%%%%%%%%%%%%%%%%%%%%%%%%%%%%%%%%%%%%%%%%%%%%%%%%%%%%%%%%%%%%%%%%%%

\documentclass[12pt]{article} 
\usepackage{alphalph}
\usepackage[utf8]{inputenc}
\usepackage[russian,english]{babel}
\usepackage{titling}
\usepackage{amsmath}
\usepackage{graphicx}
\usepackage{enumitem}
\usepackage{amssymb}
\usepackage{physics}
\usepackage{tikz}
\usepackage{mathdots}
\usepackage{yhmath}
\usepackage{cancel}
\usepackage{color}
\usepackage{siunitx}
\usepackage{array}
\usepackage{multirow}
\usepackage{gensymb}
\usepackage{tabularx}
\usepackage{booktabs}
\usepackage{pifont}
\newcommand{\xmark}{\ding{55}}
\usetikzlibrary{fadings}
\usetikzlibrary{patterns}
\usetikzlibrary{shadows.blur}
\usetikzlibrary{shapes}
\usepackage[super]{nth}
\usepackage{expl3}
\usepackage[version=4]{mhchem}
\usepackage{hpstatement}
\usepackage{rsphrase}
\usepackage{everysel}
\usepackage{ragged2e}
\usepackage{geometry}
\usepackage{fancyhdr}
\usepackage{cancel}
\usepackage{multicol}
\geometry{top=1.0in,bottom=1.0in,left=1.0in,right=1.0in}
\newcommand{\subtitle}[1]{%
  \posttitle{%
    \par\end{center}
    \begin{center}\large#1\end{center}
    \vskip0.5em}%

}
\usepackage{hyperref}
\hypersetup{
colorlinks=true,
linkcolor=blue,
filecolor=magenta,      
urlcolor=blue,
citecolor=blue,
}

\urlstyle{same}


\title{Linear Algebra 4.3 Homework}
\date{}
\author{Michael Brodskiy\\ \small Instructor: Prof. Knight}

\begin{document}

\maketitle

\begin{enumerate}

  \item $W$ is a subspace of $\bold{V}$

    \begin{enumerate}

      \item $W$ contains the origin, and is therefore not empty \textcolor{green}{\checkmark}

      \item $W\leq\bold{V}$ \textcolor{green}{\checkmark}

      \item $(x_1,x_2,x_3,0)+(y_1,y_2,y_3,0)=(x_1+y_1,x_2+y_2,x_3+y_3,0)$ closed under addition \textcolor{green}{\checkmark}

      \item $c(x_1,x_2,x_3,0)=(cx_1,cx_2,cx_3,0)$ closed under multiplication \textcolor{green}{\checkmark}

    \end{enumerate}

    \setcounter{enumi}{3}

  \item $W$ is a subspace of $\bold{V}$

    \begin{enumerate}

      \item $W$ contains the origin, and is therefore not empty \textcolor{green}{\checkmark}

      \item $W\leq\bold{V}$ \textcolor{green}{\checkmark}

      \item $w_1=\begin{bmatrix} a_1 & b_1\\ a_1-2b_1 & 0\\ 0 & c_1  \end{bmatrix}$ and $w_2=\begin{bmatrix} a_2 & b_2\\ a_2-2b_2 & 0\\ 0 & c_2  \end{bmatrix}$, then\\ $w_1+w_2=\begin{bmatrix} a_1+a_2 & b_1+b_2\\ a_1+a_2-2(b_1+b_2) & 0\\ 0 & c_1+c_2  \end{bmatrix}\text{, where }\begin{array}{c} a=a_1+a_2\\ b=b_1+b_2\\ c=c_1+c_2  \end{array}$\\ closed under addition \textcolor{green}{\checkmark}

      \item $cw_1=c\begin{bmatrix} a_1 & b_1\\ a_1-2b_1 & 0\\ 0 & c_1  \end{bmatrix}=\begin{bmatrix} ca_1 & cb_1\\ ca_1-2cb_1 & 0\\ 0 & cc_1  \end{bmatrix}\text{, where }\begin{array}{c} a=ca_1\\ b=cb_1\\ c=cc_1  \end{array}$\\ closed under multiplication \textcolor{green}{\checkmark}

    \end{enumerate}

  \item $W$ is a subspace of $\bold{V}$

    \begin{enumerate}

      \item $W$ contains the origin, and is therefore not empty \textcolor{green}{\checkmark}

      \item $W\leq\bold{V}$ \textcolor{green}{\checkmark}

      \item $f+g$ is closed under addition \textcolor{green}{\checkmark}

      \item $cf$ is closed under multiplication \textcolor{green}{\checkmark}

    \end{enumerate}

    \setcounter{enumi}{7}

  \item It is not closed under multiplication. Given some vector $c\langle 2, x_1, x_2$, where $c\neq1$, the value changes, so it is not a subspace

    \setcounter{enumi}{11}

  \item It is not closed under addition. Given $v_1=x+1$ and $v_2=1-x$, the sum is $2$, which is not a linear function of the form $ax+b$

    \setcounter{enumi}{14}

  \item It is not closed under multiplication. Given a value $c\neq1$, $cW$ is not in the vector space $\bold{V}$

  \item It is not closed under addition. Given another $M_{3,1}$ matrix, for example, $\begin{bmatrix} \sqrt{a} & 0 & a \end{bmatrix}$, and adding it to the original matrix generates a matrix that is not in $\bold{V}$

    \setcounter{enumi}{20}

  \item No, it does not contain the origin

    \setcounter{enumi}{22}

  \item Yes, because it contains the origin, and is closed under multiplication and addition

    \setcounter{enumi}{26}

  \item  Yes, because it contains the origin, and is closed under multiplication and addition

    \setcounter{enumi}{29}

  \item Yes, because it is not empty, and is closed under multiplication and addition

  \item No, because it is not closed under multiplication

    \setcounter{enumi}{32}

  \item No, because it is not closed under addition

    \setcounter{enumi}{42}

  \item

    \begin{enumerate}

      \item True $-$ This is one of the four axioms that a subspace must follow

      \item True $-$ If $\bold{V}$ and $\bold{W}$ are subspaces of the same vector space, then anything contained within $\bold{V}$ and $\bold{W}$ must also be a subspace

      \item False $-$ Although this is possible, it is not definite, as $\bold{U}$ could be smaller than $\bold{V}$, which is smaller than $\bold{U}$

    \end{enumerate}

  \item 

    \begin{enumerate}

      \item True $-$ The origin must exist in a vector space, and a subspace can be equal to the vector space itself

      \item True $-$ The origin must always be contained in a subspace

      \item True $-$ This is one of 4 axioms that a subspace must follow

      \item False $-$ It is possible that something contained within a vector space is not itself a vector space

    \end{enumerate}

    \setcounter{enumi}{46}

  \item

    \begin{enumerate}

      \item $C(-\infty,\infty)$ is a subspace of $F(-\infty,\inty)$

        \begin{enumerate}

          \item $C(-\infty,\infty)$ contains the origin \textcolor{green}{\checkmark}

          \item $C(-\infty,\infty)\leq F(-\infty,\infty)$ \textcolor{green}{\checkmark}

      \item $f+g$ is closed under addition \textcolor{green}{\checkmark}

      \item $cf$ is closed under multiplication \textcolor{green}{\checkmark}

    \end{enumerate}

    \end{enumerate}

  \item

    \begin{enumerate}

      \item $S$ contains the origin \textcolor{green}{\checkmark}

      \item $S\leq C[0,1]$ \textcolor{green}{\checkmark}

      \item $f+g$ is continuous because it is integrable and therefore closed under addition\textcolor{green}{\checkmark}

      \item $cf$ is still continuous, and, therefore, closed under multiplication \textcolor{green}{\checkmark}

    \end{enumerate}

    \setcounter{enumi}{50}

  \item $W$ is a subspace of $\bold{V}$

    \begin{enumerate}

      \item Because $W$ is a subspace, it is closed under multiplication. Since $a\bold{x}$ and $b\bold{y}$ are in $W$, so is $a\bold{x}+b\bold{y}$.

      \item If $a=1$ and $b=0$, $a\bold{x}$ is in $W$. Therefore, if $a=1$ and $b=1$, $a\bold{x}+b\bold{y}$ is in $W$, meaning $W$ is closed under addition and scalar multiplication.

    \end{enumerate}

  \item $W$ is a subspace of $\bold{V}$

    \begin{enumerate}

      \item $W$ is not empty \textcolor{green}{\checkmark}

      \item $W\leq \bold{V}$ \textcolor{green}{\checkmark}

      \item $\overrightarrow{\bold{x}}_1=ax_1+by_1+cz_1$ and $\overrightarrow{\bold{x}}_2=ax_2+by_2+cz_2$, then $\overrightarrow{\bold{x}}_1+\overrightarrow{\bold{x}}_2=a(x_1+x_2)+b(y_1+y_2)+c(z_1+z_2)$, where $\begin{array}{c} x=x_1+x_2\\y=y_1+y_2\\z=z_1+z_2\\\end{array}$ closed under addition \textcolor{green}{\checkmark}

      \item $k\overrightarrow{\bold{x}}_1=kax_1+kby_1+kcz_1$, where $\begin{array}{c} x=kx_1\\y=ky_1\\z=kz_1\\\end{array}$ closed under multiplication \textcolor{green}{\checkmark}

    \end{enumerate}

    \setcounter{enumi}{53}

  \item $W$ is a subspace $\bold{V}$

    \begin{enumerate}

      \item $W$ contains the origin $(0,0,\dots,0)$ \textcolor{green}{\checkmark}

      \item $W\leq\mathbb{R}^n$ \textcolor{green}{\checkmark}

      \item $A\bold{x}_1+A\bold{x}_2=\bold{0}+\bold{0}=\bold{0}$ closed under addition \textcolor{green}{\checkmark}

      \item $c(A\bold{x})=cA\bold{x}=c\dot0=0$ closed under multiplication \textcolor{green}{\checkmark}

    \end{enumerate}

\end{enumerate}

\end{document}

