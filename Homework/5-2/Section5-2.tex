%%%%%%%%%%%%%%%%%%%%%%%%%%%%%%%%%%%%%%%%%%%%%%%%%%%%%%%%%%%%%%%%%%%%%%%%%%%%%%%%%%%%%%%%%%%%%%%%%%%%%%%%%%%%%%%%%%%%%%%%%%%%%%%%%%%%%%%%%%%%%%%%%%%%%%%%%%%%%%%%%%%%%%%%%%%%%%%%%%%%%%%%%%%%
% Written By Michael Brodskiy
% Class: Linear Algebra
% Professor: L. Knight
%%%%%%%%%%%%%%%%%%%%%%%%%%%%%%%%%%%%%%%%%%%%%%%%%%%%%%%%%%%%%%%%%%%%%%%%%%%%%%%%%%%%%%%%%%%%%%%%%%%%%%%%%%%%%%%%%%%%%%%%%%%%%%%%%%%%%%%%%%%%%%%%%%%%%%%%%%%%%%%%%%%%%%%%%%%%%%%%%%%%%%%%%%%%

\documentclass[12pt]{article} 
\usepackage{alphalph}
\usepackage[utf8]{inputenc}
\usepackage[russian,english]{babel}
\usepackage{titling}
\usepackage{amsmath}
\usepackage{graphicx}
\usepackage{enumitem}
\usepackage{amssymb}
\usepackage{physics}
\usepackage{tikz}
\usepackage{mathdots}
\usepackage{yhmath}
\usepackage{cancel}
\usepackage{color}
\usepackage{siunitx}
\usepackage{array}
\usepackage{multirow}
\usepackage{gensymb}
\usepackage{tabularx}
\usepackage{booktabs}
\usepackage{pifont}
\newcommand{\xmark}{\ding{55}}
\usetikzlibrary{fadings}
\usetikzlibrary{patterns}
\usetikzlibrary{shadows.blur}
\usetikzlibrary{shapes}
\usepackage[super]{nth}
\usepackage{expl3}
\usepackage[version=4]{mhchem}
\usepackage{hpstatement}
\usepackage{rsphrase}
\usepackage{everysel}
\usepackage{ragged2e}
\usepackage{geometry}
\usepackage{fancyhdr}
\usepackage{cancel}
\usepackage{multicol}
\geometry{top=1.0in,bottom=1.0in,left=1.0in,right=1.0in}
\newcommand{\subtitle}[1]{%
  \posttitle{%
    \par\end{center}
    \begin{center}\large#1\end{center}
    \vskip0.5em}%

}
\usepackage{hyperref}
\hypersetup{
colorlinks=true,
linkcolor=blue,
filecolor=magenta,      
urlcolor=blue,
citecolor=blue,
}

\urlstyle{same}


\title{Linear Algebra 5.2 Homework}
\date{}
\author{Michael Brodskiy\\ \small Instructor: Prof. Knight}

\begin{document}

\maketitle

\begin{enumerate}

    \begin{center}
      \underline{Problems 1, 8, 9, 11, 13, 15, 17, 23, 24, 29, 33, 36, 39, 41, 43, 45, 47, 49, 50, 51, 65,}\\\underline{ 67, 75, 79, 85}
    \end{center}

  \item This does define an inner product

    \begin{enumerate}

      \item $(\overrightarrow{u},\overrightarrow{v})=3u_1v_1+u_2v_2=3v_1u_1+v_2u_2=(\overrightarrow{v},\overrightarrow{u})$ \textcolor{green}{\checkmark}

      \item $(\overrightarrow{u},\overrightarrow{v}+\overrightarrow{w})=3u_1(v_1+w_1)+u_2(v_2+w_2)=3u_1v_1+3u_1w_1+u_2v_2+u_2w_2=(\overrightarrow{u},\overrightarrow{v})+(\overrightarrow{u},\overrightarrow{w})$ \textcolor{green}{\checkmark}

      \item $c(\overrightarrow{u},\overrightarrow{v})=c(3u_1v_1+u_2v_2)=3cu_1v_1+cu_2v_2=(c\overrightarrow{u},\overrightarrow{v})$ \textcolor{green}{\checkmark}

      \item $(\overrightarrow{v},\overrightarrow{v})=3(v_1)^2+(v_2)^2$ \textcolor{green}{\checkmark}

    \end{enumerate}

    \setcounter{enumi}{7}

  \item This does define an inner product

    \begin{enumerate}

      \item $(\overrightarrow{u},\overrightarrow{v})=\frac{1}{2}u_1v_1+\frac{1}{4}u_2v_2+\frac{1}{2}u_3v_3=\frac{1}{2}v_1u_1+\frac{1}{4}v_2u_2+\frac{1}{2}v_3u_3=(\overrightarrow{v},\overrightarrow{u})$ \textcolor{green}{\checkmark}

      \item $(\overrightarrow{u},\overrightarrow{v}+\overrightarrow{w})=\frac{1}{2}u_1(v_1+w_1)+\frac{1}{4}u_2(v_2+w_2)+\frac{1}{2}u_3(v_3+w_3)=\frac{1}{2}u_1v_1+\frac{1}{2}u_1w_1+\frac{1}{4}u_2v_2+\frac{1}{4}u_2w_2+\frac{1}{2}u_3v_3+\frac{1}{2}u_3w_3=(\overrightarrow{u},\overrightarrow{v})+(\overrightarrow{u},\overrightarrow{w})$ \textcolor{green}{\checkmark}

      \item $c(\overrightarrow{u},\overrightarrow{v})=c(\frac{1}{2}u_1v_1+\frac{1}{4}u_2v_2+\frac{1}{2}u_3v_3)=\frac{1}{2}cu_1v_1+\frac{1}{4}cu_2v_2+\frac{1}{2}cu_3v_3=(c\overrightarrow{u},\overrightarrow{v})$ \textcolor{green}{\checkmark}

      \item $(\overrightarrow{v},\overrightarrow{v})=\frac{1}{2}(v_1)^2+\frac{1}{4}(v_2)^2+\frac{1}{2}(v_3)^2$ \textcolor{green}{\checkmark}

    \end{enumerate}

  \item This does \underline{not} define an inner product because it fails axiom 4, which states that the inner product of $\overrightarrow{v}$ with itself only equals zero if $\overrightarrow{v}$ itself is zero. This is not true, as, for $\overrightarrow{v}=\langle0, c\rangle$, the function fails the axiom.

    \setcounter{enumi}{10}

  \item This does \underline{not} define an inner product, as it fails axiom 4. This is because, for any vector $\overrightarrow{v}$ where $v_1=v_2$, the vector equals zero, which fails axiom 4.

    \setcounter{enumi}{12}

  \item This does \underline{not} define an inner product, as it fails axiom 1. This is because, for any $\overrightarrow{v}=\langle 0,0,c\rangle$, this fails, as $(\overrightarrow{u},\overrightarrow{v})=-u_1u_2u_3$, but $(\overrightarrow{v},\overrightarrow{u})=0$

    \setcounter{enumi}{14}

  \item This does \underline{not} define an inner product, as it fails axiom 3. This is because $c(\overrightarrow{u},\overrightarrow{v})=c( (u_1v_1)^2+(u_2v_2)^2+(u_3v_3)^2)$, but $(c\overrightarrow{u},\overrightarrow{v})=c^2u_1^2v_1^2+c^2u_2^2v_2^2+c^2u_3^2v_3^2$

    \setcounter{enumi}{16}

  \item

    \begin{enumerate}

      \item $3(5)+4(-12)=-33$

      \item $\sqrt{3^2+4^2}=5$

      \item $\sqrt{5^2+(-12)^2}=13$

      \item $\sqrt{(-2)^2+(16)^2}=2\sqrt{65}$

    \end{enumerate}

    \setcounter{enumi}{22}

  \item

    \begin{enumerate}

      \item $2(8)(8)+3(0)(3)+(-8)(16)=0$

      \item $\sqrt{2(8)(8)+(-8)(-8)}=8\sqrt{3}$

      \item $\sqrt{2(8)^2+3(3)^2+(16)^2}=\sqrt{411}$

      \item $||\overrightarrow{u}-\overrightarrow{v}||=\langle0,-3,-24\rangle\Rightarrow \sqrt{3(-3)^2+(-24)^2}=3\sqrt{67}$

    \end{enumerate}

  \item

    \begin{enumerate}

      \item $(1)(2)+2(1)(5)+(1)(2)=14$

      \item $\sqrt{1^2+2(1)^2+1^2}=2$

      \item $\sqrt{2^2+2(5)^2+2^2}=\sqrt{58}$

      \item $\langle-1,-4,-1\rangle\Rightarrow\sqrt{(-1)^2+2(-4)^2+(-1)^2}=\sqrt{34}$

    \end{enumerate}

    \setcounter{enumi}{28}

  \item

    \begin{enumerate}

      \item $2(2)(-2)+(-4)(1)+(-3)(1)+2(1)(0)=-15$

      \item $\sqrt{2(2)^2+(-4)^2+(-3)^2+2(1)^2}=\sqrt{35}$

      \item $\sqrt{2(-2)^2+(1)^2+(1)^2}=\sqrt{10}$

      \item $\begin{bmatrix} 4 & -5\\ -4 & 1\end{bmatrix}\Rightarrow \sqrt{2(4)^2+(-5)^2+(-4)^2+2(1)^2}=\sqrt{75}=5\sqrt{3}$

    \end{enumerate}

    \setcounter{enumi}{32}

  \item This is an inner product for $P_2$

    \begin{enumerate}

      \item $\langle p,q\rangle=a_0b_0+2a_1b_1+a_2b_2=b_0a_0+2b_1a_1+b_2a_2=\langle q,p\rangle$ \textcolor{green}{\checkmark}

      \item $\langle p, q+r\rangle=a_0(b_0+c_0)+2a_1(b_1+c_1)+a_2(b_2+c_2)=a_0b_0+a_0c_0+2a_1b_1+2a_1c_1+a_2b_2+a_2c_2=\langle p,q\rangle+\langle p,r\rangle$ \textcolor{green}{\checkmark}

      \item $c\langle p,q\rangle=c(a_0b_0+2a_1b_1+a_2b_2)=ca_0b_0+2ca_1b_1+ca_2b_2=\langle cp,q\rangle$ \textcolor{green}{\checkmark}

      \item $\langle p,p\rangle=(a_0)^2+2(a_1)^2+(a_2)^2 \geq 0$ \textcolor{green}{\checkmark}

    \end{enumerate}

    \setcounter{enumi}{35}

  \item

    \begin{enumerate}

      \item $(1)(1)+(1)(0)+\frac{1}{2}(2)=2$

      \item $\sqrt{1^2+1^2+\left(\frac{1}{2}\right)^2}=\frac{3}{2}$

      \item $\sqrt{(1)^2+(2)^2}=\sqrt{5}$

      \item $x-\frac{3}{2}x^2\Rightarrow\sqrt{1^2+\left( -\frac{3}{2} \right)^2}=\frac{\sqrt{13}}{2}$

    \end{enumerate}

    \setcounter{enumi}{38}

  \item

    \begin{enumerate}

      \item $\int_{-1}^1 4x^2-1\,dx=\left( \frac{4}{3}x^3-x \right)\Big|_{-1}^1=\frac{2}{3}$

      \item $\sqrt{\int_{-1}^1 1\,dx}=\sqrt{(x)\Big|_{-1}^1}=\sqrt{2}$

      \item $\sqrt{\int_{-1}^1 \left( 4x^2-1 \right)^2\,dx}=\sqrt{2\int_0^1 16x^4-8x^2+1\,dx}=\sqrt{\left(\frac{32}{5}x^5-\frac{16}{3}x^3+2x\right)\Big|_0^1}=\sqrt{\frac{46}{15}}$

      \item $2-4x^2\Rightarrow\sqrt{2\int_0^1 \left(2-4x^2\right)^2\,dx}=\sqrt{2\int_0^1 4-16x^2+16x^4\,dx}=$\\$\sqrt{\left(8x-\frac{32}{3}x^3+\frac{32}{5}x^5\right)\Big|_0^1}=\sqrt{\frac{56}{15}}=\frac{2\sqrt{14}}{\sqrt{15}}$

    \end{enumerate}

    \setcounter{enumi}{40}

  \item

    \begin{enumerate}

      \item $\int_{-1}^1 xe^x\,dx=\left( xe^x-e^x \right)\Big|_{-1}^1=\frac{2}{e}$

      \item $\sqrt{2\int_0^1 x^2\,dx}=\sqrt{\left( \frac{2}{3}x^3 \right)\Big|_0^1}=\frac{\sqrt{6}}{3}$

      \item $\sqrt{\int_{-1}^1 e^{2x}\,dx}=\sqrt{\left( \frac{1}{2}e^{2x} \right)\Big|_{-1}^1}=\sqrt{\frac{e^2}{2}-\frac{1}{2e^2}}$

      \item $x-e^x\Rightarrow\sqrt{\int_{-1}^1 \left( x-e^x \right)^2\,dx}=\sqrt{\int_{-1}^1 x^2-2xe^x+e^{2x}\,dx}=$\\$\sqrt{\left( \frac{1}{3}x^3-2xe^x+2e^x+\frac{1}{2}e^{2x} \right)\Big|_{-1}^1}=\sqrt{\frac{2}{3}-\frac{4}{e}+\frac{e^2}{2}-\frac{1}{2e^2}}$

    \end{enumerate}

    \setcounter{enumi}{42}

  \item $\cos^{-1}\left( \frac{-33}{5(13)} \right)=120.5^{\circ}$

    \setcounter{enumi}{44}

  \item $\cos^{-1}\left( \frac{15}{(\sqrt{3(-4)^2+(3)^2})(\sqrt{5^2})} \right)=66.6^{\circ}$

    \setcounter{enumi}{46}

  \item $\cos^{-1}\left( 0 \right)=90^{\circ}$

    \setcounter{enumi}{48}

  \item $\cos^{-1}\left( \frac{1}{3}  \right)=70.5^{\circ}$

  \item $\cos^{-1}\left( 0 \right)=90^{\circ}$

  \item $\int_{-1}^1 x^3\,dx=0\Rightarrow\cos^{-1}\left( 0 \right)=90^{\circ}$

    \setcounter{enumi}{64}

  \item $\int_{-\frac{\pi}{2}}^{\frac{\pi}{2}} \sin(x)\cos(x)\,dx=\int_0^{\frac{\pi}{2}} \sin(2x)\,dx=$\\$\left( -\frac{1}{2}\cos(2x) \right)\Big|_{0}^{\frac{\pi}{2}}=0$, so they are orthogonal

    \setcounter{enumi}{66}

  \item $\frac{1}{2}\int_{-1}^1 5x^4-3x^2\,dx\Rightarrow\int_0^1 5x^4-3x^2\,dx=\left( x^5-x^3 \right)\Big|0^1=0$, so they are orthogonal

    \setcounter{enumi}{74}

  \item

    \begin{enumerate}

      \item $\frac{\langle \overrightarrow{u},\overrightarrow{v}\rangle}{\langle \overrightarrow{v},\overrightarrow{v}\rangle}\overrightarrow{v}=\frac{1(1)+3(2)-6(2)}{(-1)^2+1^2+2^2+2^2}=\langle \frac{1}{2},-\frac{1}{2},-1,-1\rangle$

      \item $\frac{\langle \overrightarrow{u},\overrightarrow{v}\rangle}{\langle \overrightarrow{u},\overrightarrow{u}\rangle}\overrightarrow{u}=\frac{1(1)+3(2)-6(2)}{1^2+3^2+(-6)^2}=\langle 0, -\frac{5}{46}, -\frac{15}{46}, \frac{15}{23}\rangle$

    \end{enumerate}

    \setcounter{enumi}{78}

  \item $\int_0^1 xe^x\,dx=\left( xe^x-e^x \right)\Big|_0^1=1$, $\int_0^1 e^{2x}\,dx=\left( \frac{1}{2}e^{2x} \right)\Big_0^1=\frac{e^2}{2}-\frac{1}{2}\Rightarrow\frac{2}{e^2-1}g=\frac{2}{e^2-1}e^x$

    \setcounter{enumi}{84}

  \item

    \begin{enumerate}

      \item False. The dot product is the only euclidean product, but others may be defined.

      \item False. The magnitude of $\overrightarrow{v}$ can only equal zero if $\overrightarrow{v}=0$

    \end{enumerate}

\end{enumerate}

\end{document}

