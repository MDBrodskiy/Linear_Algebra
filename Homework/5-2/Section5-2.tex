%%%%%%%%%%%%%%%%%%%%%%%%%%%%%%%%%%%%%%%%%%%%%%%%%%%%%%%%%%%%%%%%%%%%%%%%%%%%%%%%%%%%%%%%%%%%%%%%%%%%%%%%%%%%%%%%%%%%%%%%%%%%%%%%%%%%%%%%%%%%%%%%%%%%%%%%%%%%%%%%%%%%%%%%%%%%%%%%%%%%%%%%%%%%
% Written By Michael Brodskiy
% Class: Linear Algebra
% Professor: L. Knight
%%%%%%%%%%%%%%%%%%%%%%%%%%%%%%%%%%%%%%%%%%%%%%%%%%%%%%%%%%%%%%%%%%%%%%%%%%%%%%%%%%%%%%%%%%%%%%%%%%%%%%%%%%%%%%%%%%%%%%%%%%%%%%%%%%%%%%%%%%%%%%%%%%%%%%%%%%%%%%%%%%%%%%%%%%%%%%%%%%%%%%%%%%%%

\documentclass[12pt]{article} 
\usepackage{alphalph}
\usepackage[utf8]{inputenc}
\usepackage[russian,english]{babel}
\usepackage{titling}
\usepackage{amsmath}
\usepackage{graphicx}
\usepackage{enumitem}
\usepackage{amssymb}
\usepackage{physics}
\usepackage{tikz}
\usepackage{mathdots}
\usepackage{yhmath}
\usepackage{cancel}
\usepackage{color}
\usepackage{siunitx}
\usepackage{array}
\usepackage{multirow}
\usepackage{gensymb}
\usepackage{tabularx}
\usepackage{booktabs}
\usepackage{pifont}
\newcommand{\xmark}{\ding{55}}
\usetikzlibrary{fadings}
\usetikzlibrary{patterns}
\usetikzlibrary{shadows.blur}
\usetikzlibrary{shapes}
\usepackage[super]{nth}
\usepackage{expl3}
\usepackage[version=4]{mhchem}
\usepackage{hpstatement}
\usepackage{rsphrase}
\usepackage{everysel}
\usepackage{ragged2e}
\usepackage{geometry}
\usepackage{fancyhdr}
\usepackage{cancel}
\usepackage{multicol}
\geometry{top=1.0in,bottom=1.0in,left=1.0in,right=1.0in}
\newcommand{\subtitle}[1]{%
  \posttitle{%
    \par\end{center}
    \begin{center}\large#1\end{center}
    \vskip0.5em}%

}
\usepackage{hyperref}
\hypersetup{
colorlinks=true,
linkcolor=blue,
filecolor=magenta,      
urlcolor=blue,
citecolor=blue,
}

\urlstyle{same}


\title{Linear Algebra 5.2 Homework}
\date{}
\author{Michael Brodskiy\\ \small Instructor: Prof. Knight}

\begin{document}

\maketitle

\begin{enumerate}

    \begin{center}
      \underline{Problems 1, 8, 9, 11, 13, 15, 17, 23, 24, 29, 33, 36, 39, 41, 43, 45, 47, 49, 50, 51, 65,}\\\underline{ 67, 75, 79, 85}
    \end{center}

  \item This does define an inner product

    \begin{enumerate}

      \item $(\overrightarrow{u},\overrightarrow{v})=3u_1v_1+u_2v_2=3v_1u_1+v_2u_2=(\overrightarrow{v},\overrightarrow{u})$ \textcolor{green}{\checkmark}

      \item $(\overrightarrow{u},\overrightarrow{v}+\overrightarrow{w})=3u_1(v_1+w_1)+u_2(v_2+w_2)=3u_1v_1+3u_1w_1+u_2v_2+u_2w_2=(\overrightarrow{u},\overrightarrow{v})+(\overrightarrow{u},\overrightarrow{w})$ \textcolor{green}{\checkmark}

      \item $c(\overrightarrow{u},\overrightarrow{v})=c(3u_1v_1+u_2v_2)=3cu_1v_1+cu_2v_2=(c\overrightarrow{u},\overrightarrow{v})$ \textcolor{green}{\checkmark}

      \item $(\overrightarrow{v},\overrightarrow{v})=3(v_1)^2+(v_2)^2$ \textcolor{green}{\checkmark}

    \end{enumerate}

    \setcounter{enumi}{7}

  \item This does define an inner product

    \begin{enumerate}

      \item $(\overrightarrow{u},\overrightarrow{v})=\frac{1}{2}u_1v_1+\frac{1}{4}u_2v_2+\frac{1}{2}u_3v_3=\frac{1}{2}v_1u_1+\frac{1}{4}v_2u_2+\frac{1}{2}v_3u_3=(\overrightarrow{v},\overrightarrow{u})$ \textcolor{green}{\checkmark}

      \item $(\overrightarrow{u},\overrightarrow{v}+\overrightarrow{w})=\frac{1}{2}u_1(v_1+w_1)+\frac{1}{4}u_2(v_2+w_2)+\frac{1}{2}u_3(v_3+w_3)=\frac{1}{2}u_1v_1+\frac{1}{2}u_1w_1+\frac{1}{4}u_2v_2+\frac{1}{4}u_2w_2+\frac{1}{2}u_3v_3+\frac{1}{2}u_3w_3=(\overrightarrow{u},\overrightarrow{v})+(\overrightarrow{u},\overrightarrow{w})$ \textcolor{green}{\checkmark}

      \item $c(\overrightarrow{u},\overrightarrow{v})=c(\frac{1}{2}u_1v_1+\frac{1}{4}u_2v_2+\frac{1}{2}u_3v_3)=\frac{1}{2}cu_1v_1+\frac{1}{4}cu_2v_2+\frac{1}{2}cu_3v_3=(c\overrightarrow{u},\overrightarrow{v})$ \textcolor{green}{\checkmark}

      \item $(\overrightarrow{v},\overrightarrow{v})=\frac{1}{2}(v_1)^2+\frac{1}{4}(v_2)^2+\frac{1}{2}(v_3)^2$ \textcolor{green}{\checkmark}

    \end{enumerate}

  \item This does \underline{not} define an inner product because it fails axiom 4, which states that the inner product of $\overrightarrow{v}$ with itself only equals zero if $\overrightarrow{v}$ itself is zero. This is not true, as, for $\overrightarrow{v}=\langle0, c\rangle$, the function fails the axiom.

    \setcounter{enumi}{10}

  \item This does \underline{not} define an inner product, as it fails axiom 4. This is because, for any vector $\overrightarrow{v}$ where $v_1=v_2$, the vector equals zero, which fails axiom 4.

    \setcounter{enumi}{12}

  \item This does \underline{not} define an inner product, as it fails axiom 1. This is because, for any $\overrightarrow{v}=\langle 0,0,c\rangle$, this fails, as $(\overrightarrow{u},\overrightarrow{v})=-u_1u_2u_3$, but $(\overrightarrow{v},\overrightarrow{u})=0$

    \setcounter{enumi}{14}

  \item This does \underline{not} define an inner product, as it fails axiom 3. This is because $c(\overrightarrow{u},\overrightarrow{v})=c( (u_1v_1)^2+(u_2v_2)^2+(u_3v_3)^2)$, but $(c\overrightarrow{u},\overrightarrow{v})=c^2u_1^2v_1^2+c^2u_2^2v_2^2+c^2u_3^2v_3^2$

    \setcounter{enumi}{16}

  \item

    \begin{enumerate}

      \item $3(5)+4(-12)=-33$

      \item $\sqrt{3^2+4^2}=5$

      \item $\sqrt{5^2+(-12)^2}=13$

      \item $\sqrt{(-2)^2+(16)^2}=2\sqrt{65}$

    \end{enumerate}

    \setcounter{enumi}{22}

  \item

  \item

    \setcounter{enumi}{28}

  \item

    \setcounter{enumi}{32}

  \item

    \setcounter{enumi}{35}

  \item

    \setcounter{enumi}{38}

  \item

    \setcounter{enumi}{40}

  \item

    \setcounter{enumi}{42}

  \item

    \setcounter{enumi}{44}

  \item

    \setcounter{enumi}{46}

  \item

    \setcounter{enumi}{48}

  \item

  \item

  \item

    \setcounter{enumi}{64}

  \item

    \setcounter{enumi}{66}

  \item

    \setcounter{enumi}{74}

  \item

    \setcounter{enumi}{78}

  \item

    \setcounter{enumi}{84}

  \item

\end{enumerate}

\end{document}

