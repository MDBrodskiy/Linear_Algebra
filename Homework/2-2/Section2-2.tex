%%%%%%%%%%%%%%%%%%%%%%%%%%%%%%%%%%%%%%%%%%%%%%%%%%%%%%%%%%%%%%%%%%%%%%%%%%%%%%%%%%%%%%%%%%%%%%%%%%%%%%%%%%%%%%%%%%%%%%%%%%%%%%%%%%%%%%%%%%%%%%%%%%%%%%%%%%%%%%%%%%%%%%%%%%%%%%%%%%%%%%%%%%%%
% Written By Michael Brodskiy
% Class: Linear Algebra
% Professor: L. Knight
%%%%%%%%%%%%%%%%%%%%%%%%%%%%%%%%%%%%%%%%%%%%%%%%%%%%%%%%%%%%%%%%%%%%%%%%%%%%%%%%%%%%%%%%%%%%%%%%%%%%%%%%%%%%%%%%%%%%%%%%%%%%%%%%%%%%%%%%%%%%%%%%%%%%%%%%%%%%%%%%%%%%%%%%%%%%%%%%%%%%%%%%%%%%

\documentclass[12pt]{article} 
\usepackage{alphalph}
\usepackage[utf8]{inputenc}
\usepackage[russian,english]{babel}
\usepackage{titling}
\usepackage{amsmath}
\usepackage{graphicx}
\usepackage{enumitem}
\usepackage{amssymb}
\usepackage{physics}
\usepackage{tikz}
\usepackage{mathdots}
\usepackage{yhmath}
\usepackage{cancel}
\usepackage{color}
\usepackage{siunitx}
\usepackage{array}
\usepackage{multirow}
\usepackage{gensymb}
\usepackage{tabularx}
\usepackage{booktabs}
\usetikzlibrary{fadings}
\usetikzlibrary{patterns}
\usetikzlibrary{shadows.blur}
\usetikzlibrary{shapes}
\usepackage[super]{nth}
\usepackage{expl3}
\usepackage[version=4]{mhchem}
\usepackage{hpstatement}
\usepackage{rsphrase}
\usepackage{everysel}
\usepackage{ragged2e}
\usepackage{geometry}
\usepackage{fancyhdr}
\usepackage{cancel}
\usepackage{multicol}
\geometry{top=1.0in,bottom=1.0in,left=1.0in,right=1.0in}
\newcommand{\subtitle}[1]{%
  \posttitle{%
    \par\end{center}
    \begin{center}\large#1\end{center}
    \vskip0.5em}%

}
\usepackage{hyperref}
\hypersetup{
colorlinks=true,
linkcolor=blue,
filecolor=magenta,      
urlcolor=blue,
citecolor=blue,
}

\urlstyle{same}


\title{Linear Algebra 2.2 Homework}
\date{}
\author{Michael Brodskiy\\ \small Instructor: Prof. Knight}

% Mathematical Operations:

% Sum: $$\sum_{n=a}^{b} f(x) $$
% Integral: $$\int_{lower}^{upper} f(x) dx$$
% Limit: $$\lim_{x\to\infty} f(x)$$

\begin{document}

\maketitle

\begin{enumerate}

  \item

    \begin{equation*}
      \begin{split}
        \left[ \begin{array}{c c} -5 & 0\\ 3 & -6 \end{array}\right]
        +
        \left[ \begin{array}{c c} 7 & 1\\ -2 & -1 \end{array}\right]
        +
        \left[ \begin{array}{c c} -10 & -8\\ 14 & 6 \end{array}\right]
        =\\
        \left[ \begin{array}{c c} -5+7-10 & 1-8\\ 3-2+14 & -6-1+6 \end{array}\right]
        =\\
        \left[ \begin{array}{c c} -8 & -7\\ 15 & -1 \end{array}\right]
      \end{split}
      \label{1}
    \end{equation}

    \setcounter{enumi}{2}
    
  \item

    \begin{equation*}
      \begin{split}
        4\left(\left[ \begin{array}{c c c} -4 & 0 & 1\\ 0 & 2 & 3 \end{array}\right]
        -
        \left[ \begin{array}{c c c} 2 & 1 & -2\\ 3 & -6 & 0 \end{array}\right]\right)
        =\\
        4\left[ \begin{array}{c c c} -6 & -1 & 3 \\ -3 & 8 & 3 \end{array}\right]
        =\\
        \left[ \begin{array}{c c c} -24 & -4 & 12\\ -12 & 32 & 12 \end{array}\right]
      \end{split}
      \label{2}
    \end{equation}

    \setcounter{enumi}{4}
    
  \item

    \begin{equation*}
      \begin{split}
        -3\left(\left[ \begin{array}{c c} 0 & -3\\ 7 & 2 \end{array}\right]
        +
        \left[ \begin{array}{c c} -6 & 3\\ 8 & 1 \end{array}\right]\right)
        -
        2\left[ \begin{array}{c c} 4 & -4\\ 7 & -9  \end{array}\right]
        =\\
        \left[ \begin{array}{c c} 18 & 0\\ -45 & -9  \end{array}\right]
        -
        2\left[ \begin{array}{c c} 4 & -4\\ 7 & -9\end{array}\right]
        =\\
        \left[ \begin{array}{c c} 10 & 8\\ -59 & 9  \end{array}\right]
      \end{split}
      \label{3}
    \end{equation}

    \setcounter{enumi}{6}
    
  \item

    \begin{equation*}
      \begin{split}
        3\left[ \begin{array}{c c} 1 & 2\\ 3 & 4 \end{array}\right]
        -
        4\left[ \begin{array}{c c} 0 & 1\\ -1 & 2 \end{array}\right]
        =\\
        \left[ \begin{array}{c c} 3 & 6\\ 9 & 12  \end{array}\right]
        +
        \left[ \begin{array}{c c} 0 & -4\\ 4 & -8\end{array}\right]
        =\\
        \left[ \begin{array}{c c} 3 & 2\\ 13 & 4  \end{array}\right]
      \end{split}
      \label{4}
    \end{equation}

    \setcounter{enumi}{8}
    
  \item

    \begin{equation*}
      \begin{split}
        (-4)(3)\left[ \begin{array}{c c} 0 & 1\\ -1 & 2 \end{array}\right]
        =\\
        \left[ \begin{array}{c c} 0 & -12\\ 12 & -24  \end{array}\right]
      \end{split}
      \label{5}
    \end{equation}

    \setcounter{enumi}{10}
    
  \item

    \begin{equation*}
      \begin{split}
        (3-(-4))\left(\left[ \begin{array}{c c} 1 & 2\\ 3 & 4 \end{array}\right]
  -\left[ \begin{array}{c c} 0 & 1\\ -1 & 2 \end{array}\right]\right)
        =\\
        7\left[ \begin{array}{c c} 1 & 1\\ 4 & 2  \end{array}\right]=\\
        \left[ \begin{array}{c c} 7 & 7\\ 28 & 14  \end{array}\right]\\
      \end{split}
      \label{6}
    \end{equation}

    \setcounter{enumi}{12}
    
  \item

    \begin{enumerate}

      \item 

    \begin{equation*}
      \begin{split}
        \bold{X}=\frac{1}{3}\bold{B}-\frac{2}{3}\bold{A}\\
        \frac{1}{3}\left[ \begin{array}{c c} 1 & 2\\ -2 & 1\\ 4 & 4  \end{array}\right]-\frac{2}{3}\left[ \begin{array}{c c} -4 & 0\\ 1 & -5\\ -3 & 2  \end{array}\right]=\\
        \left[ \begin{array}{c c} \frac{1}{3} & \frac{2}{3}\\ -\frac{2}{3} & \frac{1}{3}\\ \frac{4}{3} & \frac{4}{3}  \end{array}\right]+\left[ \begin{array}{c c} \frac{8}{3} & 0\\ -\frac{2}{3} & \frac{10}{3}\\ 2 & -\frac{4}{3}  \end{array}\right]=\\
        \left[ \begin{array}{c c} 3 & \frac{2}{3}\\ -\frac{4}{3} & \frac{11}{3}\\ \frac{10}{3} & 0  \end{array}\right]
      \end{split}
      \label{7}
    \end{equation}


      \item

    \begin{equation*}
      \begin{split}
        \bold{X}=-\frac{5}{3}\bold{B}+\frac{2}{3}\bold{A}\\
        -\frac{5}{3}\left[ \begin{array}{c c} 1 & 2\\ -2 & 1\\ 4 & 4  \end{array}\right]+\frac{2}{3}\left[ \begin{array}{c c} -4 & 0\\ 1 & -5\\ -3 & 2  \end{array}\right]=\\
        \left[ \begin{array}{c c} -\frac{5}{3} & -\frac{10}{3}\\ \frac{10}{3} & -\frac{5}{3}\\ -\frac{20}{3} & -\frac{20}{3}  \end{array}\right]+\left[ \begin{array}{c c} -\frac{8}{3} & 0\\ \frac{2}{3} & -\frac{10}{3}\\ -2 & \frac{4}{3}  \end{array}\right]=\\
        \left[ \begin{array}{c c} -\frac{13}{3} & -\frac{10}{3}\\ 4 & -5\\ -\frac{26}{3} & -\frac{16}{3}  \end{array}\right]
      \end{split}
      \label{8}
    \end{equation}


      \item

    \begin{equation*}
      \begin{split}
        \bold{X}=-2\bold{B}+3\bold{A}\\
        -2\left[ \begin{array}{c c} 1 & 2\\ -2 & 1\\ 4 & 4  \end{array}\right]+3\left[ \begin{array}{c c} -4 & 0\\ 1 & -5\\ -3 & 2  \end{array}\right]=\\
        \left[ \begin{array}{c c} -2 & -4\\ 4 & -2\\ -8 & -8  \end{array}\right]+3\left[ \begin{array}{c c} -12 & 0\\ 3 & -15\\ -9 & 6  \end{array}\right]=\\
        \left[ \begin{array}{c c} -14 & -4\\ 7 & -17\\ -17 & -2  \end{array}\right]\\
      \end{split}
      \label{9}
    \end{equation}


      \item

    \begin{equation*}
      \begin{split}
        \bold{X}=\frac{1}{2}\bold{B}+\frac{2}{3}\bold{A}\\
        \frac{1}{2}\left[ \begin{array}{c c} 1 & 2\\ -2 & 1\\ 4 & 4  \end{array}\right]+\frac{2}{3}\left[ \begin{array}{c c} -4 & 0\\ 1 & -5\\ -3 & 2  \end{array}\right]=\\
        \left[ \begin{array}{c c} \frac{1}{2} & 1 \\ -1 & \frac{1}{2}\\ 2 & 2  \end{array}\right]+\left[ \begin{array}{c c} -\frac{8}{3} & 0\\ \frac{2}{3} & -\frac{10}{3}\\ -2 & \frac{4}{3}  \end{array}\right]=\\
        \left[ \begin{array}{c c}  -\frac{13}{6} & 1 \\ -\frac{1}{3} & -\frac{17}{6}\\ 0 & \frac{10}{3}  \end{array}\right]\\
      \end{split}
      \label{10}
    \end{equation}


    \end{enumerate}

    \setcounter{enumi}{14}
    
  \item

    \begin{equation*}
      \begin{split}
        \bold{BA}=\left[ \begin{array}{c c c} 1 & 5 & 0\\ -1 & 0 & -5  \end{array}\right]\\
        c\bold{BA}=\left[ \begin{array}{c c c} -2 & -10 & 0\\ 2 & 0 & 10  \end{array}\right]
      \end{split}
      \label{11}
    \end{equation}

    \setcounter{enumi}{16}
    
  \item

    \begin{equation*}
      \begin{split}
        \bold{CA}=\left[ \begin{array}{c c c} 0 & 1 & -1\\ -1 & -2 & -3\end{array}\right]\\
        \bold{B(CA)}=\left[ \begin{array}{c c c} -3 & -5 & -10\\ -2 & -5 & -5\end{array}\right]\\
      \end{split}
      \label{12}
    \end{equation}

    \setcounter{enumi}{18}
    
  \item

    \begin{equation*}
      \begin{split}
        \bold{B} + \bold{C}=\left[ \begin{array}{c c} 1 & 4\\ -2 & 2\end{array}\right]\\
        (\bold{B} + \bold{C})\bold{A}=\left[ \begin{array}{c c c} 1 & 6 & -1 \\ -2 & -2 & -8 \end{array}\right]\\
      \end{split}
      \label{13}
    \end{equation}

    \setcounter{enumi}{20}
    
  \item

    \begin{equation*}
      \begin{split}
        2\bold{C}=\left[ \begin{array}{c c} 0 & 2\\ -2 & 0\end{array}\right]\\
        \bold{B}(2\bold{C})=\left[ \begin{array}{c c} -6 & 2\\ -4 & -2\end{array}\right]\\
        -2\bold{B}(2\bold{C})=\left[ \begin{array}{c c} 12 & -4\\ 8 & 4\end{array}\right]\\
      \end{split}
      \label{14}
    \end{equation}

    \setcounter{enumi}{22}
    
  \item

    \begin{enumerate}

      \item

    \begin{equation*}
      \begin{split}
        \bold{AB}=\left[ \begin{array}{c c} 4 & 7\\ 8 & 15\end{array}\right]\\
        (\bold{AB})\bold{C}=\left[ \begin{array}{c c} 12 & 7\\ 24 & 15\end{array}\right]\\
      \end{split}
      \label{15}
    \end{equation}

  \item

    \begin{equation*}
      \begin{split}
        \bold{BC}=\left[ \begin{array}{c c} 0 & 1\\ 6 & 3\end{array}\right]\\
        \bold{A}({BC})=\left[ \begin{array}{c c} 12 & 7\\ 24 & 15\end{array}\right]\\
      \end{split}
      \label{16}
    \end{equation}

\end{enumerate}

    \setcounter{enumi}{24}
    
  \item

    \begin{equation*}
      \begin{split}
        \bold{AB}=\left[ \begin{array}{c c} -9 & 2\\ 3 & 6\end{array}\right]\\
        \bold{BA}=\left[ \begin{array}{c c} -8 & 4\\ 2 & 5\end{array}\right]\\
        \therefore \bold{AB}\neq\bold{BA}
      \end{split}
      \label{25}
    \end{equation}

    \setcounter{enumi}{26}
    
  \item

    \begin{equation*}
      \begin{split}
        \bold{AC}=\left[ \begin{array}{c c} 2 & 3\\ 2 & 3\end{array}\right]\\
        \bold{BC}=\left[ \begin{array}{c c} 2 & 3\\ 2 & 3\end{array}\right]\\
        \therefore \bold{AC}=\bold{BC}\\
        \text{Because }\bold{C}\text{ has the same rows}
      \end{split}
      \label{18}
    \end{equation}

    \setcounter{enumi}{28}
    
  \item

    \begin{equation*}
      \begin{split}
        \bold{AB}=\left[ \begin{array}{c c} 3-3 & 3-3\\ 4-4 & 4-4\end{array}\right]\\
        \bold{AB}=\left[ \begin{array}{c c} 0 & 0\\ 0 & 0\end{array}\right]\\
      \end{split}
      \label{19}
    \end{equation}

    \setcounter{enumi}{30}
    
  \item

    \begin{equation*}
      \begin{split}
        \bold{IA}=\bold{A}=\left[ \begin{array}{c c} 1 & 2\\ 0 & -1 \end{array}\right]\\
      \end{split}
      \label{20}
    \end{equation}

    \setcounter{enumi}{32}
    
  \item

    \begin{equation*}
      \begin{split}
        \bold{A}+\bold{I}=\left[ \begin{array}{c c} 1+1 & 2+0\\ 0+0 & -1+1 \end{array}\right]\\
        \bold{A}+\bold{I}=\left[ \begin{array}{c c} 2 & 2\\ 0 & 0 \end{array}\right]\\
        \bold{A}(\bold{A}+\bold{I})=\left[ \begin{array}{c c} 2 & 2\\ 0 & 0 \end{array}\right]\\
      \end{split}
      \label{21}
    \end{equation}

    \setcounter{enumi}{40}
    
  \item

    \begin{equation*}
      \begin{split}
        \bold{A}^{\intercal}=\left[ \begin{array}{c c} -1 & 2\\ 1 & 0\\ -2 & 1 \end{array}\right]\\
        \bold{B}^{\intercal}=\left[ \begin{array}{c c c} -3 & 1 & 1\\ 0 & 2 & -1\end{array}\right]\\
        \bold{B}^{\intercal}\bold{A}^{\intercal}=\left[ \begin{array}{c c} 2 & -5\\ 4 & -1\end{array}\right]\\
        \bold{AB}=\left[ \begin{array}{c c} 2 & 4\\ -5 & -1\end{array}\right]\\
        (\bold{AB})^{\intercal}=\left[ \begin{array}{c c} 2 & -5\\ 4 & -1\end{array}\right]\\
      \end{split}
      \label{22}
    \end{equation}

    \setcounter{enumi}{44}
    
  \item

    \begin{equation*}
      \begin{split}
        \bold{A}^{\intercal}=\left[ \begin{array}{c c} 4 & 0\\ 2 & 2\\ 1 & -1 \end{array}\right]\\
        \bold{A}\cdot\bold{A}^{\intercal}=\left[ \begin{array}{c c} 21 & 3\\ 3 & 5\\ 1 & -1 \end{array}\right]\\
        \text{The 3s make the matrix symmetric}\\
        \bold{A}^{\intercal}\bold{A}=\left[ \begin{array}{c c c} 16 & 8 & 4\\ 8 & 8 & 0\\ 4 & 0 & 2 \end{array}\right]\\
        \text{The non-diagonal numbers are symmetric}\\
      \end{split}
      \label{45}
    \end{equation}

    \setcounter{enumi}{48}
    
  \item

    \begin{equation*}
      \begin{split}
      \bold{A}^{16}=\left[ \begin{array}{c c c c c} 1 & 0 & 0 & 0 & 0\\ 0 & 1 & 0 & 0 & 0\\ 0 & 0 & 1 & 0 & 0\\ 0 & 0 & 0 & 1 & 0\\ 0 & 0 & 0 & 0 & 1 \end{array}\right]&\\
      \text{Because it is a diagonal matrix, &each term is raised to the (even) exponent}
      \end{split}
      \label{24}
    \end{equation}

    \setcounter{enumi}{54}
    
  \item

    \begin{enumerate}

      \item True. Each term is added to its corresponding $ij$ term, meaning that the order does not matter.

      \item False. Problem \eqref{45} is an example of this. 

      \item True. The product of a matrix and its transpose is always symmetric.

    \end{enumerate}

  \item

    \begin{enumerate}

      \item False. Problem \eqref{25} is an example of this.

      \item False. If $\bold{A}=\bold{O}$, this is not necessarily true.

      \item True. In this case, the same terms are added, they are just located in different locations.

      \item

    \end{enumerate}

  \item

    \begin{enumerate}

      \item 

    \begin{equation*}
      \begin{split}
        \left[ \begin{array}{c} 2\\-1\\3\end{array} \right]=a\left[ \begin{array}{c} 1\\0\\1\end{array} \right]+b\left[ \begin{array}{c} 1\\1\\0\end{array} \right]\\
        b \text{ must be }-1\\
        \text{Then }a=3
      \end{split}
      \label{26}
    \end{equation}

  \item

    \begin{equation*}
      \begin{split}
        \left[ \begin{array}{c} 1\\1\\1\end{array} \right]=a\left[ \begin{array}{c} 1\\0\\1\end{array} \right]+b\left[ \begin{array}{c} 1\\1\\0\end{array} \right]\\
        b\text{ must be }1\\
        \text{Then }a\text{ must be }1\\
        \text{This does not result in the desired matrix, not possible}\\
      \end{split}
      \label{27}
    \end{equation}

  \item

    \begin{equation*}
      \begin{split}
        c\left[ \begin{array}{c} 1\\1\\1\end{array} \right]+a\left[ \begin{array}{c} 1\\0\\1\end{array} \right]+b\left[ \begin{array}{c} 1\\1\\0\end{array} \right]=0\\
        \left\{\begin{array}{c c c c c c c} a & + & b & + & c & = & 0\\ & & b & + & c & = & 0\\ a & & & + & c & = & 0\\ \end{array}\\
          \text{There is no solution! }\o
      \end{split}
      \label{28}
    \end{equation}

  \item

    \begin{equation*}
      \begin{split}
        \text{The same but negative coefficients from \eqref{26} should be used}\\
        a=-3,\text{ }b=1,\text{ }c=1\\
      \end{split}
      \label{29}
    \end{equation}

    \end{enumerate}
    


    \setcounter{enumi}{59}
    
  \item

    \begin{equation*}
      \begin{split}
        -10\left[ \begin{array}{c c c} 1 & 0 & 0\\0 & 1 & 0\\ 0 & 0 & 1 \end{array} \right]=\left[ \begin{array}{c c c} -10 & 0 & 0\\0 & -10 & 0\\ 0 & 0 & -10 \end{array} \right]+\\
        5\left[ \begin{array}{c c c} 2 & 1 & -1\\1 & 0 & 2\\ -1 & 1 & 3 \end{array} \right]=\left[ \begin{array}{c c c} 10 & 5 & -5\\5 & 0 & 10\\ -5 & 5 & 15 \end{array} \right]+\\
        -2\left[ \begin{array}{c c c} 2 & 1 & -1\\1 & 0 & 2\\ -1 & 1 & 3 \end{array} \right]^2=\left[ \begin{array}{c c c} -12 & -2 & 6\\0 & -6 & -10\\ 8 & -4 & -24 \end{array} \right]+\\
        \left[ \begin{array}{c c c} 2 & 1 & -1\\1 & 0 & 2\\ -1 & 1 & 3 \end{array} \right]^3=\left[ \begin{array}{c c c} 16 & 3 & -13\\-2 & 5 & 21\\ -18 & 8 & 44 \end{array} \right]=\\
        \left[ \begin{array}{c c c} 16 & 7 & -13 \\4 & -1 & 23\\ -16 & 10 & 38 \end{array} \right]
      \end{split}
      \label{30}
    \end{equation}

  \item

    \begin{enumerate}

      \item 

        \begin{equation*}
          \begin{split}
          \bold{A},\text{ }\bold{B},\text{ }\bold{C}=\left[ \begin{array}{c c c c} a_{11} & a_{12} & \dots & a_{1n}\\ a_{21} & a_{22} & \dots & a_{2n}\\ \vdots & \vdots & \ddots & a_{3n}\\ a_{m1} & a_{m2} & a_{m3} & a_{mn}  \end{array} \right]
          \end{split}
          \label{31}
        \end{equation}

      \item

        \begin{equation*}
          \begin{split}
            \bold{B}+\bold{C}=b_{ij}+c_{ij}\\
            \bold{A}+\bold{B}=a_{ij}+b_{ij}\\
          \end{split}
          \label{32}
        \end{equation}

      \item

        \begin{equation*}
          \begin{split}
            \bold{A}+(\bold{B}+\bold{C})=a_{ij}+(b_{ij}+c_{ij})\\
            (\bold{A}+\bold{B}) + \bold{C}=(a_{ij}+b_{ij})+c_{ij}\\
          \end{split}
          \label{33}
        \end{equation}

      \item

        \begin{equation*}
          \begin{split}
            a_{ij}+(b_{ij}+c_{ij})=(a_{ij}+b_{ij})+c_{ij}\\
            \therefore \bold{A}+(\bold{B}+\bold{C})=(\bold{A}+\bold{B})+\bold{C}
          \end{split}
          \label{34}
        \end{equation}

    \end{enumerate}

    \setcounter{enumi}{63}
    
  \item
    
    \begin{equation*}
      \begin{split}
        \bold{A}=[a_{ij}]\\
        (c+d)a_{ij}=ca_{ij}+da_{ij}\\
      \end{split}
      \label{35}
    \end{equation}

    \setcounter{enumi}{72}
    
  \item It is symmetric

    \setcounter{enumi}{76}
    
  \item

    \begin{enumerate}

      \item For a sum of matrices, $\bold{A}+\bold{A}^{\intercal}=a_{ij}+a_{ji}$

    \begin{equation*}
      \begin{split}
        \bold{A}+\bold{A}^{\intercal}=a_{ij}+a_{ji}\\
        c(a_{ij}+a_{ji})=ca_{ij}+ca_{ji}\\
        \bold{B}=\frac{1}{2}(\bold{A}+\bold{A}^{\intercal})=\frac{1}{2}\left[ \begin{array}{c c c c} 2a_{11} & a_{12} + a{21} & \dots & a_{1j} + a_{j1}\\ a_{21}+ a_{12} & 2a_{22} & \dots & a_{2j} + a_{j2} \\ \vdots & \vdots & \ddots & a_{3j} + a_{j3}\\ a_{i1}+a_{1i} & a_{i2}+a_{2i} & a_{i3}+ a_{3i} & 2a_{ij}  \end{array} \right]\\
        \bold{B}^{\intercal}=\frac{1}{2}(\bold{A}+\bold{A}^{\intercal})=\frac{1}{2}\left[ \begin{array}{c c c c} 2a_{11} & a_{12} + a{21} & \dots & a_{1j} + a_{j1}\\ a_{21}+ a_{12} & 2a_{22} & \dots & a_{2j} + a_{j2} \\ \vdots & \vdots & \ddots & a_{3j} + a_{j3}\\ a_{i1}+a_{1i} & a_{i2}+a_{2i} & a_{i3}+ a_{3i} & 2a_{ij}  \end{array} \right]\\
        \text{Because }\bold{B}=\bold{B}^{\intercal}\text{, it is symmetric}
      \end{split}
      \label{36}
    \end{equation}


      \item

    \begin{equation*}
      \begin{split}
        \bold{A}-\bold{A}^{\intercal}=a_{ij}-a_{ji}\\
        c(a_{ij}-a_{ji})=ca_{ij}-ca_{ji}\\
        \bold{B}=\frac{1}{2}(\bold{A}-\bold{A}^{\intercal})=\frac{1}{2}\left[ \begin{array}{c c c c} 0 & a_{12} - a{21} & \dots & a_{1j} - a_{j1}\\ a_{21}- a_{12} & 0 & \dots & a_{2j} - a_{j2} \\ \vdots & \vdots & \ddots & a_{3j} - a_{j3}\\ a_{i1}-a_{1i} & a_{i2}-a_{2i} & a_{i3}- a_{3i} & 0  \end{array} \right]\\
        \bold{B}^{\intercal}=\frac{1}{2}\left[ \begin{array}{c c c c} 0 & a_{12} - a{21} & \dots & a_{1j} - a_{j1}\\ a_{21}- a_{12} & 0 & \dots & a_{2j} - a_{j2} \\ \vdots & \vdots & \ddots & a_{3j} - a_{j3}\\ a_{i1}-a_{1i} & a_{i2}-a_{2i} & a_{i3}- a_{3i} & 0  \end{array} \right]\\
        \text{Because }\bold{B}=-\bold{B}^{\intercal}\text{, it is \underline{skew} symmetric}
      \end{split}
      \label{37}
    \end{equation}

    \end{enumerate}

\end{enumerate}

\end{document}

