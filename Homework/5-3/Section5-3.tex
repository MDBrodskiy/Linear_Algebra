%%%%%%%%%%%%%%%%%%%%%%%%%%%%%%%%%%%%%%%%%%%%%%%%%%%%%%%%%%%%%%%%%%%%%%%%%%%%%%%%%%%%%%%%%%%%%%%%%%%%%%%%%%%%%%%%%%%%%%%%%%%%%%%%%%%%%%%%%%%%%%%%%%%%%%%%%%%%%%%%%%%%%%%%%%%%%%%%%%%%%%%%%%%%
% Written By Michael Brodskiy
% Class: Linear Algebra
% Professor: L. Knight
%%%%%%%%%%%%%%%%%%%%%%%%%%%%%%%%%%%%%%%%%%%%%%%%%%%%%%%%%%%%%%%%%%%%%%%%%%%%%%%%%%%%%%%%%%%%%%%%%%%%%%%%%%%%%%%%%%%%%%%%%%%%%%%%%%%%%%%%%%%%%%%%%%%%%%%%%%%%%%%%%%%%%%%%%%%%%%%%%%%%%%%%%%%%

\documentclass[12pt]{article} 
\usepackage{alphalph}
\usepackage[utf8]{inputenc}
\usepackage[russian,english]{babel}
\usepackage{titling}
\usepackage{amsmath}
\usepackage{graphicx}
\usepackage{enumitem}
\usepackage{amssymb}
\usepackage{physics}
\usepackage{tikz}
\usepackage{mathdots}
\usepackage{yhmath}
\usepackage{cancel}
\usepackage{color}
\usepackage{siunitx}
\usepackage{array}
\usepackage{multirow}
\usepackage{gensymb}
\usepackage{tabularx}
\usepackage{booktabs}
\usepackage{pifont}
\newcommand{\xmark}{\ding{55}}
\usetikzlibrary{fadings}
\usetikzlibrary{patterns}
\usetikzlibrary{shadows.blur}
\usetikzlibrary{shapes}
\usepackage[super]{nth}
\usepackage{expl3}
\usepackage[version=4]{mhchem}
\usepackage{hpstatement}
\usepackage{rsphrase}
\usepackage{everysel}
\usepackage{ragged2e}
\usepackage{geometry}
\usepackage{fancyhdr}
\usepackage{cancel}
\usepackage{multicol}
\geometry{top=1.0in,bottom=1.0in,left=1.0in,right=1.0in}
\newcommand{\subtitle}[1]{%
  \posttitle{%
    \par\end{center}
    \begin{center}\large#1\end{center}
    \vskip0.5em}%

}
\usepackage{hyperref}
\hypersetup{
colorlinks=true,
linkcolor=blue,
filecolor=magenta,      
urlcolor=blue,
citecolor=blue,
}

\urlstyle{same}


\title{Linear Algebra 5.3 Homework}
\date{}
\author{Michael Brodskiy\\ \small Instructor: Prof. Knight}

\begin{document}

\maketitle

\begin{enumerate}

    \begin{center}
      \underline{5, 7, 11, 15, 17, 19, 43, 48, 56, 61, 64}
    \end{center}

    \setcounter{enumi}{4}

  \item 

    \begin{enumerate}

      \item $(4)(-1)+(1)(4)=0$, $(4)(-4)+(-17)(-1)+(1)(-1)=0$, $(-1)(-4)+(-1)(4)=0$. The set is orthogonal

      \item $\sqrt{4^2+(-1)^2+1^2}\neq 1$, so the set is not orthonormal

      \item $\begin{vmatrix} 4 & -1 & -4\\ -1 & 0 & -17\\ 1 & 4 & -1  \end{vmatrix}\widetilde{ }\begin{vmatrix} 1 & -17 & 0\\ 0 & 4 & -18\\ 0 & 21 & -1 \end{vmatrix}=4(-1)-(21)(-18)=374$, so it is linearly independent. Therefore, it is a basis

    \end{enumerate}

    \setcounter{enumi}{6}

  \item

    \begin{enumerate}

      \item $\left(-\frac{\sqrt{2}}{6}\right)\left(-\frac{\sqrt{5}}{5}\right)\neq 0$, so it is not orthogonal

      \item Because it is not orthogonal, it is not orthonormal

      \item $\begin{vmatrix} \frac{\sqrt{2}}{3} & 0 & \frac{\sqrt{5}}{5}\\ 0 & \frac{2\sqrt{5}}{5} & 0\\ -\frac{\sqrt{2}}{6} & -\frac{\sqrt{5}}{5} & \frac{1}{2}  \end{vmatrix}=\frac{2\sqrt{5}}{5}\left( \left(\frac{\sqrt{2}}{3}\right)\left( \frac{1}{2}  \right)-\left( \frac{\sqrt{5}}{5}\right)\left( -\frac{\sqrt{2}}{6} \right) \right)\neq0$, so it is linearly independent. Therefore, it is a basis

    \end{enumerate}

    \setcounter{enumi}{10}

  \item

    \begin{enumerate}

      \item $\left(\frac{\sqrt{2}}{2}\right)\left( -\frac{1}{2} \right)+\left( \frac{\sqrt{2}}{2} \right)\left( \frac{1}{2} \right)=0$, $4\left( \frac{\sqrt{2}}{2} \right)\left( 0 \right)=0$, $\left(\frac{\sqrt{2}}{2}\right)\left( -\frac{1}{2} \right)+\left( \frac{\sqrt{2}}{2} \right)\left( \frac{1}{2} \right)=0$, so it is orthogonal

      \item $\sqrt{\left( \frac{\sqrt{2}}{2} \right)^2+\left( \frac{\sqrt{2}}{2} \right)^2}=1$, $\sqrt{\left( -\frac{1}{2} \right)^2+\left( \frac{1}{2} \right)^2+\left( -\frac{1}{2} \right)^2+\left( \frac{1}{2} \right)}=1$, $\sqrt{\left( \frac{\sqrt{2}}{2} \right)^2+\left( \frac{\sqrt{2}}{2} \right)^2}=1$, so it is orthonormal

      \item There are more terms than vectors, so it is linearly dependent

    \end{enumerate}

    \setcounter{enumi}{14}

  \item

    \begin{enumerate}

      \item $(-\sqrt{2})(\sqrt{3})+(\sqrt{2})(\sqrt{3})=0$, so it is orthogonal

      \item $\frac{1}{\sqrt{(\sqrt{3})^2+(\sqrt{3})^2+(\sqrt{3})^2}}(\sqrt{3},\sqrt{3},\sqrt{3})=\left( \frac{\sqrt{3}}{3},\frac{\sqrt{3}}{3},\frac{\sqrt{3}}{3} \right)$, $\frac{1}{\sqrt{\left( -\sqrt{2} \right)^2+\left( \sqrt{2} \right)^2}}(-\sqrt{2},0,\sqrt{2})=$\\$\left( -\frac{\sqrt{2}}{2},0,\frac{\sqrt{2}}{2} \right)\Rightarrow\left\{ \left(\frac{\sqrt{3}}{3},\frac{\sqrt{3}}{3},\frac{\sqrt{3}}{3}\right),\left( -\frac{\sqrt{2}}{2},0,\frac{\sqrt{2}}{2} \right) \right\}$

    \end{enumerate}

    \setcounter{enumi}{16}

  \item $1\cdot x=0$, $x\cdot x^2=0$, $x^2\cdot x^3=0$, $1\cdot x^2=0$, $1\cdot x^3=0$, $x\cdot x^3=0$, and $||1||=1$, $||x||=1$, $||x^2||=1$, $||x^3||=1$, so it is orthonormal

    \setcounter{enumi}{18}

  \item $\begin{bmatrix} -\frac{2\sqrt{13}}{13} & \frac{3\sqrt{13}}{13}\\ \frac{3\sqrt{13}}{13} & \frac{2\sqrt{13}}{13}  \end{bmatrix}\begin{bmatrix} 1\\2  \end{bmatrix}=\begin{bmatrix} \frac{4\sqrt{13}}{13}\\ \frac{7\sqrt{13}}{13}\end{bmatrix}$

    \setcounter{enumi}{42}

  \item $\int_{-1}^1 x\,dx=\left( \frac{x^2}{2} \right)\Big|_{-1}^1=0$ \textcolor{green}{\checkmark}

    \setcounter{enumi}{47}

  \item $2\int_0^1 \left( x^2-\frac{1}{3} \right)^2\,dx=2\int_0^1 x^4-\frac{2}{3}x^2+\frac{1}{9}\,dx=\left( \frac{2x^5}{5}-\frac{4x^3}{9}+\frac{2}{9}x \right)\Big|_0^1=\frac{8}{45}$ \textcolor{green}{\checkmark}

    \setcounter{enumi}{55}

  \item

    \begin{enumerate}

      \item True. The requirement for orthonormality is for each vector to be orthogonal to each other, and for each vector to be a unit vector.

      \item False. Orthogonality does not define linear independence. 

    \end{enumerate}

    \setcounter{enumi}{60}

  \item $\left( \frac{1}{\sqrt{2}} \right)\left( -\frac{1}{\sqrt{3}}  \right)+\left( \frac{1}{\sqrt{2}} \right)\left( \frac{1}{\sqrt{3}} \right)=0\Rightarrow \sqrt{\left( \frac{1}{\sqrt{2}} \right)^2+\left( \frac{1}{\sqrt{2}} \right)^2}=1$\\ and $\sqrt{\left( \frac{1}{\sqrt{3}} \right)^2 + \left( \frac{1}{\sqrt{3}} \right)^2 + \left( \frac{1}{\sqrt{3}} \right)^2}=1$, so it is orthonormal

    \setcounter{enumi}{63}

  \item

    \begin{enumerate}

      \item $\overrightarrow{v}=c_1\overrightarrow{v}_1+c_2\overrightarrow{v}_2+\dots+\overrightarrow{v}_n$

      \item $\langle\overrightarrow{w},\overrightarrow{v}\rangle=\overrightarrow{w}_1\overrightarrow{v}_1+\overrightarrow{w}_2\overrightarrow{v}_2+\dots+\overrightarrow{w}_n\overrightarrow{v}_n$

      \item $\overrightarrow{w}_1\overrightarrow{v}_1+\overrightarrow{w}_2\overrightarrow{v}_2+\dots+\overrightarrow{w}_n\overrightarrow{v}_n=c$

      \item If $\overrightarrow{w}$ is orthogonal, then $\cancel{\overrightarrow{w}_1\overrightarrow{v}_1+\overrightarrow{w}_2\overrightarrow{v}_2+\dots+\overrightarrow{w}_n\overrightarrow{v}_n}=0$

    \end{enumerate}

\end{enumerate}

\end{document}

