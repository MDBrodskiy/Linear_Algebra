%%%%%%%%%%%%%%%%%%%%%%%%%%%%%%%%%%%%%%%%%%%%%%%%%%%%%%%%%%%%%%%%%%%%%%%%%%%%%%%%%%%%%%%%%%%%%%%%%%%%%%%%%%%%%%%%%%%%%%%%%%%%%%%%%%%%%%%%%%%%%%%%%%%%%%%%%%%%%%%%%%%%%%%%%%%%%%%%%%%%%%%%%%%%
% Written By Michael Brodskiy
% Class: Linear Algebra
% Professor: L. Knight
%%%%%%%%%%%%%%%%%%%%%%%%%%%%%%%%%%%%%%%%%%%%%%%%%%%%%%%%%%%%%%%%%%%%%%%%%%%%%%%%%%%%%%%%%%%%%%%%%%%%%%%%%%%%%%%%%%%%%%%%%%%%%%%%%%%%%%%%%%%%%%%%%%%%%%%%%%%%%%%%%%%%%%%%%%%%%%%%%%%%%%%%%%%%

\documentclass[12pt]{article} 
\usepackage{alphalph}
\usepackage[utf8]{inputenc}
\usepackage[russian,english]{babel}
\usepackage{titling}
\usepackage{amsmath}
\usepackage{graphicx}
\usepackage{enumitem}
\usepackage{amssymb}
\usepackage{physics}
\usepackage{tikz}
\usepackage{mathdots}
\usepackage{yhmath}
\usepackage{cancel}
\usepackage{color}
\usepackage{siunitx}
\usepackage{array}
\usepackage{multirow}
\usepackage{gensymb}
\usepackage{tabularx}
\usepackage{booktabs}
\usetikzlibrary{fadings}
\usetikzlibrary{patterns}
\usetikzlibrary{shadows.blur}
\usetikzlibrary{shapes}
\usepackage[super]{nth}
\usepackage{expl3}
\usepackage[version=4]{mhchem}
\usepackage{hpstatement}
\usepackage{rsphrase}
\usepackage{everysel}
\usepackage{ragged2e}
\usepackage{geometry}
\usepackage{fancyhdr}
\usepackage{cancel}
\usepackage{multicol}
\geometry{top=1.0in,bottom=1.0in,left=1.0in,right=1.0in}
\newcommand{\subtitle}[1]{%
  \posttitle{%
    \par\end{center}
    \begin{center}\large#1\end{center}
    \vskip0.5em}%

}
\usepackage{hyperref}
\hypersetup{
colorlinks=true,
linkcolor=blue,
filecolor=magenta,      
urlcolor=blue,
citecolor=blue,
}

\urlstyle{same}


\title{Linear Algebra 2.4 Homework}
\date{}
\author{Michael Brodskiy\\ \small Instructor: Prof. Knight}

% Mathematical Operations:

% Sum: $$\sum_{n=a}^{b} f(x) $$
% Integral: $$\int_{lower}^{upper} f(x) dx$$
% Limit: $$\lim_{x\to\infty} f(x)$$

\begin{document}

\maketitle

\begin{enumerate}

  \item The matrix is elementary, $2R_2\widetilde{\rightarrow}R_2$

    \setcounter{enumi}{2}

  \item The matrix is elementary, $2R_1+R_2\widetilde{\rightarrow}R_2$

    \setcounter{enumi}{4}

  \item The matrix is not elementary 

    \setcounter{enumi}{6}

  \item The matrix is elementary, $R_3-5R_2\widetilde{\rightarrow}R_3$

    \setcounter{enumi}{8}

  \item $E=\left[ \begin{array}{c c c} 0 & 0 & 1\\ 0 & 1 & 0\\ 1 & 0 & 0 \end{array} \right]$

    \setcounter{enumi}{10}

  \item $E=\left[ \begin{array}{c c c} 0 & 0 & 1\\ 0 & 1 & 0\\ 1 & 0 & 0 \end{array} \right]$

    \setcounter{enumi}{14}

  \item $E_1=\left[ \begin{array}{c c c} 1 & 0 & 0 \\ 0 & 1 & 0\\ 6 & 0 & 1 \end{array} \right],\text{ }E_2=\left[ \begin{array}{c c c} 1 & 0 & 0 \\ 0 & 1 & 0\\ 0 & 0 & \frac{1}{2} \end{array} \right],\text{ }E_3=\left[ \begin{array}{c c c} 1 & 0 & 0 \\ 0 & \frac{1}{4} & 0\\ 0 & 0 & 1 \end{array} \right]$

    \begin{equation*}
      E_3E_2E_1\bold{A}=\left[ \begin{array}{c c c c} 1 & -2 & -1 & 0\\ 0 & 1 & 2 & -1\\  0 & 0 & 1 & \frac{1}{2}  \end{array} \right]
      \label{1}
    \end{equation}

    \setcounter{enumi}{16}

  \item $E_1=\left[ \begin{array}{c c c c} 1 & 0 & 0 & 0\\ 0 & 1 & 0 & 0\\ 0 & 0 & 1 & 0  \\ 0 & 0 & 1 & 1\end{array} \right]\\
    E_2=\left[ \begin{array}{c c c c} 1 & 0 & 0 & 0\\ 0 & 1 & -3 & 0\\ 0 & 0 & 1 & 0\\ 0 & 0 & 0 & 1\end{array} \right]\\
    E_3=\left[ \begin{array}{c c c c} 1 & 0 & 0 & 0\\ 0 & 1 & 0 & 0\\ 0 & 0 & 4 & 0\\ 0 & 0 & 0 & 1\end{array} \right]\\
    E_4=\left[ \begin{array}{c c c c} 1 & 0 & 0 & 0\\ 0 & 1 & 0 & 0\\ 2 & 0 & 1 & 0\\ 0 & 0 & 0 & 1\end{array} \right]\\
    E_5=\left[ \begin{array}{c c c c} 1 & 0 & 0 & 0\\ 0 & 1 & 0 & 0\\ 0 & 3 & 1 & 0\\ 0 & 0 & 0 & 1\end{array} \right]\\
    E_6=\left[ \begin{array}{c c c c} 1 & 0 & 0 & 0\\ 0 & 1 & 0 & 0\\ 0 & 0 & -\frac{1}{10} & 0\\ 0 & 0 & 0 & 1\end{array} \right]\\
    E_7=\left[ \begin{array}{c c c c} 2 & 0 & 0 & 0\\ 0 & 1 & 6 & 0\\ 0 & 0 & 1 & 0\\ 0 & 0 & 0 & 1\end{array} \right]\\
    E_8=\left[ \begin{array}{c c c c} 1 & 0 & 0 & 0\\ 0 & \frac{1}{2} & 0 & 0\\ 0 & 0 & 1 & 0\\ 0 & 0 & 0 & 1\end{array} \right]\\
    E_9=\left[ \begin{array}{c c c c} 1 & -1 & 0 & 0\\ 0 & 1 & 0 & 0\\ 0 & 0 & 1 & 0\\ 0 & 0 & 0 & 1\end{array} \right]\\
    E_{10}=\left[ \begin{array}{c c c c} -\frac{1}{2} & 0 & 0 & 0\\ 0 & 1 & 0 & 0\\ 0 & 0 & 1 & 0\\ 0 & 0 & 0 & 1\end{array} \right]\\
    E_{11}=\left[ \begin{array}{c c c c} 1 & 0 & 0 & 0\\ 0 & 1 & 0 & 0\\ 0 & 0 & 1 & 0\\ 0 & 0 & -1 & 1\end{array} \right]$

    \begin{equation*}
      E_{11}E_{10}E_9E_8E_7E_6E_5E_4E_3E_2E_1\bold{A}=\left[ \begin{array}{c c c} 1 & 0 & 0\\ 0 & 1 & 0\\ 0 & 0 & 1\\ 0 & 0 & 0  \end{array}  \right]
      \label{2}
    \end{equation}

    \setcounter{enumi}{18}

  \item

    \begin{equation*}
      \begin{split}
        \frac{1}{0-1}\left[ \begin{array}{c c} 0 & -1\\ -1 & 0 \end{array} \right]\\
        \bold{A}^{-1}=\left[ \begin{array}{c c} 0 & 1\\ 1 & 0\\ \end{array}  \right]
      \end{split}
      \label{3}
    \end{equation}

    \setcounter{enumi}{20}

  \item

    \begin{equation*}
      \begin{split}
        \left(\left[ \begin{array}{c c c} 0 & 0 & 1\\ 0 & 1 & 0\\ 1 & 0 & 0 \end{array} \right]\right)^{-1}=\left[ \begin{array}{c c c} 0 & 0 & 1^{-1}\\ 0 & 1^{-1} & 0\\ 1^{-1} & 0 & 0 \end{array} \right]\\
        \bold{A}^{-1}=\left[ \begin{array}{c c c} 0 & 0 & 1\\ 0 & 1 & 0\\ 1 & 0 & 0 \end{array} \right]
      \end{split}
      \label{4}
    \end{equation}

    \setcounter{enumi}{22}

  \item

    \begin{equation*}
      \begin{split}
        \left(\left[ \begin{array}{c c c} k & 0 & 0\\ 0 & 1 & 0\\ 0 & 0 & 1 \end{array} \right]\right)^{-1}=\left[ \begin{array}{c c c} k^{-1} & 0 & 0\\ 0 & 1^{-1} & 0\\ 0 & 0 & 1^{-1} \end{array} \right]\\
        \bold{A}^{-1}=\left[ \begin{array}{c c c} \frac{1}{k} & 0 & 0\\ 0 & 1 & 0\\ 0 & 0 & 1 \end{array} \right],\,\,k\neq0
      \end{split}
      \label{5}
    \end{equation}

    \setcounter{enumi}{24}

  \item $E_1=\left[ \begin{array}{c c} 0 & 1\\ 0 & 1 \end{array} \right]\\
    E_2=\left[ \begin{array}{c c} 1 & 0\\ 1 & 0 \end{array} \right]\\
    E_3=\left[ \begin{array}{c c} 1 & 0\\ -3 & 1 \end{array} \right]\\
    E_4=\left[ \begin{array}{c c} 1 & 0\\ 0 & -\frac{1}{2} \end{array} \right]\\$


    \begin{equation*}
      \begin{split}
        E_4E_3E_2E_1\bold{A}=\left[ \begin{array}{c c} 1 & 0\\ 0 & 1 \end{array}\right]\\
        E_4E_3E_2E_1=\left[ \begin{array}{c c} 0 & 1\\ -\frac{1}{2} & \frac{3}{2}  \end{array} \right]\\
        \therefore E_4E_3E_2E_1=\bold{A}^{-1}\\
      \end{split}
      \label{6}
    \end{equation}

    \setcounter{enumi}{26}

  \item $E_1=\left[ \begin{array}{c c c} 1 & 0 & 0\\ 0 & 1 & 0\\ 0 & 0 & \frac{1}{4} \end{array} \right]\\
        E_2=\left[ \begin{array}{c c c} 1 & 0 & 0\\ 0 & 1 & 1\\ 0 & 0 & 1 \end{array} \right]\\
        E_3=\left[ \begin{array}{c c c} 1 & 0 & 0\\ 0 & \frac{1}{6} & 0\\ 0 & 0 & 1 \end{array} \right]\\
        E_4=\left[ \begin{array}{c c c} 1 & 0 & 1\\ 0 & 1 & 0\\ 0 & 0 & 1 \end{array} \right]\\$

    \begin{equation*}
      \begin{split}
        E_4E_3E_2E_1\bold{A}=\left[ \begin{array}{c c c} 1 & 0 & 0\\ 0 & 1 & 0\\ 0 & 0 & 1 \end{array}\right]\\
        E_4E_3E_2E_1=\left[ \begin{array}{c c c} 1 & 0 & \frac{1}{4}\\ 0 & \frac{1}{6} & \frac{1}{24}\\ 0 & 0 & \frac{1}{4}  \end{array}\right]\\
        \therefore E_4E_3E_2E_1=\bold{A}^{-1}\\
      \end{split}
      \label{7}
    \end{equation}

    \setcounter{enumi}{30}

  \item $E_1=\left[ \begin{array}{c c} 1 & 0\\ 0 & 1 \end{array} \right]\\
    E_2=\left[ \begin{array}{c c} 1 & 0\\ 0 & -1 \end{array} \right]\\$
    E_3=\left[ \begin{array}{c c} 1 & 0\\ 3 & 1 \end{array} \right]\\
    E_4=\left[ \begin{array}{c c} 1 & 1\\ 0 & 1 \end{array} \right]\\

    \begin{equation*}
      \begin{split}
        E_4E_3E_2E_1=\left[ \begin{array}{c c} 4 & -1\\ 3 & -1 \end{array}\right]\\
      \end{split}
      \label{8}
    \end{equation}

    \setcounter{enumi}{32}

  \item $E_1=\left[ \begin{array}{c c c} 1 & 0 & 0\\ 0 & 1 & 0\\ 0 & 0 & 1\\ \end{array} \right]\\
        E_2=\left[ \begin{array}{c c c} 1 & -2 & 0\\ 0 & 1 & 0\\ 0 & 0 & 1\\ \end{array} \right]\\
        E_3=\left[ \begin{array}{c c c} 1 & 0 & 0\\ 0 & -1 & 0\\ 0 & 0 & 1\\ \end{array} \right]\\
        E_4=\left[ \begin{array}{c c c} 1 & 0 & 0\\ 1 & 1 & 0\\ 0 & 0 & 1\\ \end{array} \right]\\
        E_5=\left[ \begin{array}{c c c} 1 & 0 & 0\\ 0 & -1 & 0\\ 0 & 0 & 1\\ \end{array} \right]\\
        $

    \begin{equation*}
      \begin{split}
        E_5E_4E_3E_2E_1=\left[ \begin{array}{c c c} 1 & -2 & 0\\ -1 & 3 & 0\\ 0 & 0 & 1 \end{array}\right]\\
      \end{split}
      \label{9}
    \end{equation}

    \setcounter{enumi}{36}

  \item It is \textbf{NOT} always elementary. This is evident because elementary matrices may be used to come up with non-elementary matrices, such as inverses, which means that the product is not always elementary.

    \setcounter{enumi}{39}

  \item 
    
    \begin{equation*}
      \begin{split}
        \bold{A}=\left[ \begin{array}{c c c} 1+ab & a & 0\\ b & 1 & 0\\ 0 & 0 & c  \end{array} \right]\\
        E_1=\left[ \begin{array}{c c c} 1 & -a & 0\\ 0 & 1 & 0\\ 0 & 0 & 1  \end{array} \right]\\
        E_2=\left[ \begin{array}{c c c} 1 & 0 & 0\\ -b & 1 & 0\\ 0 & 0 & 1  \end{array} \right]\\
        E_3=\left[ \begin{array}{c c c} 1 & 0 & 0\\ 0 & 1 & 0\\ 0 & 0 & \frac{1}{c}  \end{array} \right]\\
        E_3E_2E_1=\left[ \begin{array}{c c c} 1 & -a & 0\\ -b & ab+1 & 0\\ 0 & 0 & \frac{1}{c}  \end{array} \right]\\
        E_3E_2E_1\bold{A}=\left[ \begin{array}{c c c} 1 & 0 & 0\\ 0 & 1 & 0\\ 0 & 0 & 1\\ \end{array}\right]
      \end{split}
      \label{10}
    \end{equation}

  \item

    \begin{enumerate}

      \item True. It has one or less modifications done to it, with respect to the identity matrix.

      \item False. This modification can possibly affect more than one number, which would mean there is more than one modification done, which means it is no longer an elementary matrix.

      \item True. An inversed elementary matrix produces a matrix with one or less modification, meaning that it is an elementary matrix itself.

    \end{enumerate}

\end{enumerate}

\end{document}

