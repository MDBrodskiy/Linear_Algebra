%%%%%%%%%%%%%%%%%%%%%%%%%%%%%%%%%%%%%%%%%%%%%%%%%%%%%%%%%%%%%%%%%%%%%%%%%%%%%%%%%%%%%%%%%%%%%%%%%%%%%%%%%%%%%%%%%%%%%%%%%%%%%%%%%%%%%%%%%%%%%%%%%%%%%%%%%%%%%%%%%%%%%%%%%%%%%%%%%%%%%%%%%%%%
% Written By Michael Brodskiy
% Class: Linear Algebra
% Professor: L. Knight
%%%%%%%%%%%%%%%%%%%%%%%%%%%%%%%%%%%%%%%%%%%%%%%%%%%%%%%%%%%%%%%%%%%%%%%%%%%%%%%%%%%%%%%%%%%%%%%%%%%%%%%%%%%%%%%%%%%%%%%%%%%%%%%%%%%%%%%%%%%%%%%%%%%%%%%%%%%%%%%%%%%%%%%%%%%%%%%%%%%%%%%%%%%%

\documentclass[12pt]{article} 
\usepackage{alphalph}
\usepackage[utf8]{inputenc}
\usepackage[russian,english]{babel}
\usepackage{titling}
\usepackage{amsmath}
\usepackage{graphicx}
\usepackage{enumitem}
\usepackage{amssymb}
\usepackage{physics}
\usepackage{tikz}
\usepackage{mathdots}
\usepackage{yhmath}
\usepackage{cancel}
\usepackage{color}
\usepackage{siunitx}
\usepackage{array}
\usepackage{multirow}
\usepackage{gensymb}
\usepackage{tabularx}
\usepackage{booktabs}
\usepackage{pifont}
\newcommand{\xmark}{\ding{55}}
\usetikzlibrary{fadings}
\usetikzlibrary{patterns}
\usetikzlibrary{shadows.blur}
\usetikzlibrary{shapes}
\usepackage[super]{nth}
\usepackage{expl3}
\usepackage[version=4]{mhchem}
\usepackage{hpstatement}
\usepackage{rsphrase}
\usepackage{everysel}
\usepackage{ragged2e}
\usepackage{geometry}
\usepackage{fancyhdr}
\usepackage{cancel}
\usepackage{multicol}
\geometry{top=1.0in,bottom=1.0in,left=1.0in,right=1.0in}
\newcommand{\subtitle}[1]{%
  \posttitle{%
    \par\end{center}
    \begin{center}\large#1\end{center}
    \vskip0.5em}%

}
\usepackage{hyperref}
\hypersetup{
colorlinks=true,
linkcolor=blue,
filecolor=magenta,      
urlcolor=blue,
citecolor=blue,
}

\urlstyle{same}


\title{Linear Algebra 6.1 Homework}
\date{}
\author{Michael Brodskiy\\ \small Instructor: Prof. Knight}

\begin{document}

\maketitle

\begin{enumerate}

    \begin{center}
      \underline{1, 6, 11, 13, 16, 19, 21, 23, 25, 29, 33, 36, 37, 39, 45, 48, 55, 56, 57, 58, 63, 65, 69}
    \end{center}

  \item
    
    \setcounter{enumi}{5}

  \item

    \setcounter{enumi}{10}

  \item

    \setcounter{enumi}{12}

  \item

    \setcounter{enumi}{15}

  \item

    \setcounter{enumi}{18}

  \item

    \setcounter{enumi}{20}

  \item

    \setcounter{enumi}{22}

  \item

    \setcounter{enumi}{24}

  \item

    \setcounter{enumi}{28}

  \item

    \setcounter{enumi}{32}

  \item

    \setcounter{enumi}{35}

  \item

  \item

    \setcounter{enumi}{38}

  \item

    \setcounter{enumi}{44}

  \item

    \setcounter{enumi}{47}

  \item

    \setcounter{enumi}{54}

  \item

  \item

  \item

  \item

    \setcounter{enumi}{62}

  \item

    \setcounter{enumi}{64}

  \item

    \setcounter{enumi}{68}

  \item

\end{enumerate}

\end{document}

