%%%%%%%%%%%%%%%%%%%%%%%%%%%%%%%%%%%%%%%%%%%%%%%%%%%%%%%%%%%%%%%%%%%%%%%%%%%%%%%%%%%%%%%%%%%%%%%%%%%%%%%%%%%%%%%%%%%%%%%%%%%%%%%%%%%%%%%%%%%%%%%%%%%%%%%%%%%%%%%%%%%%%%%%%%%%%%%%%%%%%%%%%%%%
% Written By Michael Brodskiy
% Class: Linear Algebra
% Professor: L. Knight
%%%%%%%%%%%%%%%%%%%%%%%%%%%%%%%%%%%%%%%%%%%%%%%%%%%%%%%%%%%%%%%%%%%%%%%%%%%%%%%%%%%%%%%%%%%%%%%%%%%%%%%%%%%%%%%%%%%%%%%%%%%%%%%%%%%%%%%%%%%%%%%%%%%%%%%%%%%%%%%%%%%%%%%%%%%%%%%%%%%%%%%%%%%%

\documentclass[12pt]{article} 
\usepackage{alphalph}
\usepackage[utf8]{inputenc}
\usepackage[russian,english]{babel}
\usepackage{titling}
\usepackage{amsmath}
\usepackage{graphicx}
\usepackage{enumitem}
\usepackage{amssymb}
\usepackage{physics}
\usepackage{tikz}
\usepackage{mathdots}
\usepackage{yhmath}
\usepackage{cancel}
\usepackage{color}
\usepackage{siunitx}
\usepackage{array}
\usepackage{multirow}
\usepackage{gensymb}
\usepackage{tabularx}
\usepackage{booktabs}
\usetikzlibrary{fadings}
\usetikzlibrary{patterns}
\usetikzlibrary{shadows.blur}
\usetikzlibrary{shapes}
\usepackage[super]{nth}
\usepackage{expl3}
\usepackage[version=4]{mhchem}
\usepackage{hpstatement}
\usepackage{rsphrase}
\usepackage{everysel}
\usepackage{ragged2e}
\usepackage{geometry}
\usepackage{fancyhdr}
\usepackage{cancel}
\usepackage{multicol}
\geometry{top=1.0in,bottom=1.0in,left=1.0in,right=1.0in}
\newcommand{\subtitle}[1]{%
  \posttitle{%
    \par\end{center}
    \begin{center}\large#1\end{center}
    \vskip0.5em}%

}
\usepackage{hyperref}
\hypersetup{
colorlinks=true,
linkcolor=blue,
filecolor=magenta,      
urlcolor=blue,
citecolor=blue,
}

\urlstyle{same}


\title{Linear Algebra 3.1 Homework}
\date{}
\author{Michael Brodskiy\\ \small Instructor: Prof. Knight}

\begin{document}

\maketitle

\begin{enumerate}


    \setcounter{enumi}{2}

  \item $2(4)-1(3)=5$

    \setcounter{enumi}{4}

  \item $5(3)+12= 27$

    \setcounter{enumi}{10}

  \item $(\lambda-3)(\lambda-1)-4(2)=\lambda^2 -4\lambda-5$

    \setcounter{enumi}{12}

  \item

    \begin{multicols}{2}

      \begin{enumerate}

        \item $M_{11}=4$

        \item $M_{12}=3$

        \item $M_{21}=2$

        \item $M_{22}=1$

      \end{enumerate}


      \begin{enumerate}

        \item $C_{11}=4$

        \item $C_{12}=-3$

        \item $C_{21}=-2$

        \item $C_{22}=1$

      \end{enumerate}

    \end{multicols}

    \setcounter{enumi}{14}

  \item

    \begin{multicols}{2}

      \begin{enumerate}

        \item $M_{11}=23$

        \item $M_{12}=-8$

        \item $M_{13}=-22$

        \item $M_{21}=5$

        \item $M_{22}=-5$

        \item $M_{23}=5$

        \item $M_{31}=7$

        \item $M_{32}=-22$

        \item $M_{33}=-23$

      \end{enumerate}

      \begin{enumerate}

        \item $C_{11}=23$

        \item $C_{12}=8$

        \item $C_{13}=-22$

        \item $C_{21}=-5$

        \item $C_{22}=-5$

        \item $C_{23}=-5$

        \item $C_{31}=7$

        \item $C_{32}=22$

        \item $C_{33}=-23$

      \end{enumerate}

    \end{multicols}

    \setcounter{enumi}{16}

  \item

    \begin{enumerate}

      \item $4(-5)+5(-5)+6(-5)=-75$

      \item $2(8)+5(-5)-3(22)=-75$

    \end{enumerate}

    \setcounter{enumi}{18}

  \item About Row 2: $3[-1(3(4)-4(-2))]+2(1)=-58$

    \setcounter{enumi}{24}

  \item About Row 2: $3[-1(y+1)]+2(x+1)=-3y+2x-1$

    \setcounter{enumi}{26}

  \item About Column 1: $5[6(2)+12(-1)]+4[3(2)+6(-1)]=0$

    \setcounter{enumi}{28}

  \item About Row 1:

    \begin{enumerate}

      \item $w\{-15[32(17)]-24[-840-396]+30[32(46)]\}$

      \item $-x\{21[32(17)]-24[350+40(18)]+30[-32(50)]\}$

      \item $y\{21[-840-396]+15[350+40(18)]+30[-220+40(24)]\}$

      \item $-z\{21[32(46)]+15[-32(50)]+24[-220+24(40)]\}$

        \begin{center}
           $=65,664w+62,256x+12,294-24,672z$
        \end{center}

    \end{enumerate}

    \setcounter{enumi}{40}

  \item About Column 1: $5[0(-2)-6(0(2)+0(1))+0(2)]=0$

    \setcounter{enumi}{42}

  \item

    \begin{enumerate}

      \item False:

        \begin{equation*}
          \begin{split}
            \begin{vmatrix} a_{11} & a_{12}\\ a_{21} & a_{22}  \end{vmatrix}=a_{11}a_{22}-a_{12}a_{21}
          \end{split}
          \label{1}
        \end{equation}

  \item True. In such a case, the only possible way to find a determinant is if it equals the first (and only) entry.

  \item False. That is the definition of a minor. A cofactor could either be equal to the statement, or the negative version of the statement.

    \end{enumerate}

  \item 

    \begin{enumerate}

      \item False. One needs to form the prodcut of the diagonal entries, not the sum.

      \item True. Generally, it is better to expand on a row or column with the most zeros, but any row or column would work.

      \item True. Because the formula involves multiplying by the entry at the $ij$th point, multiplying by zero would result in zero, so this is true.

    \end{enumerate}

  \item $(x+3)(x+2)-2=0\rightarrow x^2+5x+4=0\rightarrow (x+1)(x+4)=0\rightarrow x=-1,-4$

    \setcounter{enumi}{50}

  \item $(\lambda)\left( (\lambda^2+\lambda)-2 \right)=0\rightarrow \lambda\left( \lambda^2+\lambda-2 \right)=0\rightarrow \lambda(\lambda-1)(\lambda+2)=0\rightarrow\lambda=0,1,-2$

    \setcounter{enumi}{62}

  \item $wz-xy=-(xy-wz)$ True

  \item $cwz-cxy=c(wz-xy)$ True

  \item $wz-xy=w(z+cy)-y(x+cw)\rightarrow wz+\cancel{cyw}-xy\cancel{-cyw}$ True

    \setcounter{enumi}{66}

  \item 

    \begin{equation*}
      \begin{split}
        \begin{vmatrix} 1 & x & x^2\\ 1 & y & y^2\\ 1 & z & z^2\\ \end{vmatrix}\\
        R_2-R_3\widetilde{\rightarrow}R_2\text{ and }R_1-R_3\widetilde{\rightarrow}R_1\\
        \begin{vmatrix} 0 & x-z & x^2-z^2\\ 0 & y-z & y^2-z^2\\ 1 & z & z^2\\ \end{vmatrix}=(x-z)(y^2-z^2)-(y-z)(x^2-z^2)\\
        (x-z)(y^2-z^2)-(y-z)(x^2-z^2)= (x-z)(z+y)(y-z)-(y-z)(x+z)(x-z)\\
        (x-z)(z+y)(y-z)-(y-z)(x+z)(x-z)=(x-z)(y-z)\left( z+y-x-z \right)\\
        =(x-z)(y-z)\left( z+y-x-z \right)=(x-z)(y-z)(y-x)\\
        (x-z)(y-z)(y-x)=(z-x)(z-y)(y-x)\\
      \end{split}
      \label{2}
    \end{equation}

\end{enumerate}

\end{document}

