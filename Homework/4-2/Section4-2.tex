%%%%%%%%%%%%%%%%%%%%%%%%%%%%%%%%%%%%%%%%%%%%%%%%%%%%%%%%%%%%%%%%%%%%%%%%%%%%%%%%%%%%%%%%%%%%%%%%%%%%%%%%%%%%%%%%%%%%%%%%%%%%%%%%%%%%%%%%%%%%%%%%%%%%%%%%%%%%%%%%%%%%%%%%%%%%%%%%%%%%%%%%%%%%
% Written By Michael Brodskiy
% Class: Linear Algebra
% Professor: L. Knight
%%%%%%%%%%%%%%%%%%%%%%%%%%%%%%%%%%%%%%%%%%%%%%%%%%%%%%%%%%%%%%%%%%%%%%%%%%%%%%%%%%%%%%%%%%%%%%%%%%%%%%%%%%%%%%%%%%%%%%%%%%%%%%%%%%%%%%%%%%%%%%%%%%%%%%%%%%%%%%%%%%%%%%%%%%%%%%%%%%%%%%%%%%%%

\documentclass[12pt]{article} 
\usepackage{alphalph}
\usepackage[utf8]{inputenc}
\usepackage[russian,english]{babel}
\usepackage{titling}
\usepackage{amsmath}
\usepackage{graphicx}
\usepackage{enumitem}
\usepackage{amssymb}
\usepackage{physics}
\usepackage{tikz}
\usepackage{mathdots}
\usepackage{yhmath}
\usepackage{cancel}
\usepackage{color}
\usepackage{siunitx}
\usepackage{array}
\usepackage{multirow}
\usepackage{gensymb}
\usepackage{tabularx}
\usepackage{booktabs}
\usepackage{pifont}
\newcommand{\xmark}{\ding{55}}
\usetikzlibrary{fadings}
\usetikzlibrary{patterns}
\usetikzlibrary{shadows.blur}
\usetikzlibrary{shapes}
\usepackage[super]{nth}
\usepackage{expl3}
\usepackage[version=4]{mhchem}
\usepackage{hpstatement}
\usepackage{rsphrase}
\usepackage{everysel}
\usepackage{ragged2e}
\usepackage{geometry}
\usepackage{fancyhdr}
\usepackage{cancel}
\usepackage{multicol}
\geometry{top=1.0in,bottom=1.0in,left=1.0in,right=1.0in}
\newcommand{\subtitle}[1]{%
  \posttitle{%
    \par\end{center}
    \begin{center}\large#1\end{center}
    \vskip0.5em}%

}
\usepackage{hyperref}
\hypersetup{
colorlinks=true,
linkcolor=blue,
filecolor=magenta,      
urlcolor=blue,
citecolor=blue,
}

\urlstyle{same}


\title{Linear Algebra 4.2 Homework}
\date{}
\author{Michael Brodskiy\\ \small Instructor: Prof. Knight}

\begin{document}

\maketitle

\begin{enumerate}

  \item $(0,0,0,0)$

    \setcounter{enumi}{2}

  \item $\left[ \begin{array}{ccc} 0 & 0 & 0\\ 0 & 0 & 0\\ 0 & 0 & 0\\ 0 & 0 & 0\\  \end{array} \right]$

    \setcounter{enumi}{4}

  \item $0$ or $0+0x+0x^2+0x^3$

    \setcounter{enumi}{6}

  \item $(a,b,c)+(-a,-b,-c)=(0,0,0)$

    \setcounter{enumi}{8}

  \item $\left[ \begin{array}{ccc} a & b & c\\ d & e & f \end{array} \right]+\left[ \begin{array}{ccc} -a & -b & -c\\ -d & -e & -f \end{array} \right]=\left[ \begin{array}{ccc} 0 & 0 & 0\\ 0 & 0 & 0 \end{array} \right]$

    \setcounter{enumi}{10}

  \item $a+bx+cx^2+dx^3+ex^4+(-a-bx-cx^2-dx^3-ex^4)=0$

    \setcounter{enumi}{12}

  \item $M_{4,6}$ meets all axioms, and, therefore, is a vector space

    \setcounter{enumi}{14}

  \item $P_3$ does not meet axiom one, and, therefore, is not a vector space. (For example, if $v_{1}=1-x^3$ and $v_{2}=1+x^2+x^3$, then $v_1+v_2=2+x^2$, and is not in $P_3$)

    \setcounter{enumi}{20}

  \item The set $\left\{ (x,y): x\geq0,y\text{ is a real number} \right\}$ is not a vector space because it fails axiom six. If $(x,y)=(1,1)$, and $c=-1$, then $c(x,y)$ is not in the vector space

    \setcounter{enumi}{23}

  \item The set $\left\{ \left( x,\frac{1}{2}x \right): x\text{ is a real number} \right\}$ meets all 10 axioms, and is therefore a vector space

    \setcounter{enumi}{25}

  \item The set of all $2\times2$ matrices of the form $\left[ \begin{array}{cc} a & b\\ c & 1  \end{array} \right]$ fails axiom six, and, therefore, is not a vector space (ex. $k=-1\Rightarrow k\left[ \begin{array}{cc} a & b\\ c & 1  \end{array} \right]=\left[ \begin{array}{cc} -a & -b\\ -c & -1  \end{array} \right]$, and is no longer in $\bold{V}$)

  \item The set of all $3\times3$ matrices of the form $\left[ \begin{array}{c c c} 0 & a & b\\ c & 0 & d\\ e & f & 0  \end{array} \right]$ meets all 10 axioms and is, therefore, a vector space

    \setcounter{enumi}{33}

  \item The set of all $3\times3$ upper triangular matrices meet all 10 axioms, and, therefore, are vector spaces

  \item $C\left[ 0,1 \right]$, the set of all continuous functions defined on the interval $[0,1]$ meet all 10 axioms, and, therefore, are vector spaces

  \item $C\left[ -1,1 \right]$, the set of all continuous functions defined on the interval $[-1,1]$ meet all 10 axioms, and, therefore, are vector spaces

  \item By the definition given, it is not a vector space, because it fails axioms 4,5,7, and 8.

    \begin{enumerate}

      \item $x_1+y_1=x_1y_1$ is in $\bold{V}$ \textcolor{green}{\checkmark}

      \item $x_1+y_1=x_1y_1=y_1x_1=y_1+x_1$ is true \textcolor{green}{\checkmark}

      \item  $x_1+(y_1+z_1)=x_1+y_1z_1=x_1y_1z_1=x_1y_1+z_1=(x_1+y_1)+z_1$ is true \textcolor{green}{\checkmark}

      \item $x_1+0=x_1\cdot0$ is not true \textcolor{red}{\xmark}

      \item $x_1+(-a)=0\Rightarrow-ax_1$ is not true \textcolor{red}{\xmark}

      \item $cx_1=x_1^c$ is in $\bold{V}$ \textcolor{green}{\checkmark}

      \item $c(x_1+y_1)\neq cx_1+cy_1$ \textcolor{red}{\xmark}

          \begin{enumerate}

            \item $c(x_1+y_1)=(x_1+y_1)^c$

                \item $cx_1+cy_1=x_1^c+y_1^c$

            \end{enumerate}

          \item $(c+d)x_1\neq cx_1+dx_1$ \textcolor{red}{\xmark}

          \begin{enumerate}

            \item $(c+d)x_1=x_1^{(c+d)}$

                \item $cx_1+dx_1=x_1^c+x_1^d$

            \end{enumerate}

          \item $c(dx_1)=(cd)x_1$ \textcolor{green}{\checkmark}

          \begin{enumerate}

            \item $c(dx_1)=cx_1^d=x_1^{cd}$

            \item $(cd)x_1=x_1^{cd}$

            \end{enumerate}

          \item $1x_1=x_1^1$ is true \textcolor{green}{\checkmark}

      \end{enumerate}

    \setcounter{enumi}{39}

  \item $M_{2,2}$ is a vector space because:

    \begin{enumerate}

      \item $\begin{bmatrix} a & b\\ c & d  \end{bmatrix}+\begin{bmatrix} e & f\\ g & h  \end{bmatrix}=\begin{bmatrix} a+e & b+f\\ c+g & d+h  \end{bmatrix}$ is in $\bold{V}$ \textcolor{green}{\checkmark}

      \item $\begin{bmatrix} a & b\\ c & d  \end{bmatrix}+\begin{bmatrix} e & f\\ g & h  \end{bmatrix}=\begin{bmatrix} e & f\\ g & h  \end{bmatrix}+\begin{bmatrix} a & b\\ c & d  \end{bmatrix}$ is true \textcolor{green}{\checkmark}

      \item $\begin{bmatrix} a & b\\ c & d  \end{bmatrix}+\left(\begin{bmatrix} e & f\\ g & h  \end{bmatrix}+\begin{bmatrix} i & j\\ k & l  \end{bmatrix}\right)=\left(\begin{bmatrix} a & b\\ c & d  \end{bmatrix}+\begin{bmatrix} e & f\\ g & h  \end{bmatrix}\right)+\begin{bmatrix} i & j\\ k & l  \end{bmatrix}$ is true \textcolor{green}{\checkmark} 

      \item $\begin{bmatrix} a & b\\ c & d  \end{bmatrix}+\begin{bmatrix} 0 & 0\\ 0 & 0  \end{bmatrix}=\begin{bmatrix} a & b\\ c & d  \end{bmatrix}$ exists \textcolor{green}{\checkmark}

      \item $\begin{bmatrix} a & b\\ c & d  \end{bmatrix}+\begin{bmatrix} -a & -b\\ -c & -d  \end{bmatrix}=0$ exists \textcolor{green}{\checkmark}

      \item $k\begin{bmatrix} a & b\\ c & d  \end{bmatrix}=\begin{bmatrix} ka & kb\\ kc & kd  \end{bmatrix}$ is in $\bold{V}$ \textcolor{green}{\checkmark}

      \item $k\left( \begin{bmatrix} a & b\\ c & d  \end{bmatrix} + \begin{bmatrix} e & f\\ g & h  \end{bmatrix}\right)=k\begin{bmatrix} a & b\\ c & d  \end{bmatrix}+k\begin{bmatrix} e & f\\ g & h  \end{bmatrix}$ is true \textcolor{green}{\checkmark}

      \item $(k+l)\begin{bmatrix} a & b\\ c & d  \end{bmatrix}=k\begin{bmatrix} a & b\\ c & d  \end{bmatrix}+l\begin{bmatrix} a & b\\ c & d  \end{bmatrix}$ is true \textcolor{green}{\checkmark}

      \item $k\left( l\begin{bmatrix} a & b\\ c & d  \end{bmatrix} \right)=(kl)\begin{bmatrix} a & b\\ c & d  \end{bmatrix}$ is true \textcolor{green}{\checkmark}

      \item $1\left( \begin{bmatrix} a & b\\ c & d  \end{bmatrix} \right)=\begin{bmatrix} a & b\\ c & d  \end{bmatrix}$ is true \textcolor{green}{\checkmark}

      \end{enumerate}

  \item

    \begin{enumerate}

      \item It is not a vector space because axiom eight fails. For example, if $c=5$ and $d=10$, $(5+10)(x_1,y_1)=(15x_1,y_1)$, while $5(x_1,y_1)+10(x_1,y_1)=(15x_1,2y_1)$

    \end{enumerate}

  \item

    \begin{enumerate}

        \setcounter{enumii}{3}

      \item By this $\mathbb{R}^3$ definition, it is a vector space

        \begin{enumerate}

          \item $(x_1,y_1,z_1)+(x_2,y_2,z_2)=(x_1+x_2+1,y_1+y_2+1,z_1+z_2+1)$ is in $\bold{V}$ \textcolor{green}{\checkmark}

            \item $\overrightarrow{\bold{v}}_1+\overrightarrow{\bold{v}}_2=\overrightarrow{\bold{v}}_2+\overrightarrow{\bold{v}}_1$ is true \textcolor{green}{\checkmark} 

          \item $\overrightarrow{\bold{v}}_1+\left( \overrightarrow{\bold{v}}_2+\overrightarrow{\bold{v}}_3 \right)=(\overrightarrow{\bold{v}}_1+\overrightarrow{\bold{v}}_2)+\overrightarrow{\bold{v}}_3$ is true \textcolor{green}{\checkmark}

          \item $\left( x_1,y_1,z_1 \right) + \overrightarrow{\bold{o}}= \left( x_1+o_1+1,y_2+o_2+1,z_1+o_3+1 \right)\Rightarrow\overrightarrow{\bold{o}}=(-1,-1,-1)$ \textcolor{green}{\checkmark}

          \item $\left( x_1,y_1,z_1 \right)+\left( a,b,c \right)=(-1,-1,-1)\Rightarrow \begin{array}{ccccc} a&=&-x_1&-&2\\b&=&-y_1&-&2\\c&=&z_1&-&2  \end{array}$ \textcolor{green}{\checkmark}

          \item $c\left( x_1,y_1,z_1 \right)=\left( cx+c-1,cy+x-1,cz+c-1 \right)$ is in $\bold{V}$ \textcolor{green}{\checkmark} 

          \item $c(\overrightarrow{\bold{v}}_1+\overrightarrow{\bold{v}}_2)=c\overrightarrow{\bold{v}}_1+c\overrightarrow{\bold{v}}_2$ \textcolor{green}{\checkmark}

              \begin{enumerate}

                \item $c( \left( x_1,y_1,z_1 \right)+\left( x_2,y_2,z_2 \right))=c\left( x_1+x_2+1,y_1+y_2+1,z_1+z_2+1 \right)\Rightarrow \left( c(x_1+x_2+1)+c-1,c(y_1+y_2+1)+c-1,c(z_1+z_2+1)+c-1 \right)$

                \item $c\left( x_1,y_1,z_1 \right)+c\left( x_2,y_2,z_2 \right)=\\\left( cx_1+c-1,cy_1+c-1,cz_1+c-1 \right)+\left( cx_2+c-1,cy_2+c-1,cz_2+c-1 \right)\Rightarrow \left( cx_1+cx_2+2c-2+1,cy_1+cy_2+2c-2+1,cz_1+cz_2+2c-2+1 \right)=\left( cx_1+cx_2+2c-1,cy_1+cy_2+2c-1,cz_1+cz_2+2c-1 \right)$

                \end{enumerate}

              \item $(c+d)\overrightarrow{\bold{v}}_1=c\overrightarrow{\bold{v}}_1+d\overrightarrow{\bold{v}}_1$ \textcolor{green}{\checkmark}

              \begin{enumerate}

                \item $(c+d)(x_1,y_1,z_1)=\\\left( (c+d)x_1+(c+d)-1,(c+d)y_1+(c+d)-1,(c+d)z_1+(c+d)-1 \right)$

                \item $c(x_1,y_1,z_1)+d(x_1,y_1,z_1)=\\(cx_1+c-1,cy_1+c-1,cz_1+c-1) + (dx_1+d-1,dy_1+d-1,dz_1+d-1)\Rightarrow\left( (c+d)x_1+(c+d)-1,(c+d)y_1+(c+d)-1,(c+d)z_1+(c+d)-1 \right)$

                \end{enumerate}

              \item $c(d\overrightarrow{\bold{v}}_1)=(cd)\overrightarrow{\bold{v}}_1$ \textcolor{green}{\checkmark}

              \begin{enumerate}

                \item $(cd)\overrightarrow{\bold{v}}_1=\left( cdx_1+cd-1,cdy_1+cd-1,cdz_1+cd-1 \right)$

                \item $c(d\overrightarrow{\bold{v}}_1)=c\left( dx_1+d-1,dy_1+d-1,dz_1+d-1 \right)=\\\left( c(dx_1+d-1)+c-1,c(dy_1+d-1)+c-1,c(dz_1+d-1)+c-1 \right)\Rightarrow \left( cdx_1+cd-1,cdy_1+cd-1,cdz_1+cd-1 \right)$

                \end{enumerate}

            \item $1(x_1,y_1,z_1)=(1x_1+1-1,1y_1+1-1,1z_1+1-1)=(x_1,y_1,z_1)$ \textcolor{green}{\checkmark}

          \end{enumerate}

    \end{enumerate}

\end{enumerate}

\end{document}

