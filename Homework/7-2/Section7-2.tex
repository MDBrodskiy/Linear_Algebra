%%%%%%%%%%%%%%%%%%%%%%%%%%%%%%%%%%%%%%%%%%%%%%%%%%%%%%%%%%%%%%%%%%%%%%%%%%%%%%%%%%%%%%%%%%%%%%%%%%%%%%%%%%%%%%%%%%%%%%%%%%%%%%%%%%%%%%%%%%%%%%%%%%%%%%%%%%%%%%%%%%%%%%%%%%%%%%%%%%%%%%%%%%%%
% Written By Michael Brodskiy
% Class: Linear Algebra
% Professor: L. Knight
%%%%%%%%%%%%%%%%%%%%%%%%%%%%%%%%%%%%%%%%%%%%%%%%%%%%%%%%%%%%%%%%%%%%%%%%%%%%%%%%%%%%%%%%%%%%%%%%%%%%%%%%%%%%%%%%%%%%%%%%%%%%%%%%%%%%%%%%%%%%%%%%%%%%%%%%%%%%%%%%%%%%%%%%%%%%%%%%%%%%%%%%%%%%

\documentclass[12pt]{article} 
\usepackage{alphalph}
\usepackage[utf8]{inputenc}
\usepackage[russian,english]{babel}
\usepackage{titling}
\usepackage{amsmath}
\usepackage{graphicx}
\usepackage{enumitem}
\usepackage{amssymb}
\usepackage{physics}
\usepackage{tikz}
\usepackage{mathdots}
\usepackage{yhmath}
\usepackage{cancel}
\usepackage{color}
\usepackage{siunitx}
\usepackage{array}
\usepackage{multirow}
\usepackage{gensymb}
\usepackage{tabularx}
\usepackage{booktabs}
\usepackage{pifont}
\newcommand{\xmark}{\ding{55}}
\usetikzlibrary{fadings}
\usetikzlibrary{patterns}
\usetikzlibrary{shadows.blur}
\usetikzlibrary{shapes}
\usepackage[super]{nth}
\usepackage{expl3}
\usepackage[version=4]{mhchem}
\usepackage{hpstatement}
\usepackage{rsphrase}
\usepackage{everysel}
\usepackage{ragged2e}
\usepackage{geometry}
\usepackage{fancyhdr}
\usepackage{cancel}
\usepackage{multicol}
\geometry{top=1.0in,bottom=1.0in,left=1.0in,right=1.0in}
\newcommand{\subtitle}[1]{%
  \posttitle{%
    \par\end{center}
    \begin{center}\large#1\end{center}
    \vskip0.5em}%

}
\usepackage{hyperref}
\hypersetup{
colorlinks=true,
linkcolor=blue,
filecolor=magenta,      
urlcolor=blue,
citecolor=blue,
}

\urlstyle{same}


\title{Linear Algebra 7.2 Homework}
\date{}
\author{Michael Brodskiy\\ \small Instructor: Prof. Knight}

\begin{document}

\maketitle

\begin{enumerate}

    \begin{center}
      \underline{3-23 eoo, 25}
    \end{center}

    \setcounter{enumi}{2}

  \item

    \begin{enumerate}

      \item $P^{-1}=\frac{1}{1-4}\begin{bmatrix} 1 & -2\\ -2 & 1  \end{bmatrix}=\begin{bmatrix} -\frac{1}{3} & \frac{2}{3}\\ \frac{2}{3} & -\frac{1}{3}\end{bmatrix}\Rightarrow\frac{1}{3}\begin{bmatrix} -1 & 2\\ 2 & -1\end{bmatrix}\begin{bmatrix} 3 & -2\\ 2 & -2\end{bmatrix}\begin{bmatrix} 1 & 2\\ 2 & 1\end{bmatrix}=\begin{bmatrix} \frac{1}{3} & -\frac{2}{3}\\ \frac{4}{3} & -\frac{2}{3}  \end{bmatrix}\begin{bmatrix} 1 & 2\\ 2 & 1\end{bmatrix}=\begin{bmatrix} -1 & 0\\ 0 & 2 \end{bmatrix}$

      \item $\lambda_i=-1,2$

    \end{enumerate}

    \setcounter{enumi}{6}

  \item $(\lambda-6)(\lambda-1)-6\Rightarrow \lambda(\lambda-7)=0$, so $\lambda_i=7,0\Rightarrow \left[ \begin{array}{cc|c} -6 & 3 & 0\\ 2 & -1 & 0  \end{array} \right]\Rightarrow \begin{bmatrix} 1\\2\end{bmatrix}\text{ and } \left[ \begin{array}{cc|c} 1 & 3 & 0\\ 2 & 6 & 0\end{array} \right]\Rightarrow\begin{bmatrix} 3\\-1\end{bmatrix}\Rightarrow P = \begin{bmatrix} 1 & 3\\2 & -1  \end{bmatrix}\text{ and } P^{-1}=\frac{1}{7}\begin{bmatrix}1 & 3\\ 2 & -1\end{bmatrix}\Rightarrow\frac{1}{7}\begin{bmatrix} 1 & 3\\ 2 & -1\end{bmatrix} \begin{bmatrix} 6 & -3\\ -2 & 1\end{bmatrix}\begin{bmatrix} 1 & 3\\ 2 & -1\end{bmatrix}=\begin{bmatrix} 0 & 0\\ 2 & -1\end{bmatrix}\begin{bmatrix} 1 & 3\\ 2 & -1\end{bmatrix}=\begin{bmatrix} 0 & 0\\ 0 & 7 \end{bmatrix}$ \textcolor{green}{\checkmark}

    \setcounter{enumi}{10}

  \item $\lambda I -A=\begin{vmatrix} \lambda-1 & -2 & 2\\ 2 & \lambda-5 & 2\\ 6 & -6 & \lambda+3\end{vmatrix}=(\lambda-1)\left[ (\lambda-5)(\lambda+3)+12 \right]+2\left[ 2(\lambda+3)-12 \right]+2\left[ -12-6(\lambda-5) \right]\Rightarrow (\lambda^2-1)(\lambda-3)+4\lambda-12+36-12\lambda\Rightarrow \lambda^3-3\lambda^2-\lambda+3+4\lambda-12+36-12\lambda\Rightarrow \lambda^3 -3\lambda^2 -9\lambda +27=0\Rightarrow (\lambda+3)(\lambda-3)^2=0\Rightarrow \lambda_i=3,-3\Rightarrow \left[ \begin{array}{ccc|c} 2 & -2 & 2 & 0\\ 2 & -2 & 2 & 0\\ 6 & -6 & 6 & 0  \end{array} \right]\Rightarrow \begin{bmatrix} 1\\1\\0  \end{bmatrix},\begin{bmatrix} -1\\0\\1\end{bmatrix}\text{ and }\left[ \begin{array}{ccc|c} -4 & -2 & 2 & 0\\ 2 & -8 & 2 & 0\\ 6 & -6 & 0 & 0  \end{array} \right]\Rightarrow \begin{bmatrix} 1\\1\\3\end{bmatrix}\Rightarrow P=\begin{bmatrix} 1 & -1 & 1\\1 & 0 & 1\\ 0 & 1 & 3\end{bmatrix}\Rightarrow P^{-1}=\frac{1}{3}\begin{bmatrix} -1 & 4 & -1\\ -3 & 3 & 0\\1 & -1 & 1\end{bmatrix}\Rightarrow \frac{1}{3}\begin{bmatrix} -1 & 4 & -1\\ -3 & 3 & 0\\ 1 & -1 & 1\end{bmatrix}\begin{bmatrix} 1 & 2 & -2\\ -2 & 5 & -2\\ -6 & 6 & -3\end{bmatrix}\begin{bmatrix} 1 & -1 & 1\\ 1 & 0 & 1\\ 0 & 1 & 3\end{bmatrix}=\begin{bmatrix} -1 & 4 & -1\\ -3 & 3 & 0\\ -1 & 1 & -1\end{bmatrix}\begin{bmatrix} 1 & -1 & 1\\ 1 & 0 & 1\\ 0 & 1 & 3  \end{bmatrix}=\begin{bmatrix} 3 & 0 & 0\\ 0 & 3 & 0\\ 0 & 0 & -3\end{bmatrix}$ \textcolor{green}{\checkmark}

    \setcounter{enumi}{14}

  \item $(\lambda)(\lambda)=0\Rightarrow\lambda_i=0$. There are not $n$ distinct eigenvalues, so it is not diagonalizable.

    \setcounter{enumi}{18}

  \item $(\lambda-1)\left[ (\lambda-1)(\lambda-2) \right]\Rightarrow(\lambda-1)^2(\lambda-2)=0\Rightarrow \lambda_i=1,2$. There are not $n$ distinct eigenvalues, so it is not diagonalizable.

    \setcounter{enumi}{22}

  \item $(\lambda-1)^2-1\Rightarrow \lambda^2-2\lambda=0\Rightarrow \lambda_i=0,2$, so there are enough distinct eigenvalues.

    \setcounter{enumi}{24}

  \item $(\lambda+3)\left[ (\lambda-4)(\lambda+5)+18 \right]-2\left[ -3(\lambda+5)+9 \right]-3\left[ 6+(\lambda-4) \right]\Rightarrow (\lambda+3)(\lambda^2+\lambda-2)+6\lambda+12-6-3\lambda\Rightarrow \lambda^3 +\lambda^2-2\lambda+3\lambda^2+3\lambda-6+6\lambda+12-6-3\lambda\Rightarrow \lambda^3+4\lambda^2+4\lambda=0\Rightarrow \lambda_i=0,-2$, so there are not enough distinct eigenvalues.

\end{enumerate}

\end{document}

