%%%%%%%%%%%%%%%%%%%%%%%%%%%%%%%%%%%%%%%%%%%%%%%%%%%%%%%%%%%%%%%%%%%%%%%%%%%%%%%%%%%%%%%%%%%%%%%%%%%%%%%%%%%%%%%%%%%%%%%%%%%%%%%%%%%%%%%%%%%%%%%%%%%%%%%%%%%%%%%%%%%%%%%%%%%%%%%%%%%%%%%%%%%%
% Written By Michael Brodskiy
% Class: Linear Algebra
% Professor: L. Knight
%%%%%%%%%%%%%%%%%%%%%%%%%%%%%%%%%%%%%%%%%%%%%%%%%%%%%%%%%%%%%%%%%%%%%%%%%%%%%%%%%%%%%%%%%%%%%%%%%%%%%%%%%%%%%%%%%%%%%%%%%%%%%%%%%%%%%%%%%%%%%%%%%%%%%%%%%%%%%%%%%%%%%%%%%%%%%%%%%%%%%%%%%%%%

\documentclass[12pt]{article} 
\usepackage{alphalph}
\usepackage[utf8]{inputenc}
\usepackage[russian,english]{babel}
\usepackage{titling}
\usepackage{amsmath}
\usepackage{graphicx}
\usepackage{enumitem}
\usepackage{amssymb}
\usepackage{physics}
\usepackage{tikz}
\usepackage{mathdots}
\usepackage{yhmath}
\usepackage{cancel}
\usepackage{color}
\usepackage{siunitx}
\usepackage{array}
\usepackage{multirow}
\usepackage{gensymb}
\usepackage{tabularx}
\usepackage{booktabs}
\usetikzlibrary{fadings}
\usetikzlibrary{patterns}
\usetikzlibrary{shadows.blur}
\usetikzlibrary{shapes}
\usepackage[super]{nth}
\usepackage{expl3}
\usepackage[version=4]{mhchem}
\usepackage{hpstatement}
\usepackage{rsphrase}
\usepackage{everysel}
\usepackage{ragged2e}
\usepackage{geometry}
\usepackage{fancyhdr}
\usepackage{cancel}
\usepackage{multicol}
\geometry{top=1.0in,bottom=1.0in,left=1.0in,right=1.0in}
\newcommand{\subtitle}[1]{%
  \posttitle{%
    \par\end{center}
    \begin{center}\large#1\end{center}
    \vskip0.5em}%

}
\usepackage{hyperref}
\hypersetup{
colorlinks=true,
linkcolor=blue,
filecolor=magenta,      
urlcolor=blue,
citecolor=blue,
}

\urlstyle{same}


\title{Linear Algebra 4.1 Homework}
\date{}
\author{Michael Brodskiy\\ \small Instructor: Prof. Knight}

\begin{document}

\maketitle

\begin{enumerate}

    \setcounter{enumi}{6}

  \item $\langle 1,3\rangle + \langle 2, -2 \rangle=\langle 3,1\rangle$

    \begin{figure}[h]
      \centering
      \tikzset{every picture/.style={line width=0.75pt}} %set default line width to 0.75pt        

\begin{tikzpicture}[x=0.75pt,y=0.75pt,yscale=-1,xscale=1]
%uncomment if require: \path (0,218); %set diagram left start at 0, and has height of 218


%Shape: Boxed Line [id:dp4197928522176406] 
\draw    (296,5.29) -- (296,146.71) ;
%Shape: Boxed Line [id:dp1386254206809665] 
\draw    (437.42,146.71) -- (296,146.71) ;

%Straight Lines [id:da9069771914880562] 
\draw    (296,146.71) -- (325.16,46.59) ;
\draw [shift={(326,43.71)}, rotate = 466.24] [fill={rgb, 255:red, 0; green, 0; blue, 0 }  ][line width=0.08]  [draw opacity=0] (8.93,-4.29) -- (0,0) -- (8.93,4.29) -- cycle    ;
%Shape: Boxed Line [id:dp25011998260574986] 
\draw    (296,146.71) -- (297,219.71) ;
%Straight Lines [id:da6475776498837635] 
\draw    (326,43.71) -- (391.37,98.41) ;
\draw [shift={(393.67,100.34)}, rotate = 219.92000000000002] [fill={rgb, 255:red, 0; green, 0; blue, 0 }  ][line width=0.08]  [draw opacity=0] (8.93,-4.29) -- (0,0) -- (8.93,4.29) -- cycle    ;
%Straight Lines [id:da29394742031612564] 
\draw    (296,146.71) -- (390.96,101.62) ;
\draw [shift={(393.67,100.34)}, rotate = 514.6] [fill={rgb, 255:red, 0; green, 0; blue, 0 }  ][line width=0.08]  [draw opacity=0] (8.93,-4.29) -- (0,0) -- (8.93,4.29) -- cycle    ;


% Text Node
\draw (327.38,147) node [anchor=north] [inner sep=0.75pt]   [align=left] {1};
% Text Node
\draw (363.38,147) node [anchor=north] [inner sep=0.75pt]   [align=left] {2};
% Text Node
\draw (400.38,147) node [anchor=north] [inner sep=0.75pt]   [align=left] {3};
% Text Node
\draw (276.87,43.5) node [anchor=north] [inner sep=0.75pt]  [rotate=-270] [align=left] {3};
% Text Node
\draw (276.88,80.5) node [anchor=north] [inner sep=0.75pt]  [rotate=-270] [align=left] {2};
% Text Node
\draw (276.88,116.5) node [anchor=north] [inner sep=0.75pt]  [rotate=-270] [align=left] {1};
% Text Node
\draw (309,92.21) node [anchor=south east] [inner sep=0.75pt]   [align=left] {\textbf{\textit{u}}};
% Text Node
\draw (276.87,171.5) node [anchor=north] [inner sep=0.75pt]  [rotate=-270] [align=left] {\mbox{-}1};
% Text Node
\draw (277.88,206.5) node [anchor=north] [inner sep=0.75pt]  [rotate=-270] [align=left] {\mbox{-}2};
% Text Node
\draw (361.83,69.03) node [anchor=south west] [inner sep=0.75pt]   [align=left] {\textbf{\textit{v}}};
% Text Node
\draw (347.74,125.66) node [anchor=north west][inner sep=0.75pt]  [rotate=-339.94] [align=left] {\textbf{\textit{u }}+ \textbf{\textit{v}}};


\end{tikzpicture}

      \caption{Problem 7 Figure}
      \label{fig:1}
    \end{figure}

    \setcounter{enumi}{10}

  \item $\overrightarrow{\bold{v}}=\frac{3}{2}\overrightarrow{\bold{u}}\Rightarrow \frac{3}{2}\langle -2, 3\rangle=\langle -3, \frac{9}{2}\rangle$

    \begin{figure}[h]
      \centering
      \tikzset{every picture/.style={line width=0.75pt}} %set default line width to 0.75pt        

\begin{tikzpicture}[x=0.75pt,y=0.75pt,yscale=-1,xscale=1]
%uncomment if require: \path (0,218); %set diagram left start at 0, and has height of 218

%Shape: Boxed Line [id:dp4197928522176406] 
\draw    (296,5.29) -- (296,146.71) ;
%Shape: Boxed Line [id:dp2691398772595486] 
\draw    (296,146.71) -- (154.58,146.71) ;
%Straight Lines [id:da7555271225783102] 
\draw    (235.55,46.57) -- (296,146.71) ;
\draw [shift={(234,44)}, rotate = 58.88] [fill={rgb, 255:red, 0; green, 0; blue, 0 }  ][line width=0.08]  [draw opacity=0] (8.93,-4.29) -- (0,0) -- (8.93,4.29) -- cycle    ;
%Straight Lines [id:da3247719140034826] 
\draw    (296,146.71) -- (214.56,12.56) ;
\draw [shift={(213,10)}, rotate = 418.74] [fill={rgb, 255:red, 0; green, 0; blue, 0 }  ][line width=0.08]  [draw opacity=0] (8.93,-4.29) -- (0,0) -- (8.93,4.29) -- cycle    ;

% Text Node
\draw (298.87,52.5) node [anchor=north] [inner sep=0.75pt]  [rotate=-270] [align=left] {3};
% Text Node
\draw (299.88,86.5) node [anchor=north] [inner sep=0.75pt]  [rotate=-270] [align=left] {2};
% Text Node
\draw (299.88,116.5) node [anchor=north] [inner sep=0.75pt]  [rotate=-270] [align=left] {1};
% Text Node
\draw (277.38,146) node [anchor=north] [inner sep=0.75pt]   [align=left] {\mbox{-}1};
% Text Node
\draw (239.38,146) node [anchor=north] [inner sep=0.75pt]   [align=left] {\mbox{-}2};
% Text Node
\draw (207.38,146) node [anchor=north] [inner sep=0.75pt]   [align=left] {\mbox{-}3};
% Text Node
\draw (263,98.36) node [anchor=north east] [inner sep=0.75pt]   [align=left] {\textbf{\textit{u}}};
% Text Node
\draw (298,24.29) node [anchor=north] [inner sep=0.75pt]  [rotate=-270] [align=left] {4};
% Text Node
\draw (211,13) node [anchor=north east] [inner sep=0.75pt]   [align=left] {\textbf{\textit{v}}};


\end{tikzpicture}

      \caption{Problem 11 Figure}
      \label{fig:2}
    \end{figure}

    \setcounter{enumi}{18}

  \item

    \begin{enumerate}

      \item $\overrightarrow{\bold{u}}-\overrightarrow{\bold{v}}=\langle1,2,3\rangle - \langle 2,2,-1\rangle=\langle-1,0,4\rangle$

      \item $\overrightarrow{\bold{v}}-\overrightarrow{\bold{u}}=\langle 2,2,-1\rangle-\langle1,2,3\rangle=\langle1,0,-4\rangle$


    \end{enumerate}

    \setcounter{enumi}{20}

  \item $2\overrightarrow{\bold{u}}+4\overrightarrow{\bold{v}}-\overrightarrow{\bold{w}}=2\langle1,2,3\rangle+4\langle 2,2,-1\rangle-\langle4,0,-4\rangle=\langle6,12,6\rangle$

    \setcounter{enumi}{23}

  \item $\overrightarrow{\bold{z}}=-\frac{2\overrightarrow{\bold{u}}+\overrightarrow{\bold{v}}-\overrightarrow{\bold{w}}}{3}\Rightarrow -\frac{1}{3}\left( 2\langle1,2,3\rangle+\langle2,2,-1\rangle-\langle4,0,-4\rangle  \right)=\langle0,-2,-3\rangle$

    \setcounter{enumi}{28}

  \item

    \begin{enumerate}

      \item $\overrightarrow{\bold{u}}-\overrightarrow{\bold{v}}=\langle 4,0,-3,5\rangle-\langle0,2,5,4\rangle=\langle4,-2,-8,1\rangle$

      \item $2\overrightarrow{\bold{u}}+6\overrightarrow{\bold{v}}=2\langle 4,0,-3,5\rangle+6\langle0,2,5,4\rangle=\langle8,12,24,34\rangle$

      \item $2\overrightarrow{\bold{v}}-\overrightarrow{\bold{u}}=2\langle0,2,5,4\rangle-\langle4,0,-3,5\rangle=\langle-4,4,13,3\rangle$

    \end{enumerate}

    \setcounter{enumi}{34}

  \item $\overrightarrow{\bold{w}}=\frac{1}{3}(\overrightarrow{\bold{u}}-2\overrightarrow{\bold{v}})=\frac{1}{3}\left( \langle1,-1,0,1\rangle-2\langle0,2,3,-1 \right)=\langle\frac{1}{3},-\frac{5}{3},-2,1\rangle$

    \setcounter{enumi}{40}

  \item $\overrightarrow{\bold{v}}=\overrightarrow{\bold{u}}+\overrightarrow{\bold{w}}$

    \setcounter{enumi}{48}

  \item $c_1\langle1,1,2,2\rangle+c_2\langle2,3,5,6\rangle+c_3\langle-3,1,-4,2\rangle=\langle0,5,3,0\rangle$. Such a combination is not possible.

    \setcounter{enumi}{54}

  \item Such a combination is not possible (although number 56 is)

\end{enumerate}

\end{document}

