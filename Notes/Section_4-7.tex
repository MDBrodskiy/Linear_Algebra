%%%%%%%%%%%%%%%%%%%%%%%%%%%%%%%%%%%%%%%%%%%%%%%%%%%%%%%%%%%%%%%%%%%%%%%%%%%%%%%%%%%%%%%%%%%%%%%%%%%%%%%%%%%%%%%%%%%%%%%%%%%%%%%%%%%%%%%%%%%%%%%%%%%%%%%%%%%%%%%%%%%%%%%%%%%%%%%%%%%%%%%%%%%%
% Written By Michael Brodskiy
% Class: Linear Algebra (MATH-194)
% Professor: L. Knight
%%%%%%%%%%%%%%%%%%%%%%%%%%%%%%%%%%%%%%%%%%%%%%%%%%%%%%%%%%%%%%%%%%%%%%%%%%%%%%%%%%%%%%%%%%%%%%%%%%%%%%%%%%%%%%%%%%%%%%%%%%%%%%%%%%%%%%%%%%%%%%%%%%%%%%%%%%%%%%%%%%%%%%%%%%%%%%%%%%%%%%%%%%%%

\documentclass[12pt]{article} 
\usepackage{alphalph}
\usepackage[utf8]{inputenc}
\usepackage[russian,english]{babel}
\usepackage{titling}
\usepackage{amsmath}
\usepackage{graphicx}
\usepackage{enumitem}
\usepackage{amssymb}
\usepackage[super]{nth}
\usepackage{everysel}
\usepackage{ragged2e}
\usepackage{geometry}
\usepackage{fancyhdr}
\usepackage{cancel}
\usepackage{siunitx}
\geometry{top=1.0in,bottom=1.0in,left=1.0in,right=1.0in}
\newcommand{\subtitle}[1]{%
  \posttitle{%
    \par\end{center}
    \begin{center}\large#1\end{center}
    \vskip0.5em}%

}
\usepackage{hyperref}
\hypersetup{
colorlinks=true,
linkcolor=blue,
filecolor=magenta,      
urlcolor=blue,
citecolor=blue,
}

\urlstyle{same}


\title{Coordinates and Change of Basis}
\date{\today}
\author{Michael Brodskiy\\ \small Professor: Lynn Knight}

% Mathematical Operations:

% Sum: $$\sum_{n=a}^{b} f(x) $$
% Integral: $$\int_{lower}^{upper} f(x) dx$$
% Limit: $$\lim_{x\to\infty} f(x)$$

\begin{document}

\maketitle

\begin{itemize}

  \item Suppose $\overrightarrow{u}=(3,4)$ and $B=\left\{ (1,0), (0,1) \right\}$. Then you could say $\overrightarrow{u}=3(1,0)+4(0,1)$. This could be written as $[\overrightarrow{u}]_B=\begin{bmatrix} 3\\ 4 \end{bmatrix}$

  \item Let the set of vectors $\left\{ \overrightarrow{v}_1,\overrightarrow{v}_2,\dots,\overrightarrow{v}_n \right\}$ be the basis for vector space $\bold{V}$, and $c_1,c_2,\dots,c_n$ be scalars, where $\overrightarrow{u} \in \bold{V}$ such that $\overrightarrow{u}=c_1\overrightarrow{v}_1+c_2\overrightarrow{v}_2+\dots+c_v\overrightarrow{v}_n$. This can be written as a coordinate matrix of vector $\overrightarrow{u}$ with respect to basis $B$:
    \begin{center}
      $[\overrightarrow{u}]_B=\begin{bmatrix} c_1\\c_2\\\vdots\\c_n\end{bmatrix}$
    \end{center}

  \item Change of Basis — Suppose you have basis $B=\left\{ \overrightarrow{b}_1, \overrightarrow{b}_2 \right\}$ and $C=\left\{ \overrightarrow{c}_1, \overrightarrow{c}_2 \right\}$ of vector space $V$. Then you could write $\begin{array}{c} \overrightarrow{b}_1=a\overrightarrow{c}_1+b\overrightarrow{c}_2\\ \overrightarrow{b}_2=c\overrightarrow{c}_1+d\overrightarrow{c}_2 \end{array}$, and, given $[\overrightarrow{v}]_B=\begin{bmatrix}k_1\\k_2\end{bmatrix}$, you could find $[\overrightarrow{v}]_C$ through the following method:

    \begin{equation*}
      \begin{split}
        \overrightarrow{v}=k_1\overrightarrow{b}_1+k_2\overrightarrow{b}_2\\
        \overrightarrow{v}=k_1(a\overrightarrow{c}_1+b\overrightarrow{c}_2)+k_2(c\overrightarrow{c}_1+d\overrightarrow{c}_2)\\
        [\overrightarrow{v}]_c=\begin{bmatrix} ak_1+ck_2\\ bk_1+dk_2\end{bmatrix}=\begin{bmatrix} a & b\\c & d\end{bmatrix}\begin{bmatrix} k_1\\k_2\end{bmatrix}\\
        \text{Where }P[\overrightarrow{v}]_B=\begin{bmatrix}a & b\\ c & d\end{bmatrix}\text{ is a transition matrix}
      \end{split}
      \label{1}
    \end{equation}

  \item For $B\rightarrow C$: $[\overrightarrow{v}]_C=P[\overrightarrow{v}]_B$

  \item For $C\rightarrow B$: $[\overrightarrow{v}]_B=P[\overrightarrow{v}]_C$

\end{itemize}

\end{itemize}

\end{document}

