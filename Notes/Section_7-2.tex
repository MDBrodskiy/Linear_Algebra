%%%%%%%%%%%%%%%%%%%%%%%%%%%%%%%%%%%%%%%%%%%%%%%%%%%%%%%%%%%%%%%%%%%%%%%%%%%%%%%%%%%%%%%%%%%%%%%%%%%%%%%%%%%%%%%%%%%%%%%%%%%%%%%%%%%%%%%%%%%%%%%%%%%%%%%%%%%%%%%%%%%%%%%%%%%%%%%%%%%%%%%%%%%%
% Written By Michael Brodskiy
% Class: Linear Algebra (MATH-194)
% Professor: L. Knight
%%%%%%%%%%%%%%%%%%%%%%%%%%%%%%%%%%%%%%%%%%%%%%%%%%%%%%%%%%%%%%%%%%%%%%%%%%%%%%%%%%%%%%%%%%%%%%%%%%%%%%%%%%%%%%%%%%%%%%%%%%%%%%%%%%%%%%%%%%%%%%%%%%%%%%%%%%%%%%%%%%%%%%%%%%%%%%%%%%%%%%%%%%%%

\documentclass[12pt]{article} 
\usepackage{alphalph}
\usepackage[utf8]{inputenc}
\usepackage[russian,english]{babel}
\usepackage{titling}
\usepackage{amsmath}
\usepackage{graphicx}
\usepackage{enumitem}
\usepackage{amssymb}
\usepackage[super]{nth}
\usepackage{everysel}
\usepackage{ragged2e}
\usepackage{geometry}
\usepackage{fancyhdr}
\usepackage{cancel}
\usepackage{siunitx}
\geometry{top=1.0in,bottom=1.0in,left=1.0in,right=1.0in}
\newcommand{\subtitle}[1]{%
  \posttitle{%
    \par\end{center}
    \begin{center}\large#1\end{center}
    \vskip0.5em}%

}
\usepackage{hyperref}
\hypersetup{
colorlinks=true,
linkcolor=blue,
filecolor=magenta,      
urlcolor=blue,
citecolor=blue,
}

\urlstyle{same}


\title{Diagonalization}
\date{\today}
\author{Michael Brodskiy\\ \small Professor: Lynn Knight}

% Mathematical Operations:

% Sum: $$\sum_{n=a}^{b} f(x) $$
% Integral: $$\int_{lower}^{upper} f(x) dx$$
% Limit: $$\lim_{x\to\infty} f(x)$$

\begin{document}

\maketitle

\begin{itemize}

  \item An $n\times n$ matrix $\bold{A}$ is diagonalizable if there exists an invertible matrix $\bold{P}$, such that $\bold{D}=\bold{P}\bold{A}\bold{P}^{-1}$ is a diagonal matrix. The matrix $\bold{P}$ is said to diagonalize $\bold{A}$.

  \item An $n\times n$ matrix is diagonalizable if and only if $\bold{A}$ has $n$ linearly independent eigenvectors.

  \item To Diagonalize a Matrix:

    \begin{enumerate}

      \item Find $n$ linearly independent eigenvectors $\overrightarrow{p}_1,\overrightarrow{p}_2,\dots,\overrightarrow{p}_n$

      \item Form $\bold{P}=\begin{bmatrix} \overrightarrow{p}_1 & \overrightarrow{p}_2 & \dots & \overrightarrow{p}_n\end{bmatrix}$

      \item Form $\bold{D}=\begin{bmatrix} \lambda_1 & 0 & 0 & 0\\ 0 & \lambda_2 & 0 & 0\\ 0 & 0 & \ddots & 0\\ 0 & 0 & 0 & \lambda_n \end{bmatrix}$

    \end{enumerate}

  \item If an $n\times n$ matrix has $n$ distinct eigenvalues, then corresponding eigenvectors are linearly independent, and $\bold{A}$ is diagonalizable

\end{itemize}

\end{document}

