%%%%%%%%%%%%%%%%%%%%%%%%%%%%%%%%%%%%%%%%%%%%%%%%%%%%%%%%%%%%%%%%%%%%%%%%%%%%%%%%%%%%%%%%%%%%%%%%%%%%%%%%%%%%%%%%%%%%%%%%%%%%%%%%%%%%%%%%%%%%%%%%%%%%%%%%%%%%%%%%%%%%%%%%%%%%%%%%%%%%%%%%%%%%
% Written By Michael Brodskiy
% Class: Linear Algebra (MATH-194)
% Professor: L. Knight
%%%%%%%%%%%%%%%%%%%%%%%%%%%%%%%%%%%%%%%%%%%%%%%%%%%%%%%%%%%%%%%%%%%%%%%%%%%%%%%%%%%%%%%%%%%%%%%%%%%%%%%%%%%%%%%%%%%%%%%%%%%%%%%%%%%%%%%%%%%%%%%%%%%%%%%%%%%%%%%%%%%%%%%%%%%%%%%%%%%%%%%%%%%%

\documentclass[12pt]{article} 
\usepackage{alphalph}
\usepackage[utf8]{inputenc}
\usepackage[russian,english]{babel}
\usepackage{titling}
\usepackage{amsmath}
\usepackage{graphicx}
\usepackage{enumitem}
\usepackage{amssymb}
\usepackage[super]{nth}
\usepackage{everysel}
\usepackage{ragged2e}
\usepackage{geometry}
\usepackage{fancyhdr}
\usepackage{cancel}
\usepackage{siunitx}
\geometry{top=1.0in,bottom=1.0in,left=1.0in,right=1.0in}
\newcommand{\subtitle}[1]{%
  \posttitle{%
    \par\end{center}
    \begin{center}\large#1\end{center}
    \vskip0.5em}%

}
\usepackage{hyperref}
\hypersetup{
colorlinks=true,
linkcolor=blue,
filecolor=magenta,      
urlcolor=blue,
citecolor=blue,
}

\urlstyle{same}


\title{Inverse of a Square Matrix}
\date{\today}
\author{Michael Brodskiy\\ \small Professor: Lynn Knight}

% Mathematical Operations:

% Sum: $$\sum_{n=a}^{b} f(x) $$
% Integral: $$\int_{lower}^{upper} f(x) dx$$
% Limit: $$\lim_{x\to\infty} f(x)$$

\begin{document}

\maketitle

\begin{itemize}

  \item $\bold{A}\cdot\bold{A}^{-1}=\bold{I}$

  \item If a two by two matrix is given, and the entries are $\bold{A}=\left[ \begin{array}{l l} a & b\\ c & d\\ \end{array} \right]$, then the inverse is $\bold{A}^{-1}=\frac{1}{ad-bc}\left[ \begin{array}{c c} d & -b\\ -c & a\\ \end{array} \right]$

  \item Some matrices may be non-invertible (proven if a row of zeros is obtained)

  \item $(\bold{AB})^{-1}=\bold{B}^{-1}\bold{A}^{-1}$

  \item $(\bold{A}^{-1})^{-1}=\bold{A}$

  \item $(\bold{A}^{k})^{-1}=\bold{A}^{-1}\cdot\bold{A}^{-1}\cdot\bold{A}^{-1}\dots=(\bold{A}^{-1})^k$

  \item $(c\bold{A})^{-1}=\frac{1}{c}\bold{A}^{-1}$

  \item $(\bold{A}^T)^{-1}=(\bold{A}^{-1})^T$

\end{itemize}

\end{document}

