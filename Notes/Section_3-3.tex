%%%%%%%%%%%%%%%%%%%%%%%%%%%%%%%%%%%%%%%%%%%%%%%%%%%%%%%%%%%%%%%%%%%%%%%%%%%%%%%%%%%%%%%%%%%%%%%%%%%%%%%%%%%%%%%%%%%%%%%%%%%%%%%%%%%%%%%%%%%%%%%%%%%%%%%%%%%%%%%%%%%%%%%%%%%%%%%%%%%%%%%%%%%%
% Written By Michael Brodskiy
% Class: Linear Algebra (MATH-194)
% Professor: L. Knight
%%%%%%%%%%%%%%%%%%%%%%%%%%%%%%%%%%%%%%%%%%%%%%%%%%%%%%%%%%%%%%%%%%%%%%%%%%%%%%%%%%%%%%%%%%%%%%%%%%%%%%%%%%%%%%%%%%%%%%%%%%%%%%%%%%%%%%%%%%%%%%%%%%%%%%%%%%%%%%%%%%%%%%%%%%%%%%%%%%%%%%%%%%%%

\documentclass[12pt]{article} 
\usepackage{alphalph}
\usepackage[utf8]{inputenc}
\usepackage[russian,english]{babel}
\usepackage{titling}
\usepackage{amsmath}
\usepackage{graphicx}
\usepackage{enumitem}
\usepackage{amssymb}
\usepackage[super]{nth}
\usepackage{everysel}
\usepackage{ragged2e}
\usepackage{geometry}
\usepackage{fancyhdr}
\usepackage{cancel}
\usepackage{siunitx}
\geometry{top=1.0in,bottom=1.0in,left=1.0in,right=1.0in}
\newcommand{\subtitle}[1]{%
  \posttitle{%
    \par\end{center}
    \begin{center}\large#1\end{center}
    \vskip0.5em}%

}
\usepackage{hyperref}
\hypersetup{
colorlinks=true,
linkcolor=blue,
filecolor=magenta,      
urlcolor=blue,
citecolor=blue,
}

\urlstyle{same}


\title{Properties of Determinants}
\date{\today}
\author{Michael Brodskiy\\ \small Professor: Lynn Knight}

% Mathematical Operations:

% Sum: $$\sum_{n=a}^{b} f(x) $$
% Integral: $$\int_{lower}^{upper} f(x) dx$$
% Limit: $$\lim_{x\to\infty} f(x)$$

\begin{document}

\maketitle

\begin{itemize}

  \item If $\bold{A}$ and $\bold{B}$ are square matrices of order $n$, then $\det\left( \bold{AB} \right)=\det(\bold{A})\det(\bold{B})$

  \item All row operations are extended to columns

  \item Let $\ord(\bold{A})=n$ and $c$ be a scalar, then $\det(c\bold{A})=c^n\det(\bold{A})$

  \item $\bold{A}$ is invertible iff the determinant of $\bold{A}\neq0$

  \item Fundamental Theorem (version II)

    \begin{enumerate}

      \item If $\bold{A}$ is an $n$x$n$ matrix, then the following conditions are equivalent:
        \begin{enumerate}

          \item $\bold{A}$ is invertible

          \item $\bold{A}x=\bold{B}$ has a unique solution for any $n$x1 column matrix $\bold{B}$

          \item $\bold{A}x=0$ has only trivial solution $x=0$

          \item $\bold{A}$ \widetilde{R} $I_n$

          \item $\bold{A}$ can be written as a product of elementary matrices

          \item $\det(\bold{A})\neq0$

        \end{enumerate}

    \end{enumerate}

\end{itemize}

\end{document}

