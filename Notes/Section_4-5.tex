%%%%%%%%%%%%%%%%%%%%%%%%%%%%%%%%%%%%%%%%%%%%%%%%%%%%%%%%%%%%%%%%%%%%%%%%%%%%%%%%%%%%%%%%%%%%%%%%%%%%%%%%%%%%%%%%%%%%%%%%%%%%%%%%%%%%%%%%%%%%%%%%%%%%%%%%%%%%%%%%%%%%%%%%%%%%%%%%%%%%%%%%%%%%
% Written By Michael Brodskiy
% Class: Linear Algebra (MATH-194)
% Professor: L. Knight
%%%%%%%%%%%%%%%%%%%%%%%%%%%%%%%%%%%%%%%%%%%%%%%%%%%%%%%%%%%%%%%%%%%%%%%%%%%%%%%%%%%%%%%%%%%%%%%%%%%%%%%%%%%%%%%%%%%%%%%%%%%%%%%%%%%%%%%%%%%%%%%%%%%%%%%%%%%%%%%%%%%%%%%%%%%%%%%%%%%%%%%%%%%%

\documentclass[12pt]{article} 
\usepackage{alphalph}
\usepackage[utf8]{inputenc}
\usepackage[russian,english]{babel}
\usepackage{titling}
\usepackage{amsmath}
\usepackage{graphicx}
\usepackage{enumitem}
\usepackage{amssymb}
\usepackage[super]{nth}
\usepackage{everysel}
\usepackage{ragged2e}
\usepackage{geometry}
\usepackage{fancyhdr}
\usepackage{cancel}
\usepackage{siunitx}
\geometry{top=1.0in,bottom=1.0in,left=1.0in,right=1.0in}
\newcommand{\subtitle}[1]{%
  \posttitle{%
    \par\end{center}
    \begin{center}\large#1\end{center}
    \vskip0.5em}%

}
\usepackage{hyperref}
\hypersetup{
colorlinks=true,
linkcolor=blue,
filecolor=magenta,      
urlcolor=blue,
citecolor=blue,
}

\urlstyle{same}


\title{Basis and Dimension}
\date{\today}
\author{Michael Brodskiy\\ \small Professor: Lynn Knight}

% Mathematical Operations:

% Sum: $$\sum_{n=a}^{b} f(x) $$
% Integral: $$\int_{lower}^{upper} f(x) dx$$
% Limit: $$\lim_{x\to\infty} f(x)$$

\begin{document}

\maketitle

\begin{itemize}

  \item $B=\left\{ \overrightarrow{\bold{v}}_1,\overrightarrow{\bold{v}}_2,\dots,\overrightarrow{\bold{v}}_n \right\}$ from vector space $\bold{V}$ forms basis for $\bold{V}$ if

    \begin{enumerate}

      \item $B$ is linearly independent

      \item span$(B)=\bold{V}$

      \item $\overrightarrow{\bold{v}}_i$ called basis vectors

      \item Examples:

        \begin{enumerate}

          \item $\hat{i}$ and $\hat{j}$ are basis vectors in $\mathbb{R}^2$

          \item $\hat{i}$, $\hat{j}$, and $\hat{k}$ are basis vectors in $\mathbb{R}^3$

          \item $\begin{bmatrix}1 & 0\\ 0 & 0\\\end{bmatrix}$, $\begin{bmatrix} 0 & 1\\ 0 & 0 \end{bmatrix}$, $\begin{bmatrix} 0 & 0\\ 1 & 0 \end{bmatrix}$, and $\begin{bmatrix} 0 & 0\\ 0 & 1 \end{bmatrix}$ are basis vectors in $M_{2,2}$

        \end{enumerate}

    \end{enumerate}

  \item Show $\left\{ 1,x,x^2 \right\}$ basis for $P_2$

    \begin{enumerate}

      \item Trivial Solution only (so it is linearly independent)

      \item Spans $P_2=c_1+c_2x+c_3x^2$

    \end{enumerate}

  \item If $B$ is basis there is only one set of scalars $c_1,c_2,\dots,c_n$ such that $\overrightarrow{\bold{w}}=c_1\overrightarrow{\bold{v}}_1+c_2\overrightarrow{\bold{v}}_2+\dots+c_n\overrightarrow{\bold{v}}_n$

  \item Dimension is the number of basis vectors:

    \begin{enumerate}

      \item $\mathbb{R}^n$ $-$ $n$

      \item $P_n$ $-$ $n+1$

      \item $M_{m,n}$ $-$ m\cdot n

    \end{enumerate}

  \item If $S=\left\{ \overrightarrow{\bold{v}}_1,\overrightarrow{\bold{v}}_2,\dots,\overrightarrow{\bold{v}}_n \right\}$ is basis for vector space $\bold{V}$, then every set containing more than $n$ vectors will be linearly dependent

\end{itemize}

\end{document}

