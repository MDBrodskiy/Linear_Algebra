%%%%%%%%%%%%%%%%%%%%%%%%%%%%%%%%%%%%%%%%%%%%%%%%%%%%%%%%%%%%%%%%%%%%%%%%%%%%%%%%%%%%%%%%%%%%%%%%%%%%%%%%%%%%%%%%%%%%%%%%%%%%%%%%%%%%%%%%%%%%%%%%%%%%%%%%%%%%%%%%%%%%%%%%%%%%%%%%%%%%%%%%%%%%
% Written By Michael Brodskiy
% Class: Linear Algebra (MATH-194)
% Professor: L. Knight
%%%%%%%%%%%%%%%%%%%%%%%%%%%%%%%%%%%%%%%%%%%%%%%%%%%%%%%%%%%%%%%%%%%%%%%%%%%%%%%%%%%%%%%%%%%%%%%%%%%%%%%%%%%%%%%%%%%%%%%%%%%%%%%%%%%%%%%%%%%%%%%%%%%%%%%%%%%%%%%%%%%%%%%%%%%%%%%%%%%%%%%%%%%%

\documentclass[12pt]{article} 
\usepackage{alphalph}
\usepackage[utf8]{inputenc}
\usepackage[russian,english]{babel}
\usepackage{titling}
\usepackage{amsmath}
\usepackage{graphicx}
\usepackage{enumitem}
\usepackage{amssymb}
\usepackage[super]{nth}
\usepackage{everysel}
\usepackage{ragged2e}
\usepackage{geometry}
\usepackage{fancyhdr}
\usepackage{cancel}
\usepackage{siunitx}
\geometry{top=1.0in,bottom=1.0in,left=1.0in,right=1.0in}
\newcommand{\subtitle}[1]{%
  \posttitle{%
    \par\end{center}
    \begin{center}\large#1\end{center}
    \vskip0.5em}%

}
\usepackage{hyperref}
\hypersetup{
colorlinks=true,
linkcolor=blue,
filecolor=magenta,      
urlcolor=blue,
citecolor=blue,
}

\urlstyle{same}


\title{Properties of Matrix Operations}
\date{\today}
\author{Michael Brodskiy\\ \small Professor: Lynn Knight}

% Mathematical Operations:

% Sum: $$\sum_{n=a}^{b} f(x) $$
% Integral: $$\int_{lower}^{upper} f(x) dx$$
% Limit: $$\lim_{x\to\infty} f(x)$$

\begin{document}

\maketitle

\begin{itemize}

  \item Let $\bold{A}$, $\bold{B}$, and $\bold{C}$ be matrices and $c$ and $d$ be constants

    \begin{enumerate}

      \item $\bold{A}+\bold{B}=\bold{B}+\bold{A}$

      \item $\bold{A}+(\bold{B}+\bold{C})=(\bold{A}+\bold{B})+\bold{C}$

      \item $(cd)\bold{A}=c(d\bold{A})$

      \item $1\bold{A}=\bold{A}$

      \item $(c+d)\bold{A}=c\bold{A}+d\bold{A}$

      \item $c(\bold{A}+\bold{B})=c\bold{A}+c\bold{B}$

    \end{enumerate}

  \item Let $\bold{A}$ be a matrix, $c$ be a constant, and $\bold{O}_{mn}$ be a zero matrix

    \begin{enumerate}

      \item $\bold{A}+\bold{O}_{mn}=\bold{A}$

      \item $\bold{A} + (-\bold{A})=\bold{O}_{mn}$

      \item If $c\bold{A}=\bold{O}_{mn}$, then $c=0$ or $\bold{A}=\bold{O}_{mn}$

    \end{enumerate}

  \item Let $\bold{A}$, $\bold{B}$, and $\bold{C}$ be matrices, and $c$ be a constant

    \begin{enumerate}

      \item $\bold{A}(\bold{BC})=(\bold{AB})\bold{C}$

      \item $\bold{A}(\bold{B}+\bold{C})=\bold{AB}+\bold{AC}$

      \item $(\bold{A}+\bold{B})\bold{C}=\bold{AC}+\bold{BC}$

      \item $c(\bold{AB})=(c\bold{A})\bold{B}=\bold{A}(c\bold{B})$

    \end{enumerate}

  \item The transpose of a matrix is formed by interchanging rows and columns ($\bold{A}^T$)

    \begin{enumerate}

      \item $\left( \bold{A}^T \right)^T=\bold{A}$

      \item $\left( \bold{A} + \bold{B} \right)^T=\bold{A}^T + \bold{B}^T$

      \item $(c\bold{A})^T=c\bold{A}^T$

      \item $(\bold{AB})^T=\bold{A}^T\bold{B}^T$

    \end{enumerate}

  \item In a symmetric matrix, $\bold{A}^T=\bold{A}$

\end{itemize}

\end{document}

