%%%%%%%%%%%%%%%%%%%%%%%%%%%%%%%%%%%%%%%%%%%%%%%%%%%%%%%%%%%%%%%%%%%%%%%%%%%%%%%%%%%%%%%%%%%%%%%%%%%%%%%%%%%%%%%%%%%%%%%%%%%%%%%%%%%%%%%%%%%%%%%%%%%%%%%%%%%%%%%%%%%%%%%%%%%%%%%%%%%%%%%%%%%%
% Written By Michael Brodskiy
% Class: Linear Algebra (MATH-194)
% Professor: L. Knight
%%%%%%%%%%%%%%%%%%%%%%%%%%%%%%%%%%%%%%%%%%%%%%%%%%%%%%%%%%%%%%%%%%%%%%%%%%%%%%%%%%%%%%%%%%%%%%%%%%%%%%%%%%%%%%%%%%%%%%%%%%%%%%%%%%%%%%%%%%%%%%%%%%%%%%%%%%%%%%%%%%%%%%%%%%%%%%%%%%%%%%%%%%%%

\documentclass[12pt]{article} 
\usepackage{alphalph}
\usepackage[utf8]{inputenc}
\usepackage[russian,english]{babel}
\usepackage{titling}
\usepackage{amsmath}
\usepackage{graphicx}
\usepackage{enumitem}
\usepackage{amssymb}
\usepackage[super]{nth}
\usepackage{everysel}
\usepackage{ragged2e}
\usepackage{geometry}
\usepackage{fancyhdr}
\usepackage{cancel}
\usepackage{siunitx}
\geometry{top=1.0in,bottom=1.0in,left=1.0in,right=1.0in}
\newcommand{\subtitle}[1]{%
  \posttitle{%
    \par\end{center}
    \begin{center}\large#1\end{center}
    \vskip0.5em}%

}
\usepackage{hyperref}
\hypersetup{
colorlinks=true,
linkcolor=blue,
filecolor=magenta,      
urlcolor=blue,
citecolor=blue,
}

\urlstyle{same}


\title{Kernel and Range}
\date{\today}
\author{Michael Brodskiy\\ \small Professor: Lynn Knight}

% Mathematical Operations:

% Sum: $$\sum_{n=a}^{b} f(x) $$
% Integral: $$\int_{lower}^{upper} f(x) dx$$
% Limit: $$\lim_{x\to\infty} f(x)$$

\begin{document}

\maketitle

\begin{itemize}

  \item Let $T:\,V\rightarrow W$ be a linear transformation. Then the set of all vectors $\overrightarrow{v}$ in $V$ that satisfy $T(\overrightarrow{v})=\bold{0}$ is the kernel of $T$ and is denoted by ker$(T)$ or ker$(T)=\left\{ \overrightarrow{v}\in V\, \Big| \, T(\overrightarrow{v})=\bold{0} \right\}$

  \item $T:\,\mathbb{R}^n\rightarrow\mathbb{R}^m$ matrix transformation, then $T(\overrightarrow{v})=A\overrightarrow{v}$, then ker$(T)=A\overrightarrow{v}=\bold{0}$

  \item Let $T:\, V\rightarrow W$ is a linear transformation, then ker$(T)$ is a subspace of $V$

  \item Range — Let $T:\,V\rightarrow W$ be a linear transformation. The set of all vectors in $W$ that are images under $T$ of vectors in $V$ are called the Range of $T$. range$(T)=\left\{ \overrightarrow{w}\in W\,\Big|\, T(\overrightarrow{v})=\overrightarrow{w}, \overrightarrow{v}\in W \right\}$

  \item Let $T:\,\mathbb{R}^n\rightarrow\mathbb{R}^m$ be a linear transformation defined by $T(\overrightarrow{v})=A\overrightarrow{v}$, then the range$(T)=$ the column space of $A$

  \item Rank and Nullity — Let $T:\,V\rightarrow W$ be a linear transformation. Then nullity$(T)=$ dim(ker$(T)$), rank$(T)=$ dim(range($T$)), and rank$(T)+$ nullity$(T)=$ dim$(V)$

  \item rank$(T)+$ nullity$(T)=n$

\end{itemize}

\end{document}

