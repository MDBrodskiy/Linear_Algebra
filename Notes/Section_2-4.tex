%%%%%%%%%%%%%%%%%%%%%%%%%%%%%%%%%%%%%%%%%%%%%%%%%%%%%%%%%%%%%%%%%%%%%%%%%%%%%%%%%%%%%%%%%%%%%%%%%%%%%%%%%%%%%%%%%%%%%%%%%%%%%%%%%%%%%%%%%%%%%%%%%%%%%%%%%%%%%%%%%%%%%%%%%%%%%%%%%%%%%%%%%%%%
% Written By Michael Brodskiy
% Class: Linear Algebra (MATH-194)
% Professor: L. Knight
%%%%%%%%%%%%%%%%%%%%%%%%%%%%%%%%%%%%%%%%%%%%%%%%%%%%%%%%%%%%%%%%%%%%%%%%%%%%%%%%%%%%%%%%%%%%%%%%%%%%%%%%%%%%%%%%%%%%%%%%%%%%%%%%%%%%%%%%%%%%%%%%%%%%%%%%%%%%%%%%%%%%%%%%%%%%%%%%%%%%%%%%%%%%

\documentclass[12pt]{article} 
\usepackage{alphalph}
\usepackage[utf8]{inputenc}
\usepackage[russian,english]{babel}
\usepackage{titling}
\usepackage{amsmath}
\usepackage{graphicx}
\usepackage{enumitem}
\usepackage{amssymb}
\usepackage[super]{nth}
\usepackage{everysel}
\usepackage{ragged2e}
\usepackage{geometry}
\usepackage{fancyhdr}
\usepackage{cancel}
\usepackage{siunitx}
\geometry{top=1.0in,bottom=1.0in,left=1.0in,right=1.0in}
\newcommand{\subtitle}[1]{%
  \posttitle{%
    \par\end{center}
    \begin{center}\large#1\end{center}
    \vskip0.5em}%

}
\usepackage{hyperref}
\hypersetup{
colorlinks=true,
linkcolor=blue,
filecolor=magenta,      
urlcolor=blue,
citecolor=blue,
}

\urlstyle{same}


\title{Elementary Matrices and Row Equivalence}
\date{\today}
\author{Michael Brodskiy\\ \small Professor: Lynn Knight}

% Mathematical Operations:

% Sum: $$\sum_{n=a}^{b} f(x) $$
% Integral: $$\int_{lower}^{upper} f(x) dx$$
% Limit: $$\lim_{x\to\infty} f(x)$$

\begin{document}

\maketitle

\begin{itemize}

  \item To interchange $R_1$ and $R_2$, one could use the matrix $\begin{bmatrix} 0 & 1 & 0\\ 1 & 0 & 0 \\ 0 & 0 & 1\\  \end{bmatrix}$. This is called an elementary matrix, which is obtained with exactly one operation on $\bold{I}$

  \item To multiply $R_1$ by 2, one could use the matrix $\begin{bmatrix} 2 & 0 & 0\\ 0 & 1 & 0 \\ 0 & 0 & 1\\  \end{bmatrix}$.
    
  \item To make $R_2$ equal to $R_2-2R_1$, one could use the matrix $\begin{bmatrix} 1 & 0 & 0\\ -2 & 1 & 0 \\ 0 & 0 & 1\\  \end{bmatrix}$.

  \item The inverse of an elementary matrix reverses its operation (for example, if $\bold{E}_1$ subtracts $R_2$ from $R_1$, then $\bold{E}_1^{-1}$ adds $R_2$ to $R_1$)

  \item Fundamental Theorem for Matrices:

    \begin{enumerate}

      \item If $\bold{A}$ is a square matrix of order $n$, then all of the following conditions are equivalent:

        \begin{enumerate}

          \item $\bold{A}$ is invertible

          \item $\bold{A}x=\bold{B}$ has a unique solution for any $n$ by one column matrix $\bold{B}$

          \item Only solution of $\bold{A}x=0$ is the trivial solution $x=0$

          \item $\bold{A}\widetilde{R}\bold{I}$

          \item $\bold{A}$ can be written as the product of elementary matrices

        \end{enumerate}

    \end{enumerate}

\end{itemize}

\end{document}

