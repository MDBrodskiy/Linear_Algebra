%%%%%%%%%%%%%%%%%%%%%%%%%%%%%%%%%%%%%%%%%%%%%%%%%%%%%%%%%%%%%%%%%%%%%%%%%%%%%%%%%%%%%%%%%%%%%%%%%%%%%%%%%%%%%%%%%%%%%%%%%%%%%%%%%%%%%%%%%%%%%%%%%%%%%%%%%%%%%%%%%%%%%%%%%%%%%%%%%%%%%%%%%%%%
% Written By Michael Brodskiy
% Class: Linear Algebra (MATH-194)
% Professor: L. Knight
%%%%%%%%%%%%%%%%%%%%%%%%%%%%%%%%%%%%%%%%%%%%%%%%%%%%%%%%%%%%%%%%%%%%%%%%%%%%%%%%%%%%%%%%%%%%%%%%%%%%%%%%%%%%%%%%%%%%%%%%%%%%%%%%%%%%%%%%%%%%%%%%%%%%%%%%%%%%%%%%%%%%%%%%%%%%%%%%%%%%%%%%%%%%

\documentclass[12pt]{article} 
\usepackage{alphalph}
\usepackage[utf8]{inputenc}
\usepackage[russian,english]{babel}
\usepackage{titling}
\usepackage{amsmath}
\usepackage{graphicx}
\usepackage{enumitem}
\usepackage{amssymb}
\usepackage[super]{nth}
\usepackage{everysel}
\usepackage{ragged2e}
\usepackage{geometry}
\usepackage{fancyhdr}
\usepackage{cancel}
\usepackage{siunitx}
\geometry{top=1.0in,bottom=1.0in,left=1.0in,right=1.0in}
\newcommand{\subtitle}[1]{%
  \posttitle{%
    \par\end{center}
    \begin{center}\large#1\end{center}
    \vskip0.5em}%

}
\usepackage{hyperref}
\hypersetup{
colorlinks=true,
linkcolor=blue,
filecolor=magenta,      
urlcolor=blue,
citecolor=blue,
}

\urlstyle{same}


\title{Linear Transformations}
\date{\today}
\author{Michael Brodskiy\\ \small Professor: Lynn Knight}

% Mathematical Operations:

% Sum: $$\sum_{n=a}^{b} f(x) $$
% Integral: $$\int_{lower}^{upper} f(x) dx$$
% Limit: $$\lim_{x\to\infty} f(x)$$

\begin{document}

\maketitle

\begin{itemize}

  \item $T: V\rightarrow W$

  \item Ex. $T(\overrightarrow{v})=(x+y,y+z)$, $\overrightarrow{v}=(1,2,3)$. In this case, $T(\overrightarrow{v}=(1+2,2+3)\rightarrow(3,5)$

    \item $T(\overrightarrow{u}+\overrightarrow{v})=T(\overrightarrow{u})+T(\overrightarrow{v})$

    \item $T:V\rightarrow W$ is called a linear transformation if:

      \begin{enumerate}

        \item $T(\overrightarrow{u}+\overrightarrow{v})=T(\overrightarrow{u})+T(\overrightarrow{v})\,\,\forall \overrightarrow{u},\overrightarrow{v}\in V$

        \item $T(c\overrightarrow{u}=cT(\overrightarrow{u})\,\,\forall c$

      \end{enumerate}

      \item Matrix Transformation — $T(\overrightarrow{v})=A\overrightarrow{v}$, where $A$ is an $m\times n$ matrix, and $\overrightarrow{v}=n\times1$ and $T: \mathbb{R}^n\rightarrow\mathbb{R}^m$

      \item Linear Operator: $T: \overrightarrow{v}\rightarrow\overrightarrow{v}$

      \item Differential Operator: $C'[a,b]=$ set of all functions whose derivatives are continuous on $[a,b]$. $D_x\cdot C'[a,b]\rightarrow C[a,b]$. $D_x(f)=\frac{d}{dx}(f),\,\,f\in C'[a,b]$

      \item Properties of Linear Transformations:

        \begin{enumerate}

          \item $T(\overrightarrow{0})=\overrightarrow{0}$

          \item $T(-\overrightarrow{u})=-T(\overrightarrow{u})$

          \item $T(\overrightarrow{u}-\overrightarrow{v})=T(\overrightarrow{u})-T(\overrightarrow{v})$

          \item If $\overrightarrow{v}=c_1\overrightarrow{v}_1+c_2\overrightarrow{v}_2+\dots+c_n\overrightarrow{v}_n$, then $T(\overrightarrow{v})=T(c_1\overrightarrow{v}_1+c_2\overrightarrow{v}_2+\dots+c_n\overrightarrow{v}_n)=c_1T(\overrightarrow{v}_1)+c_2T(\overrightarrow{v}_2)+\dots+c_nT(\overrightarrow{v}_n)$

        \end{enumerate}

\end{itemize}

\end{document}

