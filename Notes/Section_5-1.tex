%%%%%%%%%%%%%%%%%%%%%%%%%%%%%%%%%%%%%%%%%%%%%%%%%%%%%%%%%%%%%%%%%%%%%%%%%%%%%%%%%%%%%%%%%%%%%%%%%%%%%%%%%%%%%%%%%%%%%%%%%%%%%%%%%%%%%%%%%%%%%%%%%%%%%%%%%%%%%%%%%%%%%%%%%%%%%%%%%%%%%%%%%%%%
% Written By Michael Brodskiy
% Class: Linear Algebra (MATH-194)
% Professor: L. Knight
%%%%%%%%%%%%%%%%%%%%%%%%%%%%%%%%%%%%%%%%%%%%%%%%%%%%%%%%%%%%%%%%%%%%%%%%%%%%%%%%%%%%%%%%%%%%%%%%%%%%%%%%%%%%%%%%%%%%%%%%%%%%%%%%%%%%%%%%%%%%%%%%%%%%%%%%%%%%%%%%%%%%%%%%%%%%%%%%%%%%%%%%%%%%

\documentclass[12pt]{article} 
\usepackage{alphalph}
\usepackage[utf8]{inputenc}
\usepackage[russian,english]{babel}
\usepackage{titling}
\usepackage{amsmath}
\usepackage{graphicx}
\usepackage{enumitem}
\usepackage{amssymb}
\usepackage[super]{nth}
\usepackage{everysel}
\usepackage{ragged2e}
\usepackage{geometry}
\usepackage{fancyhdr}
\usepackage{cancel}
\usepackage{siunitx}
\geometry{top=1.0in,bottom=1.0in,left=1.0in,right=1.0in}
\newcommand{\subtitle}[1]{%
  \posttitle{%
    \par\end{center}
    \begin{center}\large#1\end{center}
    \vskip0.5em}%

}
\usepackage{hyperref}
\hypersetup{
colorlinks=true,
linkcolor=blue,
filecolor=magenta,      
urlcolor=blue,
citecolor=blue,
}

\urlstyle{same}


\title{Length and Magnitude vectors in $\mathbb{R}^n$}
\date{\today}
\author{Michael Brodskiy\\ \small Professor: Lynn Knight}

% Mathematical Operations:

% Sum: $$\sum_{n=a}^{b} f(x) $$
% Integral: $$\int_{lower}^{upper} f(x) dx$$
% Limit: $$\lim_{x\to\infty} f(x)$$

\begin{document}

\maketitle

\begin{itemize}

  \item The length (norm) — Let $\overrightarrow{v}\in\mathbb{R}^n$ such that $\overrightarrow{v}=\left( v_1,v_2,\dots,v_n \right)$, then $||\overrightarrow{v}||=\sqrt{v_1^2+v_2^2+\dots+v_n^2}$

  \item A unit vector in the direction of $\overrightarrow{u}$ can be found using $\frac{\overrightarrow{u}}{||\overrightarrow{u}||}$

  \item Let $\overrightarrow{v}\in\mathbb{R}^n$ and $c$ be a scalar. Then $||c\overrightarrow{v}||=|c|||\overrightarrow{v}||$

  \item Distance between vectors — For two vectors $\overrightarrow{u}=\left( u_1,u_2,\dots,u_n \right)$ and $\overrightarrow{v}=\left( v_1,v_2,\dots,v_n \right)$, the distance between the two is given by $\sqrt{(u_1-v_1)^2+(u_2-v_2)^2+\dots+(u_n-v_n)^2}$

  \item Properties:

    \begin{enumerate}

      \item $d(\overrightarrow{u},\overrightarrow{v})\geq0$

      \item $d(\overrightarrow{u},\overrightarrow{v})=0$ iff $\overrightarrow{u}=\overrightarrow{v}$

      \item $d(\overrightarrow{u},\overrightarrow{v})=d(\overrightarrow{v},\overrightarrow{u})$

    \end{enumerate}

  \item Dot Product — $\overrightarrow{u}\cdot\overrightarrow{v}=u_1v_1+u_2v_2+\dots+u_nv_n$

  \item Properties:

    \begin{enumerate}

      \item $\overrightarrow{u}\cdot\overrightarrow{v}=\overrightarrow{v}\cdot\overrightarrow{u}$

      \item $\overrightarrow{u}(\overrightarrow{v}+\overrightarrow{w})=\overrightarrow{u}\overrightarrow{v}+\overrightarrow{u}\overrightarrow{w}$

      \item $c(\overrightarrow{u}\overrightarrow{v})=(c\overrightarrow{u})\overrightarrow{v}$

      \item $\overrightarrow{v}\cdot\overrightarrow{v}=||\overrightarrow{v}||^2$

      \item $\overrightarrow{v}\cdot\overrightarrow{v}\geq0$ or $\overrightarrow{v}\cdot\overrightarrow{v}=0$ iff $\overrightarrow{v}=0$

    \end{enumerate}

  \item Angle Between Vectors:
    
    \begin{equation*}
      \cos(\theta)=\frac{\overrightarrow{u}\cdot\overrightarrow{v}}{||\overrightarrow{u}||||\overrightarrow{v}||},\,\,\,0\leq\theta\leq\pi
    \label{1}
  \end{equation}

  \begin{enumerate}

    \item $\overrightarrow{u}\cdot\overrightarrow{v}>0\Rightarrow 0\leq\theta\leq\pi$

    \item $\overrightarrow{u}\cdot\overrightarrow{v}<0\Rightarrow \frac{\pi}{2}\leq\theta\leq\pi$

    \item $\overrightarrow{u}\cdot\overrightarrow{v}=0\Rightarrow \theta=\frac{\pi}{2}$ (orthogonal)

  \end{enumerate}

\end{itemize}

\end{document}

