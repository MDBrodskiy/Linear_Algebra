%%%%%%%%%%%%%%%%%%%%%%%%%%%%%%%%%%%%%%%%%%%%%%%%%%%%%%%%%%%%%%%%%%%%%%%%%%%%%%%%%%%%%%%%%%%%%%%%%%%%%%%%%%%%%%%%%%%%%%%%%%%%%%%%%%%%%%%%%%%%%%%%%%%%%%%%%%%%%%%%%%%%%%%%%%%%%%%%%%%%%%%%%%%%
% Written By Michael Brodskiy
% Class: Linear Algebra (MATH-194)
% Professor: L. Knight
%%%%%%%%%%%%%%%%%%%%%%%%%%%%%%%%%%%%%%%%%%%%%%%%%%%%%%%%%%%%%%%%%%%%%%%%%%%%%%%%%%%%%%%%%%%%%%%%%%%%%%%%%%%%%%%%%%%%%%%%%%%%%%%%%%%%%%%%%%%%%%%%%%%%%%%%%%%%%%%%%%%%%%%%%%%%%%%%%%%%%%%%%%%%

\documentclass[12pt]{article} 
\usepackage{alphalph}
\usepackage[utf8]{inputenc}
\usepackage[russian,english]{babel}
\usepackage{titling}
\usepackage{amsmath}
\usepackage{graphicx}
\usepackage{enumitem}
\usepackage{amssymb}
\usepackage[super]{nth}
\usepackage{everysel}
\usepackage{ragged2e}
\usepackage{geometry}
\usepackage{fancyhdr}
\usepackage{cancel}
\usepackage{siunitx}
\geometry{top=1.0in,bottom=1.0in,left=1.0in,right=1.0in}
\newcommand{\subtitle}[1]{%
  \posttitle{%
    \par\end{center}
    \begin{center}\large#1\end{center}
    \vskip0.5em}%

}
\usepackage{hyperref}
\hypersetup{
colorlinks=true,
linkcolor=blue,
filecolor=magenta,      
urlcolor=blue,
citecolor=blue,
}

\urlstyle{same}


\title{Vector Spaces}
\date{\today}
\author{Michael Brodskiy\\ \small Professor: Lynn Knight}

% Mathematical Operations:

% Sum: $$\sum_{n=a}^{b} f(x) $$
% Integral: $$\int_{lower}^{upper} f(x) dx$$
% Limit: $$\lim_{x\to\infty} f(x)$$

\begin{document}

\maketitle

\begin{itemize}

  \item $\mathbb{R}^2$, $\mathbb{R}^3$, \dots, $\mathbb{R}^n$ are all vector spaces with certain properties.

  \item Properties of Vector Spaces, where $\bold{V}$ is a set on which vector addition and scalar multiplication are defined, and $\overrightarrow{\bold{u}}$, $\overrightarrow{\bold{v}}$, and $\overrightarrow{\bold{w}} \in \bold{V}$ and $c$ and $d$ are scalars, then $\bold{V}$ is a vector space if:

    \begin{enumerate}

      \item $\overrightarrow{\bold{u}} + \overrightarrow{\bold{v}} \in \bold{V}$

      \item $\overrightarrow{\bold{u}} + \overrightarrow{\bold{v}} = \overrightarrow{\bold{v}} + \overrightarrow{\bold{u}}$

      \item $\overrightarrow{\bold{u}} + (\overrightarrow{\bold{v}} + \overrightarrow{\bold{w}})=(\overrightarrow{\bold{u}} + \overrightarrow{\bold{v}}) + \overrightarrow{\bold{w}}$

      \item  $\bold{V}$ has a zero vector for every $\overrightarrow{\bold{u}}$ such that $\overrightarrow{\bold{u}} + \bold{0} =\overrightarrow{\bold{u}}$

      \item For every $\overrightarrow{\bold{u}}$ in $\bold{V}$ there is a vector such that $\overrightarrow{\bold{u}} + (-\overrightarrow{\bold{u}}=0$

      \item $c\overrightarrow{\bold{u}} \in \bold{V}$ 

      \item $c(\overrightarrow{\bold{u}} + \overrightarrow{\bold{v}})= c\overrightarrow{\bold{u}}+c\overrightarrow{\bold{v}}$

      \item $(c+d)\overrightarrow{\bold{u}}=c\overrightarrow{\bold{u}}+d\overrightarrow{\bold{u}}$

      \item $c(d\overrightarrow{\bold{u}})=(cd)\overrightarrow{\bold{u}}$

      \item $1(\overrightarrow{\bold{u}})=\overrightarrow{\bold{u}}$

    \end{enumerate}

  \item Polynomials of Degree $n$

    \begin{enumerate}

      \item $P_1(x)=\left\{ ax+b \big| a,b\in\mathbb{R} \right\}$ Like in $\mathbb{R}^2$

      \item $P_2(x)=\left\{ ax^2+bx+c \big| a,b,c\in\mathbb{R} \right\}$ Like in $\mathbb{R}^3$

      \item $P_n(x)=\left\{ a_0+a_1x+a_2x^2+\dots+a_nx^n \big| a_i\in\mathbb{R} \right\}$ Like in $\mathbb{R}^n$

      \item $P_0(x)=\left\{ a \big| a\in\mathbb{R} \right\}$ Like in $\mathbb{R}^1$

    \end{enumerate}

    \newpage

  \item Standard Vector Spaces

    \begin{enumerate}

      \item $\mathbb{R}$ = set of all real numbers

      \item $\mathbb{R}^2$ = set of all ordered pairs

      \item $\mathbb{R}^3$ = set of all ordered triples

      \item $\mathbb{R}^n$ = set of all $n$-tuples

      \item $C(-\infty,\infty)$ = set of all continuous functions defined on the real number line

      \item $C[a,b]$ = set of all continuous functions defined on a closed interval $[a,b]$, where $a\neq b$

      \item $P$ = set of all polynomials

      \item $P_n$ = set of all polynomials of degree $\leq n$ (together with the zero polynomial)

      \item $M_{m,n}$ = set of all $m\times n$ matrices

      \item $M_{n,n}$ = set of all $n\times n$ matrices

    \end{enumerate}

  \item Let $\overrightarrow{\bold{v}}$ be any element of a vector space $\bold{V}$, and let $c$ be any scalar. Then the properties below are true.

    \begin{enumerate}

      \item $0\overrightarrow{\bold{v}}=\bold{0}$

      \item $c\bold{0}=\bold{0}$

      \item If $c\overrightarrow{\bold{v}}=0$, then $c=0$ or $\overrightarrow{\bold{v}}=0$

      \item $(-1)\overrightarrow{\bold{v}}=-\overrightarrow{\bold{v}}$

    \end{enumerate}

\end{itemize}

\end{document}

