%%%%%%%%%%%%%%%%%%%%%%%%%%%%%%%%%%%%%%%%%%%%%%%%%%%%%%%%%%%%%%%%%%%%%%%%%%%%%%%%%%%%%%%%%%%%%%%%%%%%%%%%%%%%%%%%%%%%%%%%%%%%%%%%%%%%%%%%%%%%%%%%%%%%%%%%%%%%%%%%%%%%%%%%%%%%%%%%%%%%%%%%%%%%
% Written By Michael Brodskiy
% Class: Linear Algebra (MATH-194)
% Professor: L. Knight
%%%%%%%%%%%%%%%%%%%%%%%%%%%%%%%%%%%%%%%%%%%%%%%%%%%%%%%%%%%%%%%%%%%%%%%%%%%%%%%%%%%%%%%%%%%%%%%%%%%%%%%%%%%%%%%%%%%%%%%%%%%%%%%%%%%%%%%%%%%%%%%%%%%%%%%%%%%%%%%%%%%%%%%%%%%%%%%%%%%%%%%%%%%%

\documentclass[12pt]{article} 
\usepackage{alphalph}
\usepackage[utf8]{inputenc}
\usepackage[russian,english]{babel}
\usepackage{titling}
\usepackage{amsmath}
\usepackage{graphicx}
\usepackage{enumitem}
\usepackage{amssymb}
\usepackage[super]{nth}
\usepackage{everysel}
\usepackage{ragged2e}
\usepackage{geometry}
\usepackage{fancyhdr}
\usepackage{cancel}
\usepackage{siunitx}
\geometry{top=1.0in,bottom=1.0in,left=1.0in,right=1.0in}
\newcommand{\subtitle}[1]{%
  \posttitle{%
    \par\end{center}
    \begin{center}\large#1\end{center}
    \vskip0.5em}%

}
\usepackage{hyperref}
\hypersetup{
colorlinks=true,
linkcolor=blue,
filecolor=magenta,      
urlcolor=blue,
citecolor=blue,
}

\urlstyle{same}


\title{Subspaces of Vector Spaces}
\date{\today}
\author{Michael Brodskiy\\ \small Professor: Lynn Knight}

% Mathematical Operations:

% Sum: $$\sum_{n=a}^{b} f(x) $$
% Integral: $$\int_{lower}^{upper} f(x) dx$$
% Limit: $$\lim_{x\to\infty} f(x)$$

\begin{document}

\maketitle

\begin{itemize}

  \item A non-empty subset $\bold{W}$ of vector space $\bold{V}$ is called a subspace of $\bold{V}$ if $\bold{W}$ is itself a vector space under addition and scalar multiplication defined on $\bold{V}$

    \begin{enumerate}

      \item If $\overrightarrow{\bold{u}}$ and $\overrightarrow{\bold{v}}$ are in $\bold{W}$, then $\overrightarrow{\bold{u}} + \overrightarrow{\bold{v}}$ is in $\bold{W}$

      \item If $\overrightarrow{\bold{u}}$ is in $\bold{W}$, and $c$ is any scalar, then $c\overrightarrow{\bold{u}}$ is in $\bold{W}$

      \item Ex. $\mathbb{R}^2$ is a vector space, then $\left\{ (x,2x) \right\}$ is a subspace of $\mathbb{R}^2$, but $\left\{ (x,2x+1 \right\}$ is not

    \end{enumerate}

  \item Verify that $\bold{W}$ is a subspace of $\bold{V}$

    \begin{enumerate}

      \item $\bold{W}\leq \bold{V}$

      \item $\bold{W}$ not empty

      \item $\overrightarrow{\bold{u}} + \overrightarrow{\bold{v}} \in \bold{W}$

      \item $c\overrightarrow{\bold{u}} \in \bold{W}$

    \end{enumerate}

\end{itemize}

\end{document}

