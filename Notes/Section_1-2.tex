%%%%%%%%%%%%%%%%%%%%%%%%%%%%%%%%%%%%%%%%%%%%%%%%%%%%%%%%%%%%%%%%%%%%%%%%%%%%%%%%%%%%%%%%%%%%%%%%%%%%%%%%%%%%%%%%%%%%%%%%%%%%%%%%%%%%%%%%%%%%%%%%%%%%%%%%%%%%%%%%%%%%%%%%%%%%%%%%%%%%%%%%%%%%
% Written By Michael Brodskiy
% Class: Linear Algebra (MATH-194)
% Professor: L. Knight
%%%%%%%%%%%%%%%%%%%%%%%%%%%%%%%%%%%%%%%%%%%%%%%%%%%%%%%%%%%%%%%%%%%%%%%%%%%%%%%%%%%%%%%%%%%%%%%%%%%%%%%%%%%%%%%%%%%%%%%%%%%%%%%%%%%%%%%%%%%%%%%%%%%%%%%%%%%%%%%%%%%%%%%%%%%%%%%%%%%%%%%%%%%%

\documentclass[12pt]{article} 
\usepackage{alphalph}
\usepackage[utf8]{inputenc}
\usepackage[russian,english]{babel}
\usepackage{titling}
\usepackage{amsmath}
\usepackage{graphicx}
\usepackage{enumitem}
\usepackage{amssymb}
\usepackage[super]{nth}
\usepackage{everysel}
\usepackage{ragged2e}
\usepackage{geometry}
\usepackage{fancyhdr}
\usepackage{cancel}
\usepackage{siunitx}
\geometry{top=1.0in,bottom=1.0in,left=1.0in,right=1.0in}
\newcommand{\subtitle}[1]{%
  \posttitle{%
    \par\end{center}
    \begin{center}\large#1\end{center}
    \vskip0.5em}%

}
\usepackage{hyperref}
\hypersetup{
colorlinks=true,
linkcolor=blue,
filecolor=magenta,      
urlcolor=blue,
citecolor=blue,
}

\urlstyle{same}


\title{Augmented Matrices and Elementary Row Operations}
\date{\today}
\author{Michael Brodskiy\\ \small Professor: Lynn Knight}

% Mathematical Operations:

% Sum: $$\sum_{n=a}^{b} f(x) $$
% Integral: $$\int_{lower}^{upper} f(x) dx$$
% Limit: $$\lim_{x\to\infty} f(x)$$

\begin{document}

\maketitle

\begin{itemize}

  \item Matrix $-$ An ordered array of numbers

  \item Examples of a Matrix

    \begin{enumerate}

      \item $\begin{bmatrix} 1 & 2\\ 3 & 4\\ \end{bmatrix}$ (2x2 Matrix)
            
      \item $\begin{bmatrix} 9 & 0 & e\\ \end{bmatrix}$ (1x3 Matrix)

      \item $\begin{bmatrix} 2 & 4 \\ \pi & 6 \\ 3 & 0 \end{bmatrix}$ (3x2 Matrix)

      \item Augmented Matrix: $\left[\begin{array}{c c | c} 1 & -2 & 6 \\ 2 & 3 & -2 \end{array}\right]$

    \end{enumerate}

  \item Elementary Row Operations:

    \begin{enumerate}

      \item Interchange Rows ($R_i \leftrightarrow R_j$)

      \item Multiply a Row by a non-zero Constant ($kR_i\rightarrow R_i$)

      \item Add a Multiple of one Row to Another ($kR_i+R_j\rightarrow R_j$)

      \item Notation: $\widetilde{R}$ (Row Equivalent)

    \end{enumerate}

  \item Solving (Get $\widetilde{R}$ so that some entries are 0s)

    \newpage

  \item Example: $\begin{array}{c} x-2y=6 \\ 2x+3y=-2 \\\end{array}$

    \begin{equation*}
      \begin{split}
        \left[ \begin{array}{c c | c} 1 & -2 & 6 \\ 2 & 3 & -2 \end{array}  \right]\\
        \widetilde{-2R_1\rightarrow R_1}\\
        \left[ \begin{array}{c c | c} -2 & 4 & -12 \\ 2 & 3 & -2 \end{array}  \right]\\
        \widetilde{R_1+R_2\rightarrow R_2} \text{ and } \widetilde{-\frac{1}{2}R_1\rightarrow R_1}\\
        \left[ \begin{array}{c c | c} 1 & -2 & 6 \\ 0 & 7 & -14 \end{array}  \right]\\
        \widetilde{\frac{1}{7}R_2\rightarrow R_2}\\
        \left[ \begin{array}{c c | c} 1 & -2 & 6 \\ 0 & 1 & -2 \end{array}  \right]\\
        y=-2\\
        x-2(-2)=6\longrightarrow 2\\
        S=\left\{ (2,-2) \right\}
      \end{split}
      \label{1}
    \end{equation}

  \item Organization

    \begin{enumerate}

      \item $\begin{array}{c c} a_{11}x_1+a_{12}x_2+a_{13}x_3=c_1 & (E_1) \\ a_{21}x_1+a_{22}x_2+a_{23}x_3=c_2 & (E_2)\\ a_{31}x_1+a_{32}x_2+a_{33}x_3=c_3 & (E_3)\end{array}$

      \item $\left[\begin{array}{c c c | c} a_{11} & a_{12} & a_{13} & c_1 \\ a_{21} & a_{22} & a_{23} & c_2\\ a_{31} & a_{32} & a_{33} & c_3 \end{array}\right]$

    \end{enumerate}

  \item Example: $\begin{array}{c} x_1-2x_2+x_3=-1 \\ 2x_1-3x_2+x_3=-4 \\ 3x_1-4x_2+2x_3=-3\end{array}$

    \begin{equation*}
      \begin{split}
        \left[\begin{array}{c c c | c} 1 & -2 & 1 & -1 \\ 2 & -3 & 1 & -4\\ 3 & -4 & 2 & -3 \end{array}\right]\\
        \widetilde{-2R_1+R_2\rightarrow R_2}\\
        \widetilde{-3R_1+R_3\rightarrow R_3}\\
        \widetilde{-2R_2+R_3\rightarrow R_3}\\
        \left[\begin{array}{c c c | c} 1 & -2 & 1 & -1 \\ 0 & 1 & -1 & -2\\ 0 & 0 & 1 & 4 \end{array}\right]\\
        x_3=4\\
        x_2=(4)-2\rightarrow2\\
        x_1=2(2)-4-1\rightarrow-1\\
        S=\left\{ (-1,2,4) \right\}
      \end{split}
      \label{2}
    \end{equation}

  \item Gaussian Elimination and Gauss Jordan Elimination

    \begin{enumerate}

      \item Reduced Row-Echelon Form

        \begin{enumerate}

      \item If a row does not consist entirely of 0s, and the first non-zero element in the row is a 1, then it is called a leading one.

      \item For any two successive nonzero rows, the leading 1 in the lower row is farther to the right than the leading 1 in the higher row. 

      \item All the rows consisting entirely of 0�s are at the bottom of the matrix. If the fourth property is also satisfied, a matrix is said to be in reduced row-echelon form:

      \item Each column that contains a leading 1 has 0�s everywhere else (above and below)

      \item If properties $a-c$ are met, but $d$ is not, then the matrix is in just Row Echelon Form

    \end{enumerate}

  \item A Matrix in Reduced Row-Echelon Form: $\left[\begin{array}{c c c | c} 1 & 0 & 0 & -1 \\ 0 & 1 & 0 & -4\\ 0 & 0 & 1 & -3 \end{array}\right]\\$


    \end{enumerate}

  \item For Gaussian Elimination

    \begin{enumerate}

      \item Put Augmented Matrix in Row-Echelon Form

      \item Back Substitution

    \end{enumerate}

  \item For Gauss-Jordan Elimination

    \begin{enumerate}

      \item Put Augmented Matrix in Reduced Row-Echelon Form

      \item Back Substitution

    \end{enumerate}

\end{itemize}

\end{document}

