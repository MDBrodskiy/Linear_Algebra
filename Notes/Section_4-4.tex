%%%%%%%%%%%%%%%%%%%%%%%%%%%%%%%%%%%%%%%%%%%%%%%%%%%%%%%%%%%%%%%%%%%%%%%%%%%%%%%%%%%%%%%%%%%%%%%%%%%%%%%%%%%%%%%%%%%%%%%%%%%%%%%%%%%%%%%%%%%%%%%%%%%%%%%%%%%%%%%%%%%%%%%%%%%%%%%%%%%%%%%%%%%%
% Written By Michael Brodskiy
% Class: Linear Algebra (MATH-194)
% Professor: L. Knight
%%%%%%%%%%%%%%%%%%%%%%%%%%%%%%%%%%%%%%%%%%%%%%%%%%%%%%%%%%%%%%%%%%%%%%%%%%%%%%%%%%%%%%%%%%%%%%%%%%%%%%%%%%%%%%%%%%%%%%%%%%%%%%%%%%%%%%%%%%%%%%%%%%%%%%%%%%%%%%%%%%%%%%%%%%%%%%%%%%%%%%%%%%%%

\documentclass[12pt]{article} 
\usepackage{alphalph}
\usepackage[utf8]{inputenc}
\usepackage[russian,english]{babel}
\usepackage{titling}
\usepackage{amsmath}
\usepackage{graphicx}
\usepackage{enumitem}
\usepackage{amssymb}
\usepackage[super]{nth}
\usepackage{everysel}
\usepackage{ragged2e}
\usepackage{geometry}
\usepackage{fancyhdr}
\usepackage{cancel}
\usepackage{siunitx}
\geometry{top=1.0in,bottom=1.0in,left=1.0in,right=1.0in}
\newcommand{\subtitle}[1]{%
  \posttitle{%
    \par\end{center}
    \begin{center}\large#1\end{center}
    \vskip0.5em}%

}
\usepackage{hyperref}
\hypersetup{
colorlinks=true,
linkcolor=blue,
filecolor=magenta,      
urlcolor=blue,
citecolor=blue,
}

\urlstyle{same}


\title{Spanning Sets and Linear Independence}
\date{\today}
\author{Michael Brodskiy\\ \small Professor: Lynn Knight}

% Mathematical Operations:

% Sum: $$\sum_{n=a}^{b} f(x) $$
% Integral: $$\int_{lower}^{upper} f(x) dx$$
% Limit: $$\lim_{x\to\infty} f(x)$$

\begin{document}

\maketitle

\begin{itemize}

  \item Linear Combinations $-$ If $\overrightarrow{\bold{w}}=a\overrightarrow{\bold{u}}+b\overrightarrow{\bold{v}}$, then $\overrightarrow{\bold{w}}$ is a linear combination of $\overrightarrow{\bold{u}}$ and $\overrightarrow{\bold{v}}$. $\overrightarrow{\bold{w}}=a\overrightarrow{\bold{u}}+b\overrightarrow{\bold{v}}$ is a plane (it spans a plane).

  \item Definition: Let $\overrightarrow{\bold{v}}\in\bold{V}$. Then $\overrightarrow{\bold{v}}$ is a linear combination of $\overrightarrow{\bold{u}}_1, \overrightarrow{\bold{u}}_2, \dots,\overrightarrow{\bold{u}}_n$ if $\exists$ scalars $c_1,c_2,\dots,c_n$ such that $\overrightarrow{\bold{v}}=c_1\overrightarrow{\bold{u}}_1+c_2\overrightarrow{\bold{u}}_2+\dots+c_n\overrightarrow{\bold{u}}_n$

    \begin{enumerate}

      \item ex. $\left[ \begin{array}{cc} a & b\\ c & d\\ \end{array}\right]$ is a combination of $a\left[ \begin{array}{cc} 1 & 0\\ 0 & 0  \end{array} \right]+b\left[ \begin{array}{cc} 0 & 1\\ 0 & 0  \end{array} \right]+c\left[ \begin{array}{cc} 0 & 0\\ 1 & 0\\ \end{array}\right] + d\left[ \begin{array}{cc} 0 & 0\\ 0 & 1 \end{array}\right]$ (other solutions exist)

    \end{enumerate}

  \item The Span of a set of vectors: Let $S=\left\{ \overrightarrow{\bold{v}}_1,\overrightarrow{\bold{v}}_2,\dots,\overrightarrow{\bold{v}}_n \right\}$. Then, the span of $S$ is a set of all linear combinations of vectors in $S$ (i.e. span($S$)$=\left\{ c_1\overrightarrow{\bold{v}}_1+c_2\overrightarrow{\bold{v}}_2+\dots+\overrightarrow{\bold{v}}_n \right\}$). Note: when span($S$)$=\bold{V}$, it means $\bold{V}$ is spanned by $\overrightarrow{\bold{v}}_1,\overrightarrow{\bold{v}}_2,\dots,\overrightarrow{\bold{v}}_n$

    \begin{enumerate}

      \item ex. $\mathbb{R}^2=\text{span}\left\{ \hat{i},\hat{j} \right\}, \mathbb{R}^3=\text{span}\left\{ \hat{i},\hat{j},\hat{k} \right\}$

    \end{enumerate}

  \item The span of $S$ is always a subspace of $\bold{V}$, because closure is automatic.

  \item Linear Independence $-$ A set of vectors $S=\left\{ \overrightarrow{\bold{v}}_1,\overrightarrow{\bold{v}}_2,\dots,\overrightarrow{\bold{v}}_n \right\}$ in vector space $\bold{V}$ is linearly independent if vector equation $c_1\overrightarrow{\bold{v}}_1+c_2\overrightarrow{\bold{v}}_2+\dots+c_n\overrightarrow{\bold{v}}_n=0$ has trivial solution only ($c_1=c_2=c_n=0$). Otherwise, it is linearly dependent.

    \begin{enumerate}

      \item ex. in $\mathbb{R}^2$: $\begin{array}{c} 2x+3y=1\\ 4x+6y=2\end{array}$

      \item ex. in $\mathbb{R}^3$: $\overrightarrow{\bold{u}}_1=\langle2,1,0\rangle,\overrightarrow{\bold{u}}_2=\langle 3,5,-2\rangle,\overrightarrow{\bold{u}}_3=\langle 5,6,-2\rangle$, then $\overrightarrow{\bold{u}}_3=\overrightarrow{\bold{u}}_2+\overrightarrow{\bold{u}}_1$

    \end{enumerate}

  \item $S=\left\{ \overrightarrow{\bold{v}}_1,\overrightarrow{\bold{v}}_2,\dots,\overrightarrow{\bold{v}}_k \right\}, k\geq 2$ is linearly dependent iff $\overrightarrow{\bold{v}}_i$ can be written as a linear combination of other vectors in $S$.

    \begin{enumerate}

      \item $\left\{ \overrightarrow{\bold{o}} \right\}$ is linearly dependent

      \item $\left\{ \overrightarrow{\bold{v}}_1,\overrightarrow{\bold{v}}_2 \right\}$ is linearly dependent iff $\overrightarrow{\bold{v}}_2=c\overrightarrow{\bold{v}}_1$
        
      \item ex. Will 4 vectors in $\mathbb{R}^3$ be linearly independent? No, they will always be dependent.

      \item ex. Can 4 vectors span $\mathbb{R}^3$? Yes, they can span $\mathbb{R}^3.

    \end{enumerate}

\end{itemize}

\end{document}

