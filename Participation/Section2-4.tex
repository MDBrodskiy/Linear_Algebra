%%%%%%%%%%%%%%%%%%%%%%%%%%%%%%%%%%%%%%%%%%%%%%%%%%%%%%%%%%%%%%%%%%%%%%%%%%%%%%%%%%%%%%%%%%%%%%%%%%%%%%%%%%%%%%%%%%%%%%%%%%%%%%%%%%%%%%%%%%%%%%%%%%%%%%%%%%%%%%%%%%%%%%%%%%%%%%%%%%%%%%%%%%%%
% Written By Michael Brodskiy
% Class: Linear Algebra
% Professor: L. Knight
%%%%%%%%%%%%%%%%%%%%%%%%%%%%%%%%%%%%%%%%%%%%%%%%%%%%%%%%%%%%%%%%%%%%%%%%%%%%%%%%%%%%%%%%%%%%%%%%%%%%%%%%%%%%%%%%%%%%%%%%%%%%%%%%%%%%%%%%%%%%%%%%%%%%%%%%%%%%%%%%%%%%%%%%%%%%%%%%%%%%%%%%%%%%

\documentclass[12pt]{article} 
\usepackage{alphalph}
\usepackage[utf8]{inputenc}
\usepackage[russian,english]{babel}
\usepackage{titling}
\usepackage{amsmath}
\usepackage{graphicx}
\usepackage{enumitem}
\usepackage{amssymb}
\usepackage{physics}
\usepackage{tikz}
\usepackage{mathdots}
\usepackage{yhmath}
\usepackage{cancel}
\usepackage{color}
\usepackage{siunitx}
\usepackage{array}
\usepackage{multirow}
\usepackage{gensymb}
\usepackage{tabularx}
\usepackage{booktabs}
\usetikzlibrary{fadings}
\usetikzlibrary{patterns}
\usetikzlibrary{shadows.blur}
\usetikzlibrary{shapes}
\usepackage[super]{nth}
\usepackage{expl3}
\usepackage[version=4]{mhchem}
\usepackage{hpstatement}
\usepackage{rsphrase}
\usepackage{everysel}
\usepackage{ragged2e}
\usepackage{geometry}
\usepackage{fancyhdr}
\usepackage{cancel}
\geometry{top=1.0in,bottom=1.0in,left=1.0in,right=1.0in}
\newcommand{\subtitle}[1]{%
  \posttitle{%
    \par\end{center}
    \begin{center}\large#1\end{center}
    \vskip0.5em}%

}
\usepackage{hyperref}
\hypersetup{
colorlinks=true,
linkcolor=blue,
filecolor=magenta,      
urlcolor=blue,
citecolor=blue,
}

\urlstyle{same}


\title{Linear Algebra 2.4 Participation Problem}
\date{February 17, 2021}
\author{Michael Brodskiy\\ \small Instructor: Prof. Knight}

% Mathematical Operations:

% Sum: $$\sum_{n=a}^{b} f(x) $$
% Integral: $$\int_{lower}^{upper} f(x) dx$$
% Limit: $$\lim_{x\to\infty} f(x)$$

\begin{document}

\maketitle

\begin{enumerate}[label=\alph*)]

  \item Write $\bold{A}=\left[ \begin{array}{c c} 4 & 5\\ 1 & 2 \end{array}\right]$ as a product of elementary matrices

    \begin{equation*}
      \begin{split}
        E_1=\left[ \begin{array}{c c} 1 & 0\\ 0 & 1  \end{array} \right]\\
        E_2=\left[ \begin{array}{c c} 1 & 1\\ 0 & 1  \end{array} \right]\\
        E_3=\left[ \begin{array}{c c} 1 & 0\\ 1 & 1  \end{array} \right]\\
        E_4=\left[ \begin{array}{c c} 3 & 0\\ 0 & 1  \end{array} \right]\\
        E_5=\left[ \begin{array}{c c} 1 & 1\\ 0 & 1  \end{array} \right]\\
        E_5E_4E_3E_2E_1=\left[ \begin{array}{c c} 4 & 5\\ 1 & 2\\ \end{array} \right]
      \end{split}
      \label{1}
    \end{equation}

  \item Write $\bold{A}^{-1}$ as a product of elementary matrices

    \begin{equation*}
      \begin{split}
        \bold{A}=\left[ \begin{array}{c c} 4 & 5\\ 1 & 2\\ \end{array}\right]\\
        E_1=\left[ \begin{array}{c c} 1 & -1\\ 0 & 1  \end{array} \right]\\
        E_2=\left[ \begin{array}{c c} \frac{1}{3} & 0\\ 0 & 1  \end{array} \right]\\
        E_3=\left[ \begin{array}{c c} 1 & 0\\ -1 & 1  \end{array} \right]\\
        E_4=\left[ \begin{array}{c c} 1 & -1\\ 0 & 1  \end{array} \right]\\
        E_4E_3E_2E_1=\left[ \begin{array}{c c} \frac{2}{3} & -\frac{5}{3}\\ -\frac{1}{3} & \frac{4}{3}  \end{array} \right]\\
        E_4E_3E_2E_1\bold{A}=\left[ \begin{array}{c c} 1 & 0\\ 0 & 1\\ \end{array} \right]\\
        \therefore E_4E_3E_2E_1 = \bold{A}^{-1}
      \end{split}
      \label{2}
    \end{equation}

\end{enumerate}

\end{document}

