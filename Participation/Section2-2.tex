%%%%%%%%%%%%%%%%%%%%%%%%%%%%%%%%%%%%%%%%%%%%%%%%%%%%%%%%%%%%%%%%%%%%%%%%%%%%%%%%%%%%%%%%%%%%%%%%%%%%%%%%%%%%%%%%%%%%%%%%%%%%%%%%%%%%%%%%%%%%%%%%%%%%%%%%%%%%%%%%%%%%%%%%%%%%%%%%%%%%%%%%%%%%
% Written By Michael Brodskiy
% Class: Linear Algebra
% Professor: L. Knight
%%%%%%%%%%%%%%%%%%%%%%%%%%%%%%%%%%%%%%%%%%%%%%%%%%%%%%%%%%%%%%%%%%%%%%%%%%%%%%%%%%%%%%%%%%%%%%%%%%%%%%%%%%%%%%%%%%%%%%%%%%%%%%%%%%%%%%%%%%%%%%%%%%%%%%%%%%%%%%%%%%%%%%%%%%%%%%%%%%%%%%%%%%%%

\documentclass[12pt]{article} 
\usepackage{alphalph}
\usepackage[utf8]{inputenc}
\usepackage[russian,english]{babel}
\usepackage{titling}
\usepackage{amsmath}
\usepackage{graphicx}
\usepackage{enumitem}
\usepackage{amssymb}
\usepackage{physics}
\usepackage{tikz}
\usepackage{mathdots}
\usepackage{yhmath}
\usepackage{cancel}
\usepackage{color}
\usepackage{siunitx}
\usepackage{array}
\usepackage{multirow}
\usepackage{gensymb}
\usepackage{tabularx}
\usepackage{booktabs}
\usetikzlibrary{fadings}
\usetikzlibrary{patterns}
\usetikzlibrary{shadows.blur}
\usetikzlibrary{shapes}
\usepackage[super]{nth}
\usepackage{expl3}
\usepackage[version=4]{mhchem}
\usepackage{hpstatement}
\usepackage{rsphrase}
\usepackage{everysel}
\usepackage{ragged2e}
\usepackage{geometry}
\usepackage{fancyhdr}
\usepackage{cancel}
\geometry{top=1.0in,bottom=1.0in,left=1.0in,right=1.0in}
\newcommand{\subtitle}[1]{%
  \posttitle{%
    \par\end{center}
    \begin{center}\large#1\end{center}
    \vskip0.5em}%

}
\usepackage{hyperref}
\hypersetup{
colorlinks=true,
linkcolor=blue,
filecolor=magenta,      
urlcolor=blue,
citecolor=blue,
}

\urlstyle{same}


\title{Linear Algebra 2.2 Participation Problem}
\date{February 10, 2021}
\author{Michael Brodskiy\\ \small Instructor: Prof. Knight}

% Mathematical Operations:

% Sum: $$\sum_{n=a}^{b} f(x) $$
% Integral: $$\int_{lower}^{upper} f(x) dx$$
% Limit: $$\lim_{x\to\infty} f(x)$$

\begin{document}

\maketitle

\flushleft Solve for $a$, $b$, $c$, and $d$ in the following matrix equation:

\begin{equation*}
  \left[\begin{array}{l l}
    a & b \\
    c & d\\
\end{array}\right]
  \left[\begin{array}{l l}
    2 & 1 \\
    3 & 1\\
\end{array}\right]=
  \left[\begin{array}{l l}
    3 & 17 \\
    4 & -1\\
\end{array}\right]
  \label{1}
\end{equation}

\begin{center}
  \begin{array}{l l}
    2a+3b=3 & a+b=17\\
    2c+3d=4 & c+d=-1
  \end{array}
\end{center}

\begin{equation}
  \begin{split}
  \left[\begin{array}{c c c c | c}
      2 & 3 & 0 & 0 & 3\\
      1 & 1 & 0 & 0 & 17\\
      0 & 0 & 2 & 3 & 4\\
      0 & 0 & 1 & 1 & -1\\
  \end{array}\right]\\
  R_1-R_2\rightarrow R_1\text{ and }R_3-R_4\rightarrow R_3\\
  \left[\begin{array}{c c c c | c}
      1 & 2 & 0 & 0 & -14\\
      1 & 1 & 0 & 0 & 17\\
      0 & 0 & 1 & 2 & 5\\
      0 & 0 & 1 & 1 & -1\\
  \end{array}\right]\\
  R_1-R_2\rightarrow R_2\text{ and }R_3-R_4\rightarrow R_4\\
  \left[\begin{array}{c c c c | c}
      1 & 2 & 0 & 0 & -14\\
      0 & 1 & 0 & 0 & -31\\
      0 & 0 & 1 & 2 & 5\\
      0 & 0 & 0 & 1 & 6\\
  \end{array}\right]\\
\end{split}
  \label{2}
\end{equation}

\begin{center}
  \begin{array}{l}
    a+2b=-14\\
    b=-31\\
    c+2d=5\\
    d=6\\
    a=48\\
    c=-7
  \end{array}
\end{center}

\end{document}

